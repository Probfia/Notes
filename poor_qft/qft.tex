\documentclass[a4paper,11pt]{ctexart}

\usepackage{amsmath}
\usepackage{color}
\usepackage{mathrsfs}
\usepackage[colorlinks,
            linkcolor=blue,
		 urlcolor=black]{hyperref}
\usepackage{graphicx}
\usepackage{cleveref}

\crefname{equation}{}{}
\crefname{figure}{图}{图}
\crefname{footnote}{注释}{注释}

\newcommand{\beq}{\begin{equation}}
\newcommand{\eeq}{\end{equation}}
\newcommand{\bea}{\begin{equation}\begin{aligned}}
\newcommand{\eea}{\end{aligned}\end{equation}}
\newcommand{\red}{\color{red}}
\newcommand{\grandz}{\mathcal{Z}}
\newcommand{\lagden}{\mathcal{L}}
\newcommand{\lag}{\mathcal{L}}
\newcommand{\ham}{\mathcal{H}}
\newcommand{\diag}{\mathrm{diag}}
\newcommand{\emptyline}{\\ \ \\}

%\cpic{<尺寸>}{<文件名>}}用于生成居中的图片。
\newcommand{\cpic}[2]{
\begin{center}
\includegraphics[scale=#1]{#2}
\end{center}
}

%\cpicn{<尺寸>}{<文件名>}{<注释>}用于生成居中且带有注释的图片,其label为图片名。
\newcommand{\cpicn}[3]
{
\begin{figure}[h!]
\cpic{#1}{#2}
\caption{#3\label{#2}}
\end{figure}
}

\title{乞丐版量子场论}
\author{Probfia}
\date{}

\begin{document}
\maketitle
\tableofcontents
\vspace{5mm}
\emph{
全文中我们采取自然单位制$c = \hbar = 1$。度规$g_{\mu,\nu}$的符号是$(+,-,-,-)$,且不特别说明时它都是闵可夫斯基度规$\eta_{\mu \nu} = \diag (1,-1,-1,-1)$。
}
\section{准备工作}
\subsection{单位制和量纲}
因为我们采用$c = \hbar = 1$的单位制,因此,力学量量纲的自由度只有1。我们一般选质量量纲为基表示出其他量的量纲。首先我们有
\beq
\frac{L}{T} = \frac{ML^2}{T} = 1
\eeq
得到
\beq
L = T = M^{-1}
\eeq
通过一个量在自然单位制下的量纲和上述关系,就可以得到一个量在自由单位制下的量纲,例如能量
\beq
\dim E = \frac{ML^2}{T^2} = M
\eeq
和能量密度
\beq
\dim u = \frac{[E]}{L^3} = M^4
\eeq
我们引入记号$[Q] \equiv \log_M \dim Q$为物理量$Q$的量纲指数,很显然有$[u] = 4$等等。

\subsection{经典场论}
在$(D+1)$维经典场论中,标量场$\phi(t,\vec{x})$的拉格朗日密度为$\partial_\mu \phi$和$\phi$的函数,其中$\mu = 0,1,\cdots,D$,0指标代表时间指标。$\partial_\mu \phi$这样的偏导数是一个矢量,第$\mu$个分量是对$x_\mu$求导的结果。其次,定义$\partial^\mu \phi = g^{\mu \nu} \partial_\nu \phi$为它对应的对偶矢量。作用量为
\beq
S[\phi] = \int  \lagden(\partial_\mu \phi,\phi)dt d^D\vec{x}
\eeq
这个作用量对应的拉格朗日方程可以由以前在理论力学里学过的拉格朗日方程推广而来,这里$q$对应$\phi$,$t$对应$x^\mu$,$\dot{q}$对应$\partial_\mu \phi$,于是有
\beq
\partial_\mu (\frac{\partial \lagden}{\partial (\partial_\mu \phi)}) - \frac{\partial \lagden}{\partial \phi} = 0
\eeq
考虑二次型动能和二次势
\beq
\lagden = \frac{1}{2} \partial^\mu \phi \partial_\mu \phi - \frac{1}{2}m^2 \phi^2
\eeq
代入拉格朗日方程就可以得到场的方程了。不过初学的时候很容易因为不知道这个带下标的偏导数的偏导数怎么求,就放弃学习了(我本人)。它是这样进行的,首先,我们要求的东西是
\beq
\frac{\partial (\partial^\mu \phi \partial_\mu \phi)}{\partial (\partial_\mu \phi)}
\eeq
根据同一指标不出现三次的原则,把分子重新成$\partial^\mu \phi \partial_\mu \phi = \partial^\sigma \phi \partial_\sigma \phi = g^{\rho \sigma} \partial_\rho \phi \partial_\sigma \phi$再求导,注意到$\partial_\mu \phi$之间的独立性,有
\beq
\frac{\partial (\partial_\mu \phi )}{\partial (\partial_\nu \phi )} = \delta^\nu_\mu
\eeq
再利用乘积的求导法则得到
\bea
&\frac{\partial (g^{\rho \sigma} \partial_\rho \phi \partial_\sigma \phi)}{\partial \partial(\partial_\mu \phi)} \\
= &g^{\rho \sigma} \partial_\rho \phi \delta^\mu_\sigma + g^{\rho \sigma} \delta^\mu_\rho \partial_\sigma \phi \\
= &g^{\rho \mu}\partial_\rho \phi+ g^{\mu \sigma}\partial_\sigma \phi \\
  &\text{($\partial^\mu \phi$的定义)} \\
= & 2\partial^\mu \phi
\eea
于是这样的拉格朗日密度对应的拉格朗日方程是
\beq
\partial_\mu \partial^\mu \phi + m^2 \phi = 0
\eeq
引入记号$\partial^2 \equiv \partial_\mu \partial^\mu = g^{\mu \nu} \partial_\mu \partial_\nu$,上式可以写成一个非常漂亮的形式
\beq
(\partial^2 + m^2) \phi = 0
\eeq
假设$\phi(x^\mu) = Ae^{i k_\mu x^\mu}$,$k_\mu = (\omega,\vec{k})$。带入闵可夫斯基度规,就有
\beq
\vec{k}^2 + m^2 = \omega^2
\eeq
看,这不就是$E^2 = m^2 + \vec{p}^2$吗。
\par
将时间维特殊化,记$\dot{\phi} = \partial_0 \phi$,定义场的动量为
\beq
\pi = \frac{\partial \lag}{\partial \dot{\phi}}
\eeq
定义哈密顿密度为
\beq
\ham = \pi \dot{\phi} - \lag
\eeq
可以算出,$\lag = \frac{1}{2} \partial^\mu \phi \partial_\mu \phi - \frac{1}{2}m^2 \phi^2$的哈密顿密度为
\beq
\ham = \frac{1}{2}(\dot{\phi}^2  + (\vec{\nabla} \phi)^2 + m^2\phi^2)
\eeq
\subsection{高斯积分}
我们对下面这个公式早已滚瓜烂熟倒背如流张口就来了
\beq
\int_{-\infty}^{\infty} e^{-x^2} dx = \sqrt{\pi}
\eeq
做一点简单的变量代换就得到
\beq \label{gasint}
\int_{-\infty}^{\infty} e^{-\frac{ax^2}{2}} dx = \sqrt{\frac{2\pi}{a}}
\eeq
积分$$
\int_{-\infty}^{\infty} x^{2n} e^{-\frac{ax^2}{2}}dx
$$
可以由两种方法求得。第一种是\cref{gasint}两边对$a$求$n$次导,第二种是利用$\Gamma$函数。结果是
\beq
\int_{-\infty}^{\infty} x^{2n} e^{-\frac{ax^2}{2}}dx = \frac{(2n-1)(2n-3)\cdots 3\cdot 1}{a^n} \sqrt{\frac{2\pi}{a}}
\eeq
\par
如果指数上还有一个一次项,用初中学的配方法可以得到积分
\beq
\int_{-\infty}^{\infty} e^{-\frac{ax^2}{2} + Jx} dx = e^{\frac{J^2}{2a}} \sqrt{\frac{2\pi}{a}}
\eeq
现在将上面这个积分推广到$n$维。计算积分
$$ \int e^{- \frac{1}{2} A_{ij} x^i x^j + J_k x^k} dx^1 dx^2 \cdots dx^n$$
结果是怎么样呢?为了计算,将二次型对角化后,就可以拆成$n$个高斯积分的积了。设有一个正交矩阵$R$使得$\vec{x} = R\vec{y}$,二次型化为$\vec{x}^\mathrm{T} A \vec{x} = \vec{y}^\mathrm{T} R^{-1} A R \vec{y}$,其中$P = R^{-1} A R$是一个对角阵;一次项为$\vec{J}^\mathrm{T} \cdot \vec{x} = \vec{J}^\mathrm{T} R \vec{y}$。而且,因为坐标变换矩阵是正交矩阵,坐标变换的雅可比行列式为1,$d^n \vec{y} = d^n \vec{x}$。现在,上面这个积分是
\beq
\int e^{ - \frac{1}{2} P_{ii} (y^i)^2 + J_k R^k_{\ i} y^i} dy^1 dy^2 \cdots dy^n
\eeq
现在这个积分就变成$n$个积分的积了。单个积分的结果为
\beq
\int e^{ - \frac{1}{2} P_{ii} (y^i)^2 + J_k R^k_{\ i} y^i} dy^i = e^{\frac{(J_k R^k_{\ i})^2}{2P_{ii}}} \sqrt{\frac{2\pi}{P_{ii}}}
\eeq
指数上的分子就是$R\vec{J}$的第$i$个分量的平方。$n$个积分的积为
\beq
(\frac{2\pi}{\prod_i P_{ii}})^{n/2} e^{\frac{1}{2} \sum_i (R\vec{J})_i^2 / P_{ii}}
\eeq
和式为
\beq
\sum_i (R\vec{J})_i^2 / P_{ii}
=\frac{1}{\prod_i P_{ii}} \sum_i (R\vec{J})_i^2 M_{ii}
\eeq
其中$M_{ii} = \prod_{j \not= i} P_{jj}$是对角阵$P$的代数余子式。而$\prod_i P_{ii}$就是$P$的行列式,于是$\cfrac{M_{ii}}{\prod_i P_{ii}}$就是$P$的逆矩阵的矩阵元$P^{-1}_{ii}$。利用$P=R^{-1} A R = R^\mathrm{T} A R$,两边取逆得到$P^{-1} = R^{-1} A^{-1} R = R^\mathrm{T} A^{-1} R$,因而上式进一步化简为
\beq
\sum_i (R\vec{J})_i^2 / P_{ii} = (R \vec{J})^\mathrm{T} P^{-1} (R \vec{J}) = \vec{J}^\mathrm{T} A^{-1} \vec{J}
\eeq
由于$A$和$P$相似,$\det A = \det P =\prod_i P_{ii}$我们得到
\beq
\int e^{- \frac{1}{2} A_{ij} x^i x^j + J_k x^k} dx^1 dx^2 \cdots dx^n = (\frac{2\pi}{\det A})^{n/2} e^{\frac{1}{2} {\vec{J}}^\mathrm{T} A^{-1} \vec{J}}
\eeq

\subsection{谐振子的量子化}
经典力学中的谐振子具有哈密顿函数
\beq
H = \frac{1}{2}p^2 + \frac{1}{2}\omega^2 q^2
\eeq
量子力学中,力学量用算符表示。谐振子的哈密顿算符可以用动量算符和坐标算符表示为
\beq
\hat{H} = \frac{1}{2} \hat{p}^2 + \frac{1}{2}\omega^2 \hat{q}^2
\eeq
设产生算符$\hat{a}^\dagger$和湮灭算符$\hat{a}$使得
\beq
\hat{q} = \frac{1}{\sqrt{2\omega}}(\hat{a}^\dagger + \hat{a})
\eeq
\beq
\hat{p} = -i \sqrt{\frac{\omega}{2}}(\hat{a}^\dagger - \hat{a})
\eeq
假设坐标算符和动量算符间满足对易关系$[\hat{q},\hat{p}] = i$(由经典力学中坐标和动量的泊松括号推广而来),可以算出
\bea
i &= [\hat{q},\hat{p}] = \frac{-i}{2} [\hat{a}^\dagger + \hat{a},\hat{a}^\dagger - \hat{a}] \\
&= \frac{-i}{2}( [\hat{a}^\dagger,-\hat{a}] + [\hat{a},\hat{a}^\dagger] ) = i [\hat{a},\hat{a}^\dagger]
\eea
于是我们得到产生湮灭算符间的对易关系为$[\hat{a},\hat{a}^\dagger] = 1$。
\par
哈密顿算符用产生湮灭算符表示为
\beq
\hat{H} = \omega(\hat{a}^\dagger \hat{a} + \frac{1}{2})
\eeq
并可以定义粒子数算符
\beq
\hat{N} = \hat{a}^\dagger \hat{a}
\eeq
它可以用哈密顿算符直接表示出来,因此,$\hat{H}$与$\hat{N}$对易,粒子数$N$为一个守恒量。此外可以验证以下事实
\begin{enumerate}
\item $\hat{H}$和$\hat{N}$都是厄米算符,它们的本征值都是实数;
\item 如果$\hat{N}|n\rangle = N |n\rangle$,则$\hat{a}|n\rangle$和$\hat{a}^\dagger |n\rangle$分别是$\hat{N}$的本征值为$N-1$和$N+1$的本征态;
\end{enumerate}
\par
定义谐振子的基态满足$\hat{a} |0\rangle = 0|0\rangle$,即它不能再被湮灭算符湮灭。那么,将这个条件带入哈密顿算符的表达式并利用产生算符的性质,就可以得到谐振子的能级公式$E_N = (N+ \frac{1}{2})\omega$。

\section{自由场的量子化}
\subsection{自由实标量场}
\subsubsection{模式分解}
我们已经指出,拉格朗日密度$\lagden = \frac{1}{2} \partial^\mu \phi \partial_\mu \phi - \frac{1}{2}m^2 \phi^2$对应的标量场满足运动方程
\beq
(\partial^2 + m^2) \phi = \ddot{\phi} -\vec{\nabla}^2 \phi+ m^2 \phi = 0
\eeq
设振动模式$\vec{p}$对应的解是$\phi_{\vec{p}} (t,\vec{x}) = e^{i \vec{p} \cdot \vec{x} - i \omega_{\vec{p}} t}$,带入运动方程就得到
\beq
\omega_{\vec{p}} = \sqrt{\vec{p}^2 + m^2}
\eeq
在经典场论下,每个振动模式$\vec{p}$的行为都像单个谐振子,而对单个谐振子的量子化是容易的。类比地写出模式$\vec{p}$满足的产生湮灭算符满足
\beq
\hat{\phi}(\vec{x})  =\int \frac{d^3 \vec{p}}{(2\pi)^3} \frac{1}{\sqrt{2\omega_{\vec{p}}}} (\hat{a}_{\vec{p}} e^{i\vec{p} \cdot \vec{x}} + \hat{a}_{\vec{p}}^\dagger e^{-i\vec{p} \cdot \vec{x}})
\eeq
\beq
\hat{\pi}(\vec{x})  = -i \int \frac{d^3 \vec{p}}{(2\pi)^3} \sqrt{\frac{\omega_{\vec{p}}}{2}}(\hat{a}_{\vec{p}} e^{i\vec{p} \cdot \vec{x}} - \hat{a}_{\vec{p}}^\dagger e^{-i\vec{p} \cdot \vec{x}})
\eeq
其中$\hat{\phi}$和$\hat{\pi}$为场的场算符和动量算符。
\par
可以证明下面两组对易关系等价
\beq
[\hat{\phi}(\vec{x}),\hat{\phi}(\vec{y})] = [\hat{\pi}(\vec{x}),\hat{\pi}(\vec{y})] = 0 ,\ [\hat{\phi}(\vec{x}),\hat{\pi}(\vec{y})] = i\delta^3(\vec{x}-\vec{y})
\eeq
\beq
[\hat{a}_{\vec{p}},\hat{a}_{\vec{q}}] = [\hat{a}_{\vec{p}}^\dagger,\hat{a}_{\vec{q}}^\dagger] = 0 ,\ [\hat{a}_{\vec{p}},\hat{a}_{\vec{q}}^\dagger] = (2\pi)^3 \delta^3(\vec{p}-\vec{q})
\eeq
这与谐振子的两组对易关系对应。
\par
利用上面的对易关系和KG场哈密顿密度的定义,将对应的场和动量换成场算符和动量算符\footnote{这里有一个替换规则,把经典中的物理量换成算符时,湮灭算符必须放在产生算符的右边。这称为常序规则。},化简得到哈密顿密度算符,再对哈密顿密度算符对全空间积分得到哈密顿算符
\beq
\hat{H} = \int \frac{d^3 \vec{p}}{(2\pi)^3} \omega_{\vec{p}} \left[\hat{}a_{\vec{p}}^{\dagger} \hat{a}_{\vec{p}}+\frac{1}{2}(2 \pi)^{3} \delta^{(3)}(0)\right]
\eeq
这里的$\delta$函数项与谐振子中的零点能$\frac{1}{2}\omega$对应,对全空间的积分使得该项发散。我们重新定义哈密顿量为减去零点能的哈密顿量,哈密顿算符就是
\beq \label{singleham}
\hat{H} = \int \frac{d^3 \vec{p}}{(2\pi)^3} \omega_{\vec{p}} \hat{a}_{\vec{p}}^\dagger \hat{a}_{\vec{p}}
\eeq
\par
最后重申,这里的图像是,对每个傅立叶分解下的振动模式$\vec{p}$的行为都类似谐振子,将每个模式量子化后再按模式的体积元$\frac{d^3\vec{p}}{(2\pi)^3}$叠加(傅立叶逆变换)得到整个场的量子化。
\subsubsection{真空与粒子}
定义场的真空态$|0\rangle$,使得对于任意模式$\vec{p}$都有$\hat{a}_{\vec{p}} |0\rangle = 0|0\rangle$。带入\cref{singleham}得到$\hat{H}|0\rangle = 0 |0\rangle$,也即真空能量为0。
\par
定义粒子态
\beq
|\vec{p}_1,\vec{p}_2,\cdots,\vec{p}_n\rangle = \hat{a}_{\vec{p}_1}^\dagger \hat{a}_{\vec{p}_2}^\dagger \cdots \hat{a}_{\vec{p}_n}^\dagger
\eeq
和粒子数算符
\beq
\hat{N} = \int \frac{d^3 \vec{p}}{(2\pi)^3} \hat{a}_{\vec{p}}^\dagger \hat{a}_{\vec{p}} 
\eeq
利用产生湮灭算符间的对易关系和真空态的定义可以验证
\beq
\hat{N} |\vec{p}_1,\vec{p}_2,\cdots,\vec{p}_n\rangle = n |\vec{p}_1,\vec{p}_2,\cdots,\vec{p}_n\rangle
\eeq
也就是说,粒子态可以被诠释为具有一系列动量的$n$个粒子。而产生算符则用来产生这些粒子,因此我们说,粒子是场的激发态。此外有
\beq
|\vec{p},\vec{q} \rangle = \hat{a}_{\vec{p}}^\dagger \hat{a}_{\vec{q}}^\dagger |0\rangle = \hat{a}_{\vec{q}}^\dagger \hat{a}_{\vec{p}}^\dagger |0\rangle = |\vec{q},\vec{p} \rangle
\eeq
中间一步我们利用了产生算符间的可交换性。上式表明,KG场表征的粒子对交换算符的本征值为1,为玻色子。
\par
此外,粒子数算符和哈密顿算符对应,按量子力学,与哈密顿算符对易的算符表征的力学量为守恒量,于是我们得到了粒子数守恒的结论。

\subsection{复自由标量场}
\subsubsection{复自由场}
定义复值场$\psi$的拉格朗日密度为
\beq
\lag = \partial_\mu \psi \partial^\mu \psi^* - m^2 \psi \psi^*
\eeq
它的运动方程为
\beq
\partial_\mu \partial^\mu \psi + m^2 \psi = 0
\eeq
\beq
\partial_\mu \partial^\mu \psi^* + m^2 \psi^* = 0
\eeq
\par
场和共轭的动量为
\beq
\pi = \frac{\partial \lag}{\partial (\partial_0 \psi)} = \dot{\psi^*}
\eeq
\beq
\pi^* = \frac{\partial \lag}{\partial (\partial_0 \psi^*)} = \dot{\psi}
\eeq
哈密顿密度为
\par
\beq
\ham = \pi \dot{\psi} + \pi \dot{\psi} - \lag = \pi \pi^* + \vec{\nabla}\psi \cdot \vec{\nabla}\psi^* + m^2 \psi \psi^*
\eeq
同样的,场和它共轭分别在各个模式$\vec{p}$下的行为类似谐振子,可以同样地进行量子化。不过,这里有两个方程,也就是有原来2倍的自由度。这迫使我们定义两种产生湮灭算符$\hat{b}_{\vec{p}}^\dagger$,$\hat{c}_{\vec{p}}^\dagger$,$\hat{b}_{\vec{p}}$和$\hat{c}_{\vec{p}}$。同组产生湮灭算符内满足的对易关系与之前实标量场一致,并且增加它们之间的对易关系为$[\hat{b}_{\vec{p}},\hat{c}_{\vec{p}}^\dagger] = [\hat{c}_{\vec{p}},\hat{b}_{\vec{p}}^\dagger] = 0$。写出场及其共轭的场算符和动量算符为
\beq
\hat{\psi}(\vec{x}) = \int \frac{d^3 \vec{p}}{(2\pi)^3} \frac{1}{\sqrt{2 \omega_{\vec{p}}}} (\hat{b}_{\vec{p}} e^{i\vec{p} \cdot \vec{x}} + \hat{c}_{\vec{p}}^\dagger e^{-i\vec{p}\cdot \vec{x}})
\eeq
\beq
\hat{\psi}^\dagger (\vec{x}) = \int \frac{d^3 \vec{p}}{(2\pi)^3} \frac{1}{\sqrt{2 \omega_{\vec{p}}}} (\hat{b}_{\vec{p}}^\dagger e^{-i\vec{p} \cdot \vec{x}} + \hat{c}_{\vec{p}} e^{i\vec{p}\cdot \vec{x}})
\eeq
\beq
\hat{\pi}(\vec{x}) = i \int \frac{d^3\vec{p}}{(2\pi)^3} \sqrt{\frac{\omega_{\vec{p}}}{2}} (\hat{b}_{\vec{p}}^\dagger e^{-i\vec{p} \cdot \vec{x}} - \hat{c}_{\vec{p}} e^{i\vec{p}\cdot \vec{x}})
\eeq
\beq
\hat{\pi}(\vec{x})^\dagger = -i \int \frac{d^3\vec{p}}{(2\pi)^3} \sqrt{\frac{\omega_{\vec{p}}}{2}} (\hat{b}_{\vec{p}} e^{i\vec{p} \cdot \vec{x}}  -\hat{c}_{\vec{p}}^\dagger e^{-i\vec{p}\cdot \vec{x}})
\eeq

\subsubsection{粒子与反粒子}
采用与之前相当的真空定义,定义正粒子数算符
\beq
\hat{N}_b = \int \frac{d^3 \vec{p}}{(2\pi)^3}   \hat{b}_{\vec{p}}^\dagger \hat{b}_{\vec{p}}
\eeq
和反粒子数算符
\beq
\hat{N}_c = \int \frac{d^3 \vec{p}}{(2\pi)^3} \hat{c}_{\vec{p}}^\dagger \hat{c}_{\vec{p}}
\eeq
\par
正粒子产生算符作用到真空态上产生一个正粒子,如$\hat{b}_{\vec{p}}^\dagger |0\rangle = |\vec{p}_b \rangle$,等等,反粒子亦如此。粒子数算符的本征值为正粒子或反粒子的粒子数。
\par
替换规则可以给出去掉零点能的全空间哈密顿量为
\beq \label{complexham}
\hat{H} = \int \frac{d^3 \vec{p}}{(2\pi)^3} \omega_{\vec{p}} (\hat{b}_{\vec{p}}^\dagger \hat{b}_{\vec{p}} +\hat{c}_{\vec{p}}^\dagger \hat{c}_{\vec{p}}
\eeq
很容易看出,哈密顿量和各个粒子数算符间都是对易的,因此在自由场下,正粒子和反粒子数分别守恒。
\par
在复自由场的经典场论中,$Q = i \int d^3 \vec{x} (\dot{\psi}^* \psi - \dot{\psi} \psi^*)$为一个守恒量,这可以从诺特定理直接推出,也可以对时间求导,并假定场$\phi$在无穷远处趋于零(或满足一般零边界条件)推出。在量子场论中,带入各个产生湮灭算符的定义和常序规则,化简得到$Q$的算符对应为
\beq
\hat{Q} = \hat{N}_b - \hat{N}_c
\eeq
也就是说,正粒子数与反粒子数的差为一个常数。这个结论在相互作用的场论中依然成立。

\section{传播子和路径积分}

\end{document}