\documentclass[CJK]{beamer}
\input{macros.tex}
\usepackage{mathrsfs}

\newcommand{\field}{\mathscr{F}}

\newcommand{\reals}{\mathbb{R}}
\newcommand{\complexs}{\mathbb{C}}
\newcommand{\ints}{\mathbb{Z}}
%\newcommand{\dim}{\mathrm{dim\ }}
\newcommand{\diag}{\mathrm{diag \ }}
\newcommand{\up}{\uparrow}
\newcommand{\down}{\downarrow}
\newcommand{\su}{\mathfrak{su}}
\newcommand{\so}{\mathfrak{so}}
\newcommand{\tr}{\mathrm{tr\ }}
\newcommand{\card}{\mathrm{card \ }}
\newcommand{\mani}{\mathcal{M}}
\newcommand{\surface}{\mathcal{S}}

\newtheorem{thm}{定理}[section]
\newtheorem{axm}{公理}[section]
\newtheorem{dfn}{定义}[section]

%\cpic{<尺寸>}{<文件名>}}用于生成居中的图片。
\newcommand{\cpic}[2]{
\begin{center}
\includegraphics[scale=#1]{#2}
\end{center}
}

%\cpicn{<尺寸>}{<文件名>}{<注释>}用于生成居中且带有注释的图片,其label为图片名。
\newcommand{\cpicn}[3]
{
\begin{figure}[h!]
\cpic{#1}{#2}
\caption{#3\label{#2}}
\end{figure}
}

\title{DiffGeo\\ Talk 2-Vectors and Tensors on a Manifold}
  \author{}
  \date{}


\begin{document}

\begin{frame}
 
\begin{center}
\begin{Large}
\bch
{\bf DiffGeo}

{\vskip 0.3in}

Talk 2-Vectors and Tensors on a Manifold

\ech
\end{Large}
\end{center}


\end{frame}

\section{Vectors}
\begin{frame}
\frametitle{\bch 什么是流形 \ech}
\bch
\begin{itemize}
\item
一个被许多个小开集覆盖的拓扑空间。
\item
每个小开集上定义了局域坐标。
\cpic{0.15}{ok}
\end{itemize}
\ech
\end{frame}

\begin{frame}
\frametitle{\bch 什么是矢量 \ech}
\bch
我们先在流形上点点定义矢量。
\cpic{0.2}{happy}
\ech
\end{frame}

\begin{frame}
\frametitle{\bch 标量场 \ech}
\bch
\begin{itemize}
\item
流形$\mani$上的{\bf \color{purple} 标量场}是映射
$$
f: \mani \to \reals
$$
即将流形上的点$p$映射到一个实数$f(p)$。
\item
流形$\mani$上所有标量场的集合记为$\field_\mani$。$\mani$上$r$阶连续(指它的分量形式$r$阶连续)的标量场组成的集合记作$C^r_\mani$。
\item
请举出标量场的一些物理或几何例子。
\end{itemize}

\ech
\end{frame}


\begin{frame}
\frametitle{\bch 流形上某一点的(切)矢量 \ech}
\bch

流形$\mani$上点$p$的(切)矢量是标量场到实数的映射
$$
X_p: \field_\mani \to \reals
$$
并且满足
\begin{itemize}
\item
$X_p$是{\bf 线性映射},即
$$
X_p(\alpha f + \beta g) = \alpha X_p(f) + \beta X_p(g)
$$
其中$\alpha,\beta \in \reals$。
\item
$X_p$满足{\bf 莱布尼兹律}{\color{blue}
$$
X_p(fg) = f(p) X_p(g) + g(p) X_p(f)
$$}
\end{itemize}
使验证:$\reals^3$上的方向导数算符$$\partial_{\vec v} \equiv \vec v \cdot \vec \nabla $$是$\reals^3$流形上的一个矢量。

\ech
\end{frame}


\begin{frame}
\frametitle{\bch 课堂小练习:切矢的性质 \ech}
\bch
利用切矢$X$的一般定义证明
\begin{itemize}
\item $X_p(1) = 0$;
\item $X_p(c) = 0$,其中$c$为常数函数;
\item $X_p(\alpha f) = \alpha X_p(f)$,其中$\alpha \in \reals$为常数。
\end{itemize}
\cpic{0.2}{sad}

\ech
\end{frame}

\begin{frame}
\frametitle{\bch 切矢的运算 \ech}
\bch
我们定义$p$点处切矢的线性运算:
\begin{itemize}
\item 加法:$(X_p+Y_p)(f) = X_p(f) + Y_p(f)$;
\item 零元:存在一个特殊的$0_p\in T_p\mani$使得对任意$f\in \field_\mani$有$0_p (f) = 0$
\item 数乘:$(\alpha X_p)( f) = \alpha X_p(f)$。
\end{itemize}
在这样的定义下,我们稍后会说明,所有的$p$点处的切矢构成线性空间$T_p \mani$,且其维度$\dim T_p \mani = \dim \mani$。
\ech
\end{frame}

\begin{frame}
\frametitle{\bch 线性空间的对偶空间 \ech}
\bch
对一般的线性空间$V$,我们可以定义它的{\bf \color{purple}对偶空间}$V^*$,它的元素是$V$上的所有线性函数,步骤是
\begin{itemize}
\item 线性空间$V$上的线性函数定义为满足$f(\alpha u + \beta v)= \alpha f(u) + \beta f(v)$的函数;
\item 定义线性函数间的加法和数乘为$(\alpha f + \beta g) (v) = \alpha f(v) + \beta g(v)$;
\item 假设$V$有一组基底$\{e_1,\cdots,e_n\}$,定义线性函数$e^i$满足$e^i(e_j) = \delta^i_j$,证明所有的线性函数都可以用$e^i$线性表示;
\item 于是$V^*$是一个和$V$同维的线性空间。
\end{itemize}
\ech
\end{frame}


\begin{frame}
\frametitle{\bch 切空间的对偶空间 \ech}
\bch
因为切空间$T_p\mani$是一个线性空间,所以就可以定义它的对偶空间$T_p^*\mani$(称为余切空间),它的元素记作$df_p: T_p \mani \to \reals$。
\begin{itemize}
\item 按定义,$df_p$应该满足$df_p(\alpha X_p + \beta Y_p) = \alpha df_p(X) + \beta df_p(Y)$。
\item 因为$T_p^*\mani$也是一个有限维线性空间,所以只用找到一种合理的$df_p$定义即可。
\item 试验证:定义{\color{blue}$$df_p(X_p) \equiv X_p(f)$$}满足$df_p$是线性,且$df_p$和$dg_p$线性可加的要求。
\end{itemize}
今后在不引起歧义的情况下,我们可能省略$X_p$和$df_p$中指代点$p$的下标。
\ech
\end{frame}

\begin{frame}
\frametitle{\bch 切空间上的张量 \ech}
\bch
有了切空间和余切空间的概念,我们就可以定义切空间上的张量。
\begin{itemize}
\item 对一般的线性空间$V$,若其对偶空间为$V^*$,其上的$(s,t)$型张量一般定义为一个$s+t$元多重线性函数(对每一个自变量呈线性){\color{blue}
$$
T:\underbrace{V\times V\times \cdots \times V}_{s\text{ 个}} \times  \underbrace{V^* \times V^* \times \cdots \times V^*}_{t\text{ 个}} \to \reals
$$}
也就是一个接受$s$个矢量和$t$个对偶矢量的多重线性函数$T(v_1,\cdots,v_s;f_1,\cdots,f_t)$。
\item 流形$\mani$上点$p$处的张量定义只用把上面的$V$换成$T_p\mani$,也就是多重线性函数$T(X_1,\cdots,X_s;df_1,\cdots,df_t)$。
\end{itemize}
\ech
\end{frame}

\begin{frame}
\frametitle{\bch 度规张量 \ech}
\bch
流形$\mani$上的度规张量是切空间$T_p\mani$上的一个$(2,0)$型对称张量,它定义了$T_p \mani$上两个矢量的内积
$$
\langle X,Y\rangle \equiv g(X,Y)
$$
\begin{itemize}
\item 度规按定义必须满足$g(X,Y) = g(Y,X)$;
\item 根据线性空间上内积的一般定义,度规还必须是非退化的,也就是如果$\forall Y \in T_p\mani,\ g(X,Y) = 0$当且仅当$X=0$。
\item 如果对任意$X\in T_p\mani$都有$g(X,X) > 0$,则称装备了这样正定度规的流形为{\bf \color{purple}黎曼流形}。
\end{itemize}
\ech
\end{frame}

\begin{frame}
\frametitle{\bch 切矢的对应余切矢 \ech}
\bch
假设流形$\mani$装备了度规$g$,那么对{\bf \color{red}给定的}$X\in T_p \mani$,映射
$$
g_X(Y) \equiv g(X,Y)
$$
是$T_p \mani \to \reals$的线性映射,因此按定义$g_X \in T_p^* \mani$。这称为切矢$X$自然对应的余切矢。
\cpic{0.15}{eat}
\ech
\end{frame}

\section{Coor Repre}
\begin{frame}
\frametitle{\bch 流形上的局部坐标 \ech}
\bch
回忆以下事实
\begin{itemize}
\item
对于流形$\mani$上的一个小开集$O$(小恐龙的鳞片)上都定义了一个到$\reals^n$的一一映射$\psi$,称为局部坐标。
\cpic{0.15}{dina}
\item
结构$(O,\psi)$称作图。
\item
两个图$(O_a,\psi_a)$和$(O_b,\psi_b)$可能有重叠部分,重叠部分的坐标变换为$\psi_b \circ \psi_a^{-1}: \reals^n \to \reals^n$。假如这个映射在流形上处处至少是$C^r$的,就说$\mani$是$C^r$流形。$C^\infty$流形称为光滑流形。
\end{itemize}
\ech
\end{frame}

\begin{frame}
\frametitle{\bch 标量场的坐标表象 \ech}
\bch
假如$p\in O$,开集$O$的局部坐标为$\psi$。则$\mani$上的标量场$f(p)$在$p$附近可以表现称一个$\reals^n \to \reals$的函数
$$
F \equiv f \circ \psi^{-1} :\reals^n \to \reals
$$
它就是一个我们熟悉的$n$元函数$F(x^1,\cdots,x^n)$,它将代表点的局部坐标映到一个实数。
\ech
\end{frame}


\begin{frame}
\frametitle{\bch 切空间的基 \ech}
\bch
在定义了局域坐标的开集$O$中的一点$p$处,定义$n$个算符:
$$
\partial_\mu (f) \equiv \frac{\partial F}{\partial x^\mu}
$$
\begin{itemize}
\item 证明上面的$n$个$\partial_\mu$满足线性和莱布尼兹律(提示:两个标量场乘积的坐标表示等于其坐标表示之积),从而它们是切矢。
\item 所有的切矢$X$都可以用$\partial_\mu$唯一地线性表示{\color{blue} $$ X = X^\mu \partial_\mu$$}(很好理解,证明困难,所以不证)
\item 从而这$n$个$X_\mu$张成$T_p \mani$。
\end{itemize}
\ech
\end{frame}

\begin{frame}
\frametitle{\bch 余切空间的基 \ech}
\bch
\begin{itemize}
\item 显然,第$\nu$个局部坐标$x^\mu$的坐标表象是它本身$F = x^\nu$。
\item 按定义,$\partial_\mu (x^\nu) = \delta_\mu^\nu$。
\item 这显然满足对偶空间基底的一般定义。
\item 于是有$n$个特殊的余切矢$dx^\nu$满足$dx^\nu(\partial_\mu) \equiv \partial_\mu (x^\nu) = \delta_\mu^\nu$。
\cpic{0.15}{ques}
\item 任意余切矢$df\in T_p^* \mani $可以按$df = f_\nu dx^\nu$展开。
\end{itemize}

\ech
\end{frame}


\begin{frame}
\frametitle{\bch 切矢和余切矢的分量 \ech}
\bch
\begin{itemize}
\item 对一般的切矢$X = X^\mu \partial_\mu$,有$dx^\nu (X) = dx^\nu (X^\mu \partial_\mu) = X^\mu dx^\nu ( \partial_\mu) = X^\nu$。于是余切矢基底$dx^\nu$作用在切矢的作用是取出切矢的第$\nu$分量。
\item 同样的,对余切矢$df = f_\nu dx^\nu$,有$df(\partial_\mu) = f_\mu$。
\item 对任意的$X\in T_p \mani$和$df\in T_p^* \mani$,有{\color{blue}$$df(X) = f_\mu X^\mu$$}
\item 给定$p$所在图的局部坐标就相当于指定了切空间和余切空间的基底,从而给出切矢和余切矢的分量表示。
\end{itemize}
\ech
\end{frame}

\begin{frame}
\frametitle{\bch 张量的分量 \ech}
\bch
\begin{itemize}
\item 一般的$(s,t)$型张量接受$s$个切矢和$t$个余切矢作为自变量。
\item 在给定基的情况下,第$i$个切矢展开为$X_i = X_i^{\mu_i} \partial_{\mu_i}$,第$j$个余切矢展开为$df_j = f_{j\nu_j} dx^{\nu_j}$。
\item 根据$T$的多重线性要求有$$
T(X_1,\cdots,X_s;df^1,\cdots,df^t) $$
$$
=X_1^{\mu_1} \cdots X_s^{\mu_s} f_{1\nu_1} \cdots f_{t\nu_t} T(\partial_{\mu_1} ,\cdots,\partial_{\mu_s};dx^{\nu_1},\cdots,dx^{\nu_t})
$$
\item 记{\color{blue}$$T^{\nu_1\cdots \nu_t}_{\mu_1\cdots \mu_s} \equiv  T(\partial_{\mu_1} ,\cdots,\partial_{\mu_s};dx^{\nu_1},\cdots,dx^{\nu_t})$$}
称为张量(在给定坐标下)的分量
\end{itemize}
\ech
\end{frame}

\begin{frame}
\frametitle{\bch 度规张量的分量 \ech}
\bch
\begin{itemize}
\item
度规张量是一个$(2,0)$张量,所以它的分量是
$$
g_{\mu \nu} = g(\partial_\mu,\partial_\nu)
$$
\item
两个切矢$X,Y$的内积
$$
\langle X,Y\rangle \equiv g(X,Y) = g_{\mu \nu} X^\mu Y^\nu
$$
\item 切矢$X$的自然对偶的分量
$$(g_X)_\nu \equiv g_X(\partial_\nu) \equiv g(X,\partial_\nu) = g_{\mu \nu} X^\mu 
$$
将$(g_X)_\nu$就记作$X_\nu$不至于引起歧义,于是有著名的升降指标公式
{\color{blue}
$$
X_\mu = g_{\mu \nu} X^\nu
$$}
\end{itemize}
\ech
\end{frame}

\section{Curves and Vectors, Examples}
\begin{frame}
\frametitle{\bch 例子1:$3$维空间中的切矢和余切矢 \ech}
\bch
\begin{itemize}
\item
我们考虑$3$维欧氏空间$\reals^n$点$P(x,y,z)$处的切空间。
\item
根据定义,$P$处的切矢可以按$\partial_i$展开为$X = X\partial_x + Y\partial_y + Z \partial_z$。这就是$\vec{X} = (X,Y,Z)$矢量的方向导数算符(再乘以$\vec X$的长度)
\item
我们考虑$X$对标量场$f(x,y,z)$的作用。有
$$
X(f) = X\frac{\partial f}{\partial x} + Y \frac{\partial f}{\partial y} + Z \frac{\partial f}{\partial z}
$$
这就是求方向导数。另一方面也可以看出,$T_P \reals^3 = \reals^3$。
\end{itemize}
\ech
\end{frame}

\begin{frame}
\frametitle{\bch 例子1:$3$维空间中的切矢和余切矢 \ech}
\bch
\begin{itemize}
\item 另一方面,从余切矢的角度看,$f$定义出的余切矢的第$i$分量按定义为
$$
df(\partial_i) \equiv \partial_i f
$$
这可以看成$f$在$i$方向的增长速率乘以单位长度。
\item 记号$df$的实际值取决于$X$的选择:$df(X) \equiv X(f) = \vec X \cdot \vec \nabla f$。也就是说,$f$在$P$处的增量取决于选定参考矢量$\vec X$的方向和大小。这和我们一般将$df$理解成线性阶增量的想法是一致的,增量当然依赖于参考矢量的大小和方向,这使得$df$不再代表一个无穷小量。
\item 当然,无穷小量本身就是一个不严谨的概念。

\end{itemize}
\ech
\end{frame}


\begin{frame}
\frametitle{\bch 例子2:2维正则曲面 \ech}
\bch
\begin{itemize}
\item 我们在古典微分几何那里已经研究过正则曲面$\surface$,它由两个参数$(u,v)$表征,这就是这块正则曲面的局部坐标。给定$(u,v)$,它们就唯一确定了曲面上的点$p(u,v)$。
\item 切向量按基展开为$X = X^u \partial_u + X^v \partial_v$。
\item 余切矢按基展开为$df = f_u du + f_v dv$。
\item $df(X) = f_u X^u + f_v X^v$就是标量函数$f$沿给定切向量$X$方向的线性阶增量。
\item 这里还不怎么能看出和之前学过的曲面论的联系,因为我们没有利用度规去决定曲面的形状,也没有把曲面放在$\reals^3$中去研究。我们稍后再返回这个例子。
\end{itemize}
\ech
\end{frame}

\begin{frame}
\frametitle{\bch 切矢的另一种解释 \ech}
\bch
我们之前是利用了曲面$\surface$上的曲线定义了切向量。我们可以仿照曲面上曲线的定义定义流形$\mani$上的曲线:
\begin{itemize}
\item 定义流形上的曲线为$\reals$上一个开区间$(a,b)$到$\mani$上点的映射
$$
\gamma: (a,b) \to \mani
$$
\item 利用流形上某一点$p$附近的局部坐标映射,曲线可以表现为一个单变量向量函数
$$
C \equiv \psi \circ \gamma : (a,b) \to \reals^n
$$
\item 考虑一个标量场$f$,其坐标表象为$F\equiv f \circ \psi^{-1}$,$f(\gamma(t))$就是一个$(a,b) \to \reals$的普通单变量函数。
\end{itemize}
\ech
\end{frame}

\begin{frame}
\frametitle{\bch 切矢的另一种解释 \ech}
\bch
我们自然地会希望去研究$f(\gamma(t))$的导数:
\begin{itemize}
\item 考虑$f(\gamma(t))$的变化:显然有$f\circ \gamma = f \circ \psi^{-1} \circ \psi \circ \gamma = F\circ C$(请画图理解这个等式),也即$f(\gamma(t)) = F(C(t))$。
\item 于是$f(\gamma(t))$的线性阶变化为
$$
\frac{\Delta f}{\Delta t} \big|_p \simeq\frac{dC^\mu }{dt} \partial_\mu F 
$$
这给出两个向量:$\frac{dC^\mu }{dt}$和$\partial_\mu F$,一个可以看成切矢的分量;另一个看成余切矢的分量。
\end{itemize}
\ech
\end{frame}

\section{More about Tensors}
\begin{frame}
\frametitle{\bch 张量 \ech}
\bch
像我们之前说的,张量的本质是一个多重线性映射,将几个矢量(和余切矢量)映射到一个数。
\begin{itemize}
\item 转动惯量张量$I(,)$接受两个角速度$\vec \omega$,返回刚体转动动能的两倍$T = \frac{1}{2} I (\vec \omega,\vec \omega)$。
\item 电导率张量接受两个电场,返回热功率密度$h = \sigma(\vec E,\vec E)$。
\end{itemize}
当我们选取了空间的一组基底,张量就有了坐标表示,也就是我们熟悉的分量形式$T_{\mu \nu}$
\ech
\end{frame}

\begin{frame}
\frametitle{\bch 标量和矢量作为特殊的张量 \ech}
\bch
既然张量是一个多重线性映射,那么把“多”改成“单”并不会有什么实质性的影响。
\begin{itemize}
\item 切矢$X$可以看成将余切矢映射到实数的函数$X:T_p^*\mani \to \reals$,具体形式是$X(df) \equiv X(f)$。
\item 余切矢$X$可以看成将切矢映射到实数的函数$X:T_p\mani \to \reals$,具体形式是$df(X) \equiv X(f)$。
\item 而张量的一般定义是$T:\underbrace{V\times V\times \cdots \times V}_{s\text{ 个}} \times  \underbrace{V^* \times V^* \times \cdots \times V^*}_{t\text{ 个}} \to \reals$
\end{itemize}
也就是说,切矢就是一个$(0,1)$型张量;余切矢就是一个$(1,0)$型张量。
\ech
\end{frame}

\begin{frame}
\frametitle{\bch 度规的逆 \ech}
\bch
我们可以利用度规把一个切矢变成余切矢
$$g_X \equiv g(X,)$$
我们也可以定义余切空间$T_p^*\mani$上的度规$h(,)$满足把两个切矢对应的余切矢映射到实数,且值等于切矢的内积
$$
h(g_X,g_Y) = g(X,Y)
$$
$h$就称作度规的逆,也就是$h$定义了余切空间上的内积
\ech
\end{frame}

\begin{frame}
\frametitle{\bch 度规的逆的分量表示 \ech}
\bch
利用$(0,2)$张量分量的定义$h(dx^\mu,dx^\nu) \equiv h^{\mu \nu}$证明,在给定基底的情况下,上面的等式$h(g_X,g_Y) = g(X,Y)$等价于
$$
g_{\mu \nu} g_{\rho \sigma} h^{\nu \sigma} = g_{\mu \rho}
$$
或
$$g_{\rho \sigma} h^{\sigma \nu} = \delta_\rho^{\ \nu}
$$
上面的等式对应的矩阵形式就是说$h$是$g$的逆矩阵。通常将$h^{\mu \nu}$直接记作$g^{\mu \nu}$不会引发歧义。
\ech
\end{frame}

\begin{frame}
\frametitle{\bch 度规的逆的分量表示 \ech}
\bch
度规的逆又可以把余切矢自然对应到一个切矢:
$$
h_{df} \equiv h(df,) : V_p^* \mani \to \reals
$$
使验证:
\begin{itemize}
\item
上面定义的合理性:也即$h_{g_X} = X$。
\item
分量形式为
$$
(h_{df})^\nu = h(f_\mu dx^\mu,dx^\nu) = f_\mu h^{\mu \nu}
$$
\end{itemize}
\ech
\end{frame}

\begin{frame}
\frametitle{\bch 张量的升降指标 \ech}
\bch
把矢量升降指标的定义推广,按照以下方法升降张量的指标:
\begin{itemize}
\item 升指标:乘以度规的逆:
$$T_{\cdots \mu \cdots}^{\cdots\ \cdots}  g^{\mu \rho} = T_{\cdots\ \cdots}^{\cdots \rho \cdots}$$
\item 降指标:乘以度规本身:
$$T_{\cdots\ \cdots}^{\cdots \mu \cdots} g_{\mu \rho} = T_{\cdots \rho \cdots}^{\cdots\ \cdots}$$
\end{itemize}
请学有余力的同学用映射语言重新叙述上面的两个等式。
\ech
\end{frame}

\section{Fields}
\begin{frame}
\frametitle{\bch 场 \ech}
\bch
我们之前在流形$\mani$上的某一点定义了矢量和张量。
\cpic{0.2}{good}
假如我们能在每一点都定义矢量或张量,我们就得到了一个矢量场或张量场。
\ech
\end{frame}

\begin{frame}
\frametitle{\bch 矢量场 \ech}
\bch
流形$\mani$上的矢量场$X$定义为光滑标量场间的映射
$$
X: C^\infty_\mani \to C^\infty_\mani
$$
如何找到这样的一个映射,而且它和我们之前对给定点定义的矢量建立联系呢?
\cpic{0.15}{speechless}
\ech
\end{frame}

\begin{frame}
\frametitle{\bch 矢量场 \ech}
\bch
\begin{itemize}
\item
我们回忆,给定点$p$,矢量$X_p$是标量场到实数的映射$C^\infty_\mani \to \reals$。
\item
如果我们在每一点都定义了矢量$X_p$,对给定的标量场$f$,$X_p(f)$(随$p$变动时)显然也构成了一个标量场。
\item
定义矢量场{\color{blue}$$(X(f))(p) = X_p(f)$$}坐标表象和$X_p$一致,$X = X^\mu \partial_\mu$,不过现在是对一片点都有定义。
\item 流形$\mani$上的所有矢量场记作$\mathcal{V}_\mani$。
\end{itemize}
\ech
\end{frame}

\begin{frame}
\frametitle{\bch 矢量场的复合不是矢量场 \ech}
\bch
\begin{itemize}
\item
矢量场$X$是标量场到标量场的映射$(X(f))(p) = X_p(f)$。
\item
这让我们产生一种感觉:我们可以把得到的标量场$X(f)$再喂给另一个矢量场$Y$产生另一个新的标量场$Y(X(f))$。
\item
但并不能说复合函数$Y \circ X$是一个矢量场:使验证它不满足莱布尼兹律,也就是$(Y\circ X)_p$不是一个$T_p\mani$上的元素。
\cpic{0.15}{sad}
\item 粗略地讲,$Y\circ X$是一个二阶导数,而矢量是一个一阶导数。
\end{itemize}
\ech
\end{frame}

\begin{frame}
\frametitle{\bch 矢量场对易子 \ech}
\bch
\begin{itemize}
\item 一般我们有
$$(X\circ Y)(fg) = X(Y(fg)) = X(f Y(g) + gY(f)) $$ $$= X(f) Y(g) + f X(Y(g)) + X(g) Y(f) + g X(Y(f)) $$ $$\not= fX(Y(g)) + g X(Y(f))
$$
\item
但我们把$X$和$Y$交换位置再相减就可以消去不符合莱布尼兹律的项。
$$(XY-YX)(fg) = f(XY-YX)(g) + g(XY-YX)(f)$$
\item
因此,两个矢量场的对易子$[X,Y] \equiv XY-YX$还是矢量场。
\item
使验证:对易子的坐标表象是$$[X,Y] = \left( X^\mu \partial_\mu Y^\nu - Y^\mu \partial_\mu X^\nu\right) \partial_\nu$$
\end{itemize}
\ech
\end{frame}


\begin{frame}
\frametitle{\bch 李导数 \ech}
\bch
因为两个矢量场的对易子还是矢量场,当我们固定$X$而变化$Y$,就相当于得到了一个矢量场到矢量场的映射
$$
\mathcal{L}_X\equiv [X,]: \mathcal{V}_\mani \to \mathcal{V}_\mani
$$
具体形式为
$$
\mathcal{L}_X(Y) = [X,Y]
$$

$\mathcal{L}_X$称为{\color{purple} \bf 李导数}。
\ech
\end{frame}

\begin{frame}
\frametitle{\bch 余切矢量场 \ech}
\bch
现在事情就简单了:对一个矢量场,把每点的矢量对偶一下就得到了余切矢量场。所有余切矢量场的集合记作$\mathcal{V}^*_\mani$。
\cpic{0.2}{good}
\ech
\end{frame}

\begin{frame}
\frametitle{\bch 张量场 \ech}
\bch
同样的,如果我们对$\mani$上的每点都定义了张量$T$,就得到了一个张量场。
\cpic{0.3}{happy}
一般我们都假定张量场是光滑的。
\ech
\end{frame}

\begin{frame}
\frametitle{\bch 张量场的更严格定义 \ech}
\bch
当然上面这个定义不是特别严格,从映射的角度,一个$(s,t)$型张量场就是多重线性映射
$$
T: \underbrace{\mathcal{V}_\mani\times \mathcal{V}_\mani\times \cdots \times \mathcal{V}_\mani}_{s\text{ 个}} \times  \underbrace{\mathcal{V}^*_\mani \times \mathcal{V}^*_\mani \times \cdots \times \mathcal{V}^*_\mani}_{t\text{ 个}} \to \field_\mani
$$
也就是把$s$个矢量场和$t$个对偶矢量场映射到一个标量场。因此张量场记作
$$T(X_1,\cdots,X_s;df_1,\cdots,df_t)(p) \equiv T(X_{1p},\cdots,X_{sp};df_{1p},\cdots,df_{tp}) $$

\ech
\end{frame}

\begin{frame}
\frametitle{\bch 度规张量场 \ech}
\bch
特别的我们有度规张量场$g(X,Y)$。
\ech
\end{frame}



\section{Changing Basis}

\begin{frame}
\frametitle{\bch 局部坐标的变换 \ech}
\bch
我们知道,在同时存在两个(或多个)图$(O_a,\psi_a)$和$(O_b,\psi_b)$覆盖的开集内,两种局部坐标之间的变换为
$$
\psi_a \circ \psi_b^{-1}
$$
假设$a$代表的局部坐标为$x^\mu(p)$,$b$代表的局部坐标为$u^\nu (p)$,坐标变换映射就可以显式地写成一个$\reals^n \to \reals^n$的映射
$$
x^\mu = x^\mu ( u^\nu) 
$$
\ech
\end{frame}

\begin{frame}
\frametitle{\bch 基的变换 \ech}
\bch
$\frac{\partial}{\partial x^\mu}$是切空间$T_p\mani$的一组基。假如我们用$u$而不用$x$的话,这组基就相应地变成
$$
\frac{\partial}{\partial u^\mu}
$$
但根据多元函数的链式法则我们有{\color{blue}
$$
\frac{\partial}{\partial x^\mu} = \frac{\partial u^\nu}{\partial x^\mu} \frac{\partial}{\partial u^\nu}
$$}
于是基底的变换矩阵就是
$$
\frac{\partial x^\mu}{\partial u^\nu} 
$$
变换矩阵$J^\mu_{\ \nu} = \frac{\partial x^\mu}{\partial u^\nu} $也是流形$\mani$上的一个$(1,1)$张量场。
\ech
\end{frame}

\begin{frame}
\frametitle{\bch 切矢分量的变换 \ech}
\bch
任意的$V \in T_p \mani$按两种基展开为
$$V = X^\mu \frac{\partial}{\partial x^\mu}  = U^\nu \frac{\partial}{\partial u^\nu}$$
根据之前的基变换公式又有
$$
U^\nu \frac{\partial}{\partial u^\nu} = X^\mu \frac{\partial}{\partial x^\mu}  = X^\mu \frac{\partial u^\nu}{\partial x^\mu} \frac{\partial}{\partial u^\nu}
$$
于是我们得到任意切矢的分量的变换关系
$$
U^\nu = \frac{\partial u^\nu}{\partial x^\mu} X^\mu 
$$
\ech
\end{frame}


\begin{frame}
\frametitle{\bch 张量分量的变换 \ech}
\bch
请自行仿照上面过程推导
\begin{itemize}
\item 余切矢基底的变换:$$dx^\mu = \frac{\partial x^\mu}{\partial u^\nu} du^\nu$$
\item 余切矢分量的变换:$$U_\nu = \frac{\partial x^\mu}{\partial u^\nu} X_\mu$$
\item $(s,t)$张量分量的变换:$$U^{\nu_1\cdots \nu_t}_{\mu_1\cdots \mu_s} = \frac{\partial u^{\nu_1}}{\partial x^{\sigma_1}} \cdots \frac{\partial u^{\nu_t}}{\partial x^{\sigma_t}} \frac{\partial x^{\rho_1}}{\partial u^{\mu_1} }\cdots \frac{\partial x^{\rho_s}}{\partial u^{\mu_s}} X^{\sigma_1\cdots \sigma_t}_{\rho_1\cdots \rho_s}
$$
(提示:回忆张量分量的定义)
\end{itemize}
\ech
\end{frame}

\begin{frame}
\frametitle{\bch 我傻了,你们呢 \ech}
\bch
还记得$\partial_\mu$和$dx^\mu$是什么意思的同学请举手。
\cpic{0.3}{sad}
\ech
\end{frame}

\section{homework}
\begin{frame}
\frametitle{\bch 练习:球面的坐标变换 \ech}
\bch
我们知道描述一个球面$S^2$至少需要两个图:南极投影$P_S$和北极投影$P_N$。
\cpic{0.11}{seph}
\begin{itemize}
\item
南极投影可以完成除南极点以外所有点到$\reals^2$(下方红色平面)的映射;
\item
北极投影可以完成除北极点以外所有点到$\reals^2$(上方蓝色平面)的映射。
\end{itemize}
\ech
\end{frame}

\begin{frame}
\frametitle{\bch 练习:球面的坐标变换 \ech}
\bch
\begin{enumerate}
\item 假设已知单位球面上非极点的南极坐标(南极投影点$P_S$在红色平面上的坐标)为$(x,y)$,求这个点的北极坐标为$(u,v)$。
\item 试用$(\partial_x,\partial_y)$表示$(\partial_u,\partial_v)$。
\item 在南极放置一个$q=4\pi\epsilon_0$的点电荷,点电荷的电势是球面上的标量场$f$。求这个标量场在南极坐标和北极坐标下的表示。
\item 在南极坐标为$(2,1)$的点有一个切矢$X = \partial_x + 2 \partial_y$,求这个切矢对$f$的作用$X(f)$。写出这个切矢在北极坐标基$(\partial_u,\partial_v)$下的表达式,重新求这个切矢对$f$的作用,验证两者是相等的。说出$X(f)$的物理意义。
\end{enumerate}
\ech
\end{frame}

\end{document}