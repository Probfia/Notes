\documentclass[a4paper,11pt]{ctexart}

\usepackage{amsmath}
\usepackage{color}
\usepackage{mathrsfs}
\usepackage[colorlinks,
            linkcolor=blue,
		 urlcolor=black]{hyperref}
\usepackage{graphicx}
\usepackage{cleveref}
\usepackage{float}

\crefname{equation}{}{}
\crefname{figure}{图}{图}
\crefname{footnote}{注释}{注释}

%\cpic{<尺寸>}{<文件名>}}用于生成居中的图片。
\newcommand{\cpic}[2]{
\begin{center}
\includegraphics[scale=#1]{#2}
\end{center}
}

%\cpicn{<尺寸>}{<文件名>}{<注释>}用于生成居中且带有注释的图片,其label为图片名。
\newcommand{\cpicn}[3]
{
\begin{figure}[H]
\cpic{#1}{#2}
\caption{#3\label{#2}}
\end{figure}
}

\newtheorem{eg}{例}[section]
\newtheorem{ans}{解答}[section]

\newcommand{\beq}{\begin{equation}}
\newcommand{\eeq}{\end{equation}}
\newcommand{\bea}{\begin{equation}\begin{aligned}}
\newcommand{\eea}{\end{aligned}\end{equation}}
\newcommand{\red}{\color{red}}
\newcommand{\lag}{\mathcal{L}}
\newcommand{\diag}{\mathrm{diag}}
\newcommand{\del}{\vec{\nabla}}
\newcommand{\epv}{\epsilon_0}
\newcommand{\pfrac}[2]{\frac{\partial #1}{\partial #2}}
\newcommand{\emptyline}{\vspace{0.7cm} \\}
\newcommand{\dist}{\left| \vec{x} - \vec{x}' \right|}
\newcommand{\sgn}{\mathrm{sgn}}
\newcommand{\res}{\mathrm{res\ }}


\title{Classical Electrodynamic}
\author{Probfia}
\date{\today}


\begin{document}
\maketitle
\tableofcontents



\section{无边界的经典电磁场}
\subsection{麦克斯韦方程}
真空中与源耦合的麦克斯韦方程具有如下形式
\bea \label{maxeq}
\del \cdot \vec{E} &= \frac{\rho}{\epv} \\
\del \times \vec{E} &= -\pfrac{\vec{B}}{t} \\
\del \cdot \vec{B} &= 0 \\ 
\del \times \vec{B} &= \pfrac{\vec{E}}{t}+ \mu_0 \vec{j}
\eea
为了简化上面的方程,引入势函数$\phi$和$\vec{A}$使得
\bea
\vec{E} &= -\del \phi - \pfrac{\vec{A}}{t} \\
\vec{B} & = \del \times \vec{A}
\eea
这使得麦克斯韦方程中的$\del \times \vec{E}$和$\del \cdot \vec{B}$式自动满足。势函数的选取依然存在不定性,可以验证,若$(\phi,\vec A)$给出的物理场是$( \vec E , \vec B )$,对任意的标量场$\lambda (\vec{x},t)$,$(\phi -\partial_t \lambda, \vec A - \del \lambda )$也给出同样的物理场。这种不变性称为势函数的规范不变性,它事实上反应了我们定义出的规范场的冗余自由度。规范给出对势函数的一个具体限制从而去掉冗余自由度,我们常用洛伦兹规范$\partial_t \phi + \del \cdot \vec A = 0$。对静场,洛伦兹规范退化到库伦规范$\del \cdot \vec A = 0$。
\par
把势函数和规范要求带入麦克斯韦方程的剩下两式就得到
\bea \label{maxpot}
\frac{\partial^2 \phi}{\partial t^2} - \nabla^2 \phi &= \frac{\rho }{\epv} \\
\frac{\partial^2 \vec A}{\partial t^2} - \nabla^2 \vec A &= \mu_0 \vec j 
\eea
推导第二式需要将原等式右边的$\vec E$用势函数表示,并用洛伦兹规范条件与左边的$\del \cdot ( \del \cdot \vec A)$抵消。
\par
当源$(\rho, \vec j )$不存在时,势函数$\phi$和$\vec A$的各个分量满足自由空间中的波动方程。
\subsection{格林函数}
当空间中存在源$(\rho,\vec{j})$时,容易看出$\phi \sim \rho$,$A_i \sim j_i$。方程形式的类似性告诉我们为了具体讨论源对势函数的影响,只用讨论$\phi$对$\rho$的影响。
\par
由于电磁场的叠加原理,我们只需要考虑点源的影响。考虑方程
\beq
\frac{\partial^2 \phi}{\partial t^2} - \nabla^2 \phi = \delta(t-t_0,\vec{x}-\vec{x}_0)
\eeq
两边乘$e^{-i\omega t + i \vec{k} \cdot \vec{x}} dt d^3 \vec{x}$做积分,得到傅立叶空间中的方程为
\beq
(\vec{k}^2 - \omega^2 ) \phi = e^{-i\omega t_0 + i \vec{k} \cdot \vec{x}_0}
\eeq
傅立叶逆变换为
\beq
\phi = \int \frac{d\omega}{2\pi} \int \frac{d^3 \vec{k}}{(2\pi)^3} \frac{1}{\vec{k}^2-\omega^2} e^{-i\omega t_0 + i \vec{k} \cdot \vec{x}_0} e^{i\omega t - i \vec{k} \cdot \vec{x}}
\eeq
这个积分可以这样进行,首先,由于傅立叶空间中$\vec{x}$是一个给定的量,记$\vec{x}-\vec{x}_0$的长度为$r$,$\vec{x}-\vec{x}_0$与$\vec{k}$的夹角为$\theta$(随$\vec{k}$变化而变化,从而被对$d^3 \vec{k}$的积分积掉。第二个积分因此可以改到在球坐标下进行。球坐标下的体积元$d^3 \vec{k} = k^2 \sin \theta dk d\theta d\phi$,因此有
\beq
\phi = \int \frac{d\omega}{2\pi} e^{i\omega ( t - t_0)}  \int_{k=0}^\infty \frac{k^2 dk}{(2\pi)^3} \frac{1}{k^2 - \omega^2} \int_{\theta = 0}^\pi e^{ikr\cos \theta } \sin \theta d\theta \int_{\phi = 0}^{2\pi}d\phi
\eeq
最后两个个角向积分很容易进行
\bea
\int_{\theta = 0}^\pi e^{ikr \cos \theta} \sin \theta d\theta &= -\int_{\theta = 0}^{\pi} e^{ikr \cos \theta} d(\cos \theta) \\
&= \int_{x = -1}^1 e^{ikrx}dx = \left. \frac{e^{ikrx}}{ikr} \right|_{-1}^1 = 2\frac{\sin kr}{kr}
\eea
现在可以进行对$k$的积分,首先为了利用复变函数积分技巧,有
\beq
\frac{2}{r} \int_{k=0}^\infty \frac{k dk}{(2\pi)^3} \frac{1}{k^2 - \omega^2}\sin kr = \frac{1}{r} \mathrm{Im} \int_{k=-\infty}^\infty \frac{k dk}{(2\pi)^3} \frac{1}{k^2 - \omega^2} e^{ikr} 
\eeq
由于$r>0$,在上半复平面积分可以让$e^{ikr}$在上半围道消失,但极点$k = \pm \omega$出现在围道上,为了解决这个困境,考虑
\beq
\int \frac{k dk}{(2\pi)^3} \frac{1}{k^2 - \omega^2} e^{ikr}  = \lim_{\epsilon \to 0} \int \frac{k dk}{(2\pi)^3} \frac{1}{k^2 - \omega^2 + i \epsilon} e^{ikr} 
\eeq
这时被积函数的极点就出现在$k = \pm \sqrt{\omega^2 - i\epsilon} \simeq \pm \omega \mp i\frac{\epsilon}{2\omega}$,极点$-\omega + i \frac{\epsilon}{2\omega} $出现在上半复平面,对积分有贡献。由于
\beq
\frac{k}{k^2 - \omega^2} e^{ikr} = \frac{1}{k-(-\omega)} \frac{k}{k-\omega}e^{ikr}
\eeq
该点的留数为$\frac{1}{2(2\pi)^3} e^{-i\omega r}$,积分的最终结果因此为$\frac{1}{2(2\pi)^2 r} \cos \omega r$。\par
最后进行对$\omega$的积分,有
\bea
\phi &= 2\pi \int_{\omega = -\infty}^\infty \frac{d\omega}{2\pi} e^{i\omega (t-t_0) }\frac{ \cos \omega r }{2(2\pi)^2r} \\
& = \frac{1}{2(2\pi)^2 r}  \int_{\omega = -\infty}^\infty d\omega \cos \omega(t-t_0) \cos \omega r \\
&= \frac{1}{4(2\pi)^2 r} \int_{\omega = -\infty}^\infty d\omega \left[ \cos \omega(t-t_0 -r) + \cos \omega(t-t_0 + r)\right] \\
&= \frac{1}{4 \pi r} \frac{1}{2}\left[\delta (t-t_0-r) + \delta (t-t_0+r) \right]
\eea
其中利用了等式$\int_{-\infty}^\infty e^{ikx} dk = \int_{-\infty}^\infty \cos kx dk = 2\pi \delta(x)$。这就是势函数对电源的响应,也即源对势的格林函数。
\subsection{推迟势}
根据格林函数的公式,对一般的源$\epv \rho(\vec{x},t)$有
\bea
\phi(\vec{x},t) &=\frac{1}{4\pi \epv}  \int dt'd^3 \vec{x}' \frac{1}{|\vec{x}-\vec{x}'|} \rho(\vec{x}',t') \frac{1}{2}\left[\delta (t-t'-r) + \delta (t-t'+r) \right]\\
&= \frac{1}{4\pi \epv} \int d^3 \vec{x}' \frac{1}{|\vec{x}-\vec{x}'|}\frac{1}{2} \left[\rho(\vec{x}',t-|\vec{x}-\vec{x}'|) + \rho(\vec{x},t+|\vec{x}'-\vec{x}'|) \right]
\eea
这里发现,势函数$\phi$可以分解为推迟势
\beq
\phi_{ret} (\vec{x},t) = \frac{1}{4\pi \epv} \int d^3 \vec{x}' \frac{1}{|\vec{x}-\vec{x}'|} \rho(\vec{x}',t-|\vec{x}-\vec{x}'|) 
\eeq
和超前势
\beq
\phi_{adv} (\vec{x},t) = \frac{1}{4\pi \epv} \int d^3 \vec{x}' \frac{1}{|\vec{x}-\vec{x}'|} \rho(\vec{x}',t+|\vec{x}-\vec{x}'| )
\eeq
的组合
\beq
\phi = \frac{1}{2}(\phi_{ret} + \phi_{adv})
\eeq
\par
可以验证,推迟势和超前势都各自满足有源的势函数波动方程,它们之间任意的归一线性组合$\phi = \alpha \phi_{ret} + (1-\alpha) \phi_{adv}$也满足有源的势函数波动方程\footnote{文中这种特定组合出现的原因是上一节算格林函数时,加$i \epsilon$的方法有些任意性,比如你也可以把两个极点全部推到上面。},但超前势却面临严重的因果律问题(未来时间对现在世界的影响),因此,我们以后将只考虑推迟势。势对源的响应因此概括为
\bea \label{retp}
\phi(\vec x,t) &= \frac{1}{4\pi \epv} \int d^3 \vec{x}' \frac{1}{|\vec{x}-\vec{x}'|} \rho(\vec{x}',t-|\vec{x}-\vec{x}'|)  \\
\vec A (\vec x,t) &= \frac{\mu_0 }{4\pi } \int d^3 \vec{x}' \frac{1}{|\vec{x}-\vec{x}'|} \vec j (\vec{x}',t-|\vec{x}-\vec{x}'|) 
\eea
如果要放回所有的光速$c$,则是
\bea
\phi(\vec x,t) &= \frac{1}{4\pi \epv} \int d^3 \vec{x}' \frac{1}{|\vec{x}-\vec{x}'|} \rho(\vec{x}',t-\frac{1}{c}|\vec{x}-\vec{x}'|)  \\
\vec A (\vec x,t) &= \frac{\mu_0 }{4\pi } \int d^3 \vec{x}' \frac{1}{|\vec{x}-\vec{x}'|} \vec j (\vec{x}',t-\frac{1}{c}|\vec{x}-\vec{x}'|) 
\eea
推迟势的物理意义在于否定了超距作用:一个位于$\vec{x}'$点的电子必须经过$\sim \frac{1}{c} \left| \vec{x}' - \vec{x} \right|$的时间后才能感受到位于$\vec{x}$处的电子发生的变化,信息以光速传递。此外可以看出,$\phi \sim \frac{1}{r}$,即势在无穷远处消散。这都是符合我们的物理直觉的。

\section{静场}
\subsection{拉普拉斯方程}
假设电荷分布和电流分布都不显含时间$t$,则从\cref{retp}可以看出,势函数同样不显含时间。这样不随时间演化的场称为静场。此时\cref{maxpot}简化为
\bea
\nabla^2 \phi &= - \frac{\rho}{\epv}\\
\nabla^2 \vec A &= - \mu_0 \vec j
\eea
即势函数的各个分量都分别满足柏松方程。同样的,由于它们相仿的形式,我们暂且将目光局限在决定$\phi$的方程上。在无界空间中,$\phi$的解可以直接套用推迟势公式\cref{retp},此时$\rho$无时间依赖
\beq
\phi(\vec{x}) = \frac{1}{4\pi \epv} \int d^3 \vec{x}' \frac{\rho(\vec{x}')}{\left|\vec{x} - \vec{x}'\right|}
\eeq
但实际问题中,空间中存在着各种介质和导体,因此,电势$\phi$需要满足一些有限空间的边界条件。
\par
边界条件分为两类,第一类边界条件给出边界上$\phi$取值的直接限制
\beq
\phi \big|_{S} = f(\vec{x})
\eeq
第二类边界条件给出$\phi$梯度的法向分量,即电场垂直于边界面分量的限制,也就相当于面电荷分布的限制。
\beq
\del \phi \cdot \vec{n} \big|_S= f(\vec{x})
\eeq
静场问题归结于在边界条件下解泊松方程的问题。好的一点在于,若给定电荷分布$\rho(\vec{x})$,我们总能做变换$\psi = \phi - \frac{1}{4\pi \epv} \int d^3 \vec{x}' \frac{\rho(\vec{x}')}{\left|\vec{x} - \vec{x}'\right|}$。此时,$\psi$满足拉普拉斯方程
\beq
\nabla^2 \psi = 0
\eeq
而$\psi$的第一类(第二类)边界条件相当于原来$\phi$的边界条件减去对于自由解在边界上(的法向分量)的取值。也就是说,我们总能靠减去自由解的方法把泊松方程问题转化为拉普拉斯方程问题。

\subsection{分离变量解}
\subsubsection{直角区域分离变量}
在直角区域内,一个函数可以按正余弦函数展开。对于有界空间,电势表达为一系列傅立叶级数形式;而对无界空间,电势表达为傅立叶积分。我们分别举一个有界和无界的例子表明直角区域静电问题的解法。\begin{eg}
在正方形区域$0<x,y<L$中,除顶边$0<x<L, y=L$的电势为$V(x)$外,其余三边接地。求该正方形区域内的电势分布。
\end{eg}
\begin{ans}
设电势$\phi( x,y) = X(x) Y(y)$,带入拉普拉斯方程有
\beq
\frac{1}{X} \frac{d^2 X}{d x^2} = - \frac{1}{Y} \frac{d^2 Y}{dy^2} = -k^2
\eeq
右边取$-k^2$可以让$X(x)$取震荡形式从而满足两边的零边界条件,从而有分离变量解
\begin{equation}
X_n = a_n \sin \frac{n\pi x}{L}, n=0,1,2,\cdots
\end{equation}

\begin{equation}
Y_n = b_n \sinh \frac{n\pi y}{L}, n=0,1,2,\cdots
\end{equation}
于是写出
\begin{equation}
\phi(x,y) = \sum_{n=1}^\infty c_n \sin \frac{n\pi x}{L} \sinh \frac{n\pi y}{L}
\end{equation}
将上式代入边界条件$u(x,L) =V(x)$,得到
\begin{equation}
\sum_{n=1}^\infty c_n \sin \frac{n\pi x}{L} \sinh n\pi = V(x)
\end{equation}
上式两边同乘$\sin \cfrac{m\pi x}{L}$并对$x$从0到$L$积分(利用函数集$\{ \sin \cfrac{n \pi x}{L}\}$的正交性,相当于求$V$的傅里叶正弦展开系数),得到
\beq
\frac{L}{2}  c_n \sinh n \pi=  \int_{x=0}^L V(x) \sin \frac{n \pi x}{L} dx
\eeq
于是
\beq
\phi(x,y) = \frac{2}{L} \sum_{n=1}^\infty \frac{1}{\sinh n \pi}  \left( \int_{x=0}^L V(x) \sin \frac{n \pi x}{L} dx \right)
\sin \frac{n\pi x}{L} \sinh \frac{n\pi y }{L}
\eeq
当$V(x)=V$为一常数时
\begin{equation}
c_n = \frac{2V[1-(-1)^n]}{n\pi \sinh n \pi} \end{equation}
故正方形区域内的电势分布为
\begin{equation}
\phi(x,y) = 4V\sum_{n=1}^{\infty} \frac{1}{n\pi \sinh n \pi}  \sin \frac{n\pi x}{L} \sinh \frac{n\pi y}{L}
\end{equation}
取$n=1000$的截断作出的图像如\cref{es2}。
\cpicn{0.4}{es2}{本例题中方形区域内的电势分布}
\end{ans}
\par
下面一个例子说明无界空间中的分离变量。
\begin{eg}
在三维空间中的$xy$平面上有一个无限大平板,其上的电势给定为$V(x,y)$。求全空间的电势分布和板两侧的电荷分布。
\end{eg}
\begin{ans}
根据对称性,先分离$z$变量。设$\phi(x,y,z) = \psi(x,y) Z(z)$,有
\beq
\frac{1}{\psi} \nabla^2 \psi  = - \frac{1}{Z} \frac{d^2 Z}{dz^2} =- k^2
\eeq
右边选择$-k^2$是因为$Z(z)$必然在足够远处按指数形式$\sim e^{-k|z|}$消散而不能持续震荡。因此
\beq
\frac{d^2 Z}{dz^2} - k^2 Z = 0
\eeq
根据$z$的正负选取合适的解
\beq
Z(z) \sim \begin{cases} e^{-kz} &,z>0; \\ e^{kz} &,z<0 \end{cases}
\eeq
进一步分离变量,设$\psi(x,y) = X(x)Y(y)$,有
\bea
\frac{d^2 X}{dx^2} + k_1^2 X &= 0 \\
\frac{d^2 Y}{dy^2} + k_2^2 Y &= 0
\eea
其中$k_1^2 + k_2^2 = k^2$。$xy$平面的无界性给出模式$(k_1,k_2)$的连续谱,$X\sim e^{ik_1 x}$,$Y \sim e^{ik_2 y}$。解形式上为各个模式的线性组合
\beq
\phi(x,y,z) = \int \frac{dk_1 dk_2}{(2\pi)^2} c(k_1,k_2) e^{i(k_1 x + k_2 y)} e^{-\sqrt{k_1^2 + k_2^2} |z|}
\eeq
为了确定系数$c(k_1,k_2)$,令$z=0$有
\beq
V(x,y) = \int \frac{dk_1 dk_2}{(2\pi)^2} c(k_1,k_2) e^{i(k_1 x + k_2 y)}
\eeq
可以看出$V(x,y)$就是$c(k_1,k_2)$的傅立叶逆变换,于是
\beq
c(k_1,k_2) = \int dx dy V(x,y) e^{-i(k_1 x + k_2 y)}
\eeq
由电磁学的高斯定理知道表面面电荷分布$\sigma = \epv \vec{E} \cdot \vec{n}$,于是需要知道电场的$z$分量
\beq
E_z = - \pfrac{\phi}{z} =\sgn(z) \int \frac{dk_1 dk_2}{(2\pi)^2}\frac{c(k_1,k_2)}{\sqrt{k_1^2 + k_2^2}} e^{i(k_1 x + k_2 y)} e^{-\sqrt{k_1^2 + k_2^2} |z|}
\eeq
符号函数的出现是因为对$z$的绝对值求导。于是面电荷分布
\bea
\sigma_\pm &= \left. \epv\sgn(z) \int \frac{dk_1 dk_2}{(2\pi)^2}\frac{c(k_1,k_2)}{\sqrt{k_1^2 + k_2^2}} e^{i(k_1 x + k_2 y)} e^{-\sqrt{k_1^2 + k_2^2} |z|} \vec{e}_z \cdot (\pm \vec{e}_z) \right|_{z=0} \\ &= \epv \int \frac{dk_1 dk_2}{(2\pi)^2}\frac{c(k_1,k_2)}{\sqrt{k_1^2 + k_2^2}} e^{i(k_1 x + k_2 y)}
\eea
正负号的出现是因为上下表面的法向量方向不同,这个正负号于$\sgn(z)$抵消。我们发现上下表面带等量同种电荷,这其实从对称性可以很容易看出。\par
下面去特例对结果进行讨论。当$V(x,y) = V$为一常数时,有
\beq
c(k_1,k_2) = \int dx dy V e^{-i(k_1 x + k_2 y)} = (2\pi)^2 \delta(k_1) \delta(k_2)
\eeq
于是
\bea
\phi(x,y,z) &= \int \frac{dk_1 dk_2}{(2\pi)^2} (2\pi)^2 V \delta(k_1) \delta(k_2) e^{i(k_1 x + k_2 y)} e^{-\sqrt{k_1^2 + k_2^2} |z|} \\ 
&= V
\eea
即全空间的电势为一个常数,这乍看不符合直觉,但显然$\phi = V$是$\nabla^2 \phi = 0$符合给定边界条件的一个解,而且唯一性定理保证了它就是唯一的解。\par
再例如$V(x,y) = \frac{1}{1+x^2}$。有
\beq
c(k_1,k_2) = \int dx dy \frac{1}{1+x^2} e^{-i(k_1 x + k_2 y)} = 2\pi \delta(k_2) \int dx \frac{1}{1+x^2} e^{-ik_1 x}
\eeq
积分可以用复变函数的方法进行,注意到被积函数$f(z) = \frac{1}{1+z^2} e^{-ik_1 z}$下有两个极点$\res f(i) = \frac{e^{-k_1}}{2i}$,$\res f(-i) = - \frac{e^{k_1}}{2i}$。当$k_1 > 0$,积分在上半围道进行,结果为$\pi e^{-k_1}$;当$k_1 < 0 $,积分在下半围道进行,这时在实轴上$x$从正无穷走向负无穷,因此需要在$2\pi i \res f(-i)$上再添一个符号才是原积分的结果,为$\pi e^{k_1}$。两种情况积分可以统一表示成$\pi e^{-|k_1|}$。故
\beq
c(k_1,k_2) = 2\pi^2 \delta(k_2) e^{-|k_1|}
\eeq
于是全空间电势分布
\bea
\phi(x,y,z) &= \int \frac{dk_1 dk_2}{(2\pi)^2} 2\pi^2 \delta(k_2) e^{-|k_1|} e^{i(k_1 x + k_2 y)} e^{-\sqrt{k_1^2 + k_2^2} |z|} \\ 
&= \int_{k_1 = -\infty}^\infty dk_1 \frac{1}{2} e^{-|k_1|} e^{ik_1 x} e^{-|k_1| |z|} \\
&= \mathrm{Re}\ \int_{k_1 = 0}^\infty dk_1 e^{-k_1(1+|z|-ix)}  \\ 
&= \mathrm{Re}\ \frac{1}{1+z-ix} = \frac{1+|z|}{(1+|z|)^2 + x^2}
\eea
电场分布
\beq
\vec{E} = -\pfrac{\phi}{x}\vec{e}_x - \pfrac{\phi}{z} \vec{e}_z  = \sgn z\left ( \frac{2 x (z+1)}{\left(x^2+(z+1)^2\right)^2}\vec{e}_x +\left[\frac{2 (z+1)^2}{\left(x^2+(z+1)^2\right)^2}-\frac{1}{x^2+(z+1)^2}\right] \vec{e}_z \right)
\eeq
表面电荷分布
\beq
\sigma(x,y) = \left. \epv \vec{E} \cdot \vec{e}_z \right|_{z=0} = \epv \frac{2-x^2}{x^4}
\eeq
\par
容易计算的情形是$V(x,y) = V(\sqrt{x^2 + y^2}) \equiv V(r)$的情形。给定$\vec{k} = (k_1,k_2)$,记$\theta$为$\vec{k}$和$\vec{r}$之间的夹角。容易想到,现在$c(k_1,k_2)$也只依赖于$\vec{k}$的模长$k$。在极坐标下积分,有
\bea
c(k) &= \int_{r=0}^\infty rdr \int_{\theta = 0}^{2\pi} V(r) e^{-ikr\cos \theta} d\theta \\
&= 2\pi \int_{r=0}^\infty rdr V(r) J_0(kr)
\eea
其中引入了贝塞尔函数$J_0$及其积分表达式$J_0(x) = \frac{1}{2\pi}\int_{-\pi}^\pi  e^{ix\cos \theta} d\theta$。\par
全空间的电场于是为
\bea
\phi(r,z) &= \int_{k=0}^\infty \frac{kdk}{(2\pi)^2} \int_{\varphi=0}^{2\pi} d\varphi \left(2\pi \int_{r'=0}^\infty r'dr' V(r') J_0(kr) \right) e^{ikr\cos \varphi} e^{-k|z|} \\
&= \int_{k=0}^\infty kdk e^{-k|z|} J_0 (kr)  \int_{r'=0}^\infty r' dr' V(r') J_0(kr')
\eea
此处必须知道$V(r)$的具体表达式,否则无法继续再化简。例如当$V(r) = e^{-\frac{r^2}{2a^2}}$,后一个积分可以这样计算:回忆第一类贝塞尔函数的级数定义
\beq
J_\nu (x) = \sum_{m=0}^\infty \frac{(-1)^m}{m! (m+\nu)!} \left( \frac{x}{2} \right)^{2m + \nu}
\eeq
于是有
\bea
\int_{x=0}^\infty x e^{-\frac{x^2}{2}} J_0(\alpha x) dx &= \int_{x=0}^\infty x e^{-\frac{x^2}{2}} \sum_{m=0}^\infty \frac{(-1)^m}{(m!)^2} \left( \frac{\alpha x}{2} \right)^{2m} dx\\
&=\sum_{m=0}^\infty \frac{(-1)^m}{(m!)^2} \left( \frac{\alpha^2}{2} \right)^{m} \int_{x=0}^\infty x e^{-\frac{x^2}{2}} \left( \frac{x^2}{2}\right)^m \\
&(u = \frac{x^2}{2})\\
&=\sum_{m=0}^\infty \frac{(-1)^m}{(m!)^2} \left( \frac{\alpha^2}{2} \right)^{m}  \int_{u=0}^\infty e^{-u} u^m du \\
&= \sum_{m=0}^\infty \frac{(-1)^m}{(m!)^2} \left( \frac{\alpha^2}{2} \right)^{m} m!\\
&= \sum_{m=0}^\infty \frac{(-1)^m}{m!} \left( \frac{\alpha^2}{2} \right)^{m} = e^{-\frac{\alpha^2}{2}}
\eea
再作代换$x \to \frac{r}{a}$,$\alpha \to ak$,就得到
\beq
\int_{r=0}^\infty e^{-\frac{r^2}{2a^2}} J_0(kr) rdr = a^2 e^{-\frac{a^2k^2}{2}}
\eeq
再带入积分就得到
\beq
\phi(r,z) = a^2 \int_{k=0}^\infty kdk e^{-k|z|-\frac{1}{2} a^2 k^2} J_0 (kr)
\eeq
这里还是化简不了,不过我们依然是得到了一个积分表达式。可以用这个表达式进行数值计算观察势的行为。结果如\cref{2_2p1}。
\cpicn{0.25}{2_2p1}{本例中电势随$(r,z)$的分布}
可以看到,在$r$方向,电势大致呈高斯型,特别是在$r=0$处电势的导数为0,体现出原有电势分布的对称性;而在$z$方向,电势大致是按$\sim e^{-z/a}$下降的,这点可以从$z=0$处电势的导数非零看出。
\par
也可以将积分表达式展开成双级数,有
\bea
\phi(r,z) &= a^2 \int_{k=0}^\infty kdk e^{-k|z|-\frac{1}{2} a^2 k^2} J_0 (kr) \\
&=a^2 \int_{k=0}^\infty k e^{-k|z|-\frac{1}{2} a^2 k^2} \sum_{m=0}^\infty \frac{(-1)^m}{(m!)^2} \left( \frac{kr}{2} \right)^{2m} dk \\
&=a^2 \sum_{m=0}^\infty \frac{(-1)^m}{(m!)^2} \left( \frac{r^2}{2} \right)^{m}  \int_{k=0}^\infty k e^{-k|z|-\frac{1}{2} a^2 k^2} \left( \frac{k^2}{2} \right)^{m} dk \\
&(u=\frac{k^2}{2})\\
&=a^2 \sum_{m=0}^\infty \frac{(-1)^m}{(m!)^2} \left( \frac{r^2}{2} \right)^{m}  \int_{u=0}^\infty e^{-\sqrt{2u}|z|-a^2u} u^m du \\
&= a^2 \sum_{m=0}^\infty \frac{(-1)^m}{(m!)^2} \left( \frac{r^2}{2} \right)^{m}  \int_{u=0}^\infty e^{-a^2u} \sum_{n=0}^\infty \frac{1}{n!} (-\sqrt{2u}|z|)^n u^m du \\
&= a^2 \sum_{m=0}^\infty \sum_{n=0}^\infty \frac{(-1)^{m+n}}{(m!)^2} \left( \frac{r^2}{2} \right)^{m} \frac{1}{n!} (\sqrt{2}|z|)^n \int_{u=0}^\infty e^{-a^2u} u^{\frac{n}{2}+m} du\\
&(v = a^2 u)\\
&= a^2 \sum_{m=0}^\infty \sum_{n=0}^\infty \frac{(-1)^{m+n}}{(m!)^2} \left( \frac{r^2}{2} \right)^{m} \frac{1}{n!} (\sqrt{2}|z|)^n \left(\frac{1}{a^2}\right)^{\frac{n}{2} + m +1} \int_{v=0}^\infty e^{-v} v^{\frac{n}{2}+m} dv\\
&= \sum_{m=0}^\infty \sum_{n=0}^\infty \frac{(-1)^{m+n}}{(m!)^2} \left( \frac{r^2}{2a^2} \right)^{m} \frac{1}{n!} \left(\sqrt{2}\frac{|z|}{a}\right)^n \int_{v=0}^\infty e^{-v} v^{\frac{n}{2}+m} dv\\
&=\sum_{m=0}^\infty \sum_{n=0}^\infty \frac{(-1)^{m+n}\left(m+ \frac{n}{2}\right)!}{(m!)^2 n!} \left( \frac{r^2}{2a^2} \right)^{m}  \left(\sqrt{2}\frac{|z|}{a}\right)^n 
\eea
非常之壮观。最低一阶近似为
\beq
\phi \simeq 1 - \frac{r^2}{2a^2} - \sqrt{\frac{\pi}{2}} \frac{|z|}{a},\ r,|z| \ll a
\eeq



\end{ans}

\subsubsection{扇形区域分离变量}
采用柱坐标在扇形或圆形区域分离变量时,解的形式依赖于$z$轴依赖是否存在。当电势不依赖于$z$时,可以假设$\phi(r,\theta) = R(r) \Theta(\theta)$,进而有
\beq
\Theta \frac{1}{r} \frac{d}{dr} \left( r \frac{dR}{dr}\right) + R \frac{1}{r^2} \frac{d^2 \Theta}{d \theta^2} = 0
\eeq
进一步整理得到
\beq
\frac{r}{R} \frac{d}{dr} \left( r \frac{dR}{dr} \right) = - \frac{1}{\Theta} \frac{d^2 \Theta}{d \theta^2} = \nu^2
\eeq
角向方程的解写成
\beq
\Theta(\theta) = c_\nu \cos \nu \theta + s_\nu \sin \nu \theta
\eeq
当$\nu=0$时,由于二阶常微分线性方程总有两个线性无关解,$\Theta = c_0$不足以构成完备解,但容易验证,$\Theta = c_0 + s_0 \theta$也满足$\nu=0$时的方程。
$\nu$的取值与边界条件有关,当形区域为圆形区域时,需要添加自然周期性边界条件(PBC)$\Theta(\theta + 2\pi) = \Theta(\theta)$,这要求$\nu$只能取整数,且$s_0=0$。
\par
径向方程整理为
\beq
r^2 \frac{d^2 R}{dr^2} + r \frac{dR}{dr} - \nu^2 R = 0
\eeq
这个方程可以简单求解如下:假设$R \sim r^p$,有
\beq
p(p-1) + p - \nu^2 = 0
\eeq
当$\nu \not= 0$时,解为$R \sim r^\nu + r^{-\nu}$;当$\nu=0$,由于二阶常微分线性方程总有两个线性无关解,$R \sim \mathrm{const.}$不足以构成完备解,但容易验证,$R \sim \ln r$也满足$\nu=0$时的方程。
\par
于是,无$z$依赖时,扇形区域分离变量的形式解写成
\beq
\phi(r,\theta) = (a_0 + b_0 \ln r)(c_0 + s_0 \theta) + \sum_{\nu} \left(a_\nu r^\nu + b_\nu \frac{1}{r^\nu} \right) \left( c_\nu \cos \nu \theta + s_\nu \sin \nu \theta \right)
\eeq
当考虑的空间区域包括原点时,电势在原点不发散要求$b$项全部为0;剔除原点时,所有项都可能对最终表达式做贡献。
\begin{eg}
空间中两个半无限大金属板张成一个二面角$\beta$,下板接地,上板固定电势为$V$,两板间的接触线绝缘。求二面角内的电势分布。
\end{eg}
\begin{ans}
显然,在通解中取出
\beq
\phi(r,\theta) = \frac{\theta}{\beta} V
\eeq
就满足边界条件和拉普拉斯方程。
\end{ans}
\begin{eg}
空间内有一半径为$a$的无限长圆柱面,其轴与$z$轴重合。圆柱面上的电势$V(\theta)$给定且只依赖于极角。求全空间的电势分布。
\end{eg}
\begin{ans}
按之前的讨论,在$r<a$范围内,解按
\beq
\phi(r,\theta) = \sum_{m=0}^\infty \left( \frac{r}{a} \right)^m \left( c_m \cos m \theta + s_m \sin m \theta \right)
\eeq
展开。令$r=a$得到
\beq
V(\theta) = \sum_{m=0}^\infty  \left( c_m \cos m \theta + s_m \sin m \theta \right)
\eeq
于是
\bea
c_m &= \frac{1}{\pi} \int_{\theta=0}^{2\pi} V(\theta) \cos m \theta d\theta \\
s_m &= \frac{1}{\pi} \int_{\theta=0}^{2\pi} V(\theta) \sin m \theta d\theta
\eea
圆柱面以外的解只需要将$\frac{r}{a}$替换为$\frac{a}{r}$。
\par
例如,当$V(\theta) = V_0 \sin 2\theta$时,展开系数只有$s_2 = V_0$非零,故解为
\beq
\phi(r,\theta) = V_0 \min \{ \left( \frac{r}{a}\right)^2 ,\left( \frac{a}{r} \right)^2\}  \sin 2\theta
\eeq
对应的图像如$\cref{2_3p1}$。
\cpicn{0.12}{2_3p1}{$V(\theta) = V_0 \sin 2\theta$时全空间的电势分布}
\par
再例如$V(\theta) = \begin{cases}
V_0 &, 0<\theta \leq \pi \\
0 &, \pi <\theta \leq 2\pi
\end{cases}$
可以很容易算出
\bea
c_0 &= \frac{V_0}{2}\\
c_n&=0,\ n \geq 1 \\
s_n &= V_0\frac{1 - (-1)^n}{n\pi}
\eea
于是
\beq
\phi(r,\theta) = \frac{V_0}{2}+V_0 \sum_{n=1}^\infty  \frac{1 - (-1)^n}{n\pi} \min \{ \left(\frac{r}{a}\right)^n , \left(\frac{a}{r} \right)^n \} \sin n \theta
\eeq

\end{ans}
\par
下面一个例子说明扇形区域下的解法。
\begin{eg}
空间中一扇形区域边界为两个半无限大接地金属板,其张角为$\beta$。求扇形区域内电势分布和金属板上电荷分布的通解,并讨论$\beta \to 0$,$\beta = \pi$和$\beta \to 2\pi$的行为。
\end{eg}
\begin{ans}
由于原点包括在内,形式上的分离变量解为
\beq
\phi(r,\theta) = a_0 + \sum_\nu (a_\nu \cos \nu \theta + b_\nu \sin \nu \theta) r^\nu
\eeq
边界条件为
\bea
\phi(r,\theta=0) &= a_0 \\
\phi(r,\theta=\beta) &= a_0
\eea
可知$a_\nu = 0,\ \nu>0$,$\nu = n\pi \theta /\beta$。于是通解为
\beq
\phi(r,\theta) = a_0 + \sum_{n=1}^\infty b_n r^{\frac{n\pi }{\beta}} \sin \frac{n\pi \theta}{\beta}
\eeq
取一阶近似得到
\beq
\phi(r,\theta) \simeq a_0 + a_1 r^\frac{\pi }{\beta} \sin \frac{\pi \theta}{\beta}
\eeq
于是电场为
\bea
\vec{E} &\simeq -\pfrac{\phi}{r} \vec{e}_r - \frac{1}{r} \pfrac{\phi}{\theta} \vec{e}_\theta \\
&=-\frac{\pi }{\beta}a_1 r^{\frac{\pi }{\beta}-1} \sin \frac{\pi  \theta}{\beta} \vec{e}_r-\frac{\pi }{\beta} a_1 r^{\frac{\pi }{\beta}-1} \cos \frac{\pi \theta }{\beta} \vec{e}_\theta
\eea
电场大小$E \sim r^{\frac{\pi }{\beta}-1}$,板上的面电荷密度为
\beq
\sigma = \epv \vec{E} \cdot \vec{e}_\theta  = -\frac{\pi }{\beta} a_1 r^{\frac{\pi }{\beta}-1}\sim r^{\frac{\pi}{\beta} - 1}
\eeq
\par
当$\beta = \epsilon \to 0$,图像上对应一个带狭缝的无限长金属棒。狭缝上的电荷分布$\sigma \sim r^{1/\epsilon}$,即在原点处,狭缝边缘几乎没有电荷,跟容易想出这是由于该处电荷间的强烈排斥导致的。
\par
当$\theta = \pi$,图像上对应一个无限大金属板。此时的电场分布为
\beq
\vec{E} \simeq - a_1 \sin \theta \vec{e}_r - a_1 \cos \theta \vec{e}_\theta = -a_1 \vec{e}_y
\eeq
对应一个匀强电场。从电磁学知识我们已经知道,一个无限大带电金属板对应的电场就是一个匀强电场,于是系数$a_1=-E$。再看面电荷分布,有
\beq
\sigma = - \epv a_1 =\epv E
\eeq
与熟知的结果$E = \frac{\sigma}{\epv}$一致。
\par
当$\theta \to 2\pi$,图像上对应一个二维劈尖,面电荷分布和电场大小$\sigma \sim E\sim \frac{1}{\sqrt{r}}$,即在尖端处,电场和面电荷都非常大,之后随远离尖端而衰减,符合尖端放电的物理事实。
\end{ans}
下面一个例子包括原点。
\begin{eg}
在$xz$平面上有一个无限大金属板,在原点处有一个无限长半圆柱状的突起,圆柱的半径为$a$。给出区域内空间电势分布的通解,并讨论一阶近似下的解。
\end{eg}
\begin{ans}
除去原点下,极坐标下的通解可以写成
\beq
\phi(r,\theta) = a_0 + \sum_{\nu} \left(a_\nu r^\nu + b_\nu \frac{1}{r^\nu} \right) \left( c_\nu \cos \nu \theta + s_\nu \sin \nu \theta \right)
\eeq
其中正比于$\theta$的第一项由于周期性差被舍去。边界条件为
\bea
\phi(r,\theta = 0) &= a_0\\
\phi(r=a,\theta)&=a_0\\
\phi(r,\theta = \pi) &= a_0
\eea
第二个边界条件给出
\beq
\sum_\nu \left(a_\nu a^\nu + b_\nu \frac{1}{a^\nu} \right) \left( c_\nu \cos \nu \theta + s_\nu \sin \nu \theta \right) = 0
\eeq
只能有
\beq
b_\nu = - a_\nu a^{2\nu}
\eeq
才能使得上式对所有$\theta$成立。第一个边界条件给出
\beq
\sum_\nu \left(a_\nu r^\nu + b_\nu \frac{1}{r^\nu} \right) c_\nu = 0
\eeq
只能取$c_\nu =0$。第三个边界条件给出
\beq
\sum_{\nu} \left(a_\nu r^\nu + b_\nu \frac{1}{r^\nu} \right) s_\nu \sin \nu \pi = 0
\eeq
只能取$\nu = n$。于是通解为
\beq
\phi(r,\theta) = a_0 + \sum_{n=1}^\infty a_n \left(r^n - \frac{a^{2n}}{r^n}\right) \sin n \theta
\eeq
当$a=0$时恢复到上一例的解。一阶近似
\beq
\phi(r,\theta) \simeq a_0 + a_1 \left(r-\frac{a^2}{r}\right) \sin \theta
\eeq
对应图像如\cref{2_3p2},可以看到,在远处,电势分布几乎上一个匀强电场。
\cpicn{0.3}{2_3p2}{该例中的电势分布,取$a_1 = -1$}
电场
\bea
\vec{E} &= -\pfrac{\phi}{r} \vec{e}_r - \frac{1}{r} \pfrac{\phi}{\theta} \vec{e}_\theta \\
&= -a_1\left( 1 + \frac{a^2}{r^2} \right) \sin \theta \vec{e}_r - a_1 \left(1 - \frac{a^2}{r^2}\right)\cos \theta \vec{e}_\theta \\
&= -a_1 \vec{e}_y -a_1 \frac{a^2}{r^2} (\sin \theta \vec{e}_r-\cos \theta \vec{e}_\theta)
\eea
在远处,第二项消失,对应无限大平板的电场$E\vec{e}_y$,于是有$a_1 = -E$。第二项$\sim a^2$,描述了柱状突起对外电场的响应。
\par
面电荷分布为
\bea
\sigma(r,\theta = 0) = \sigma(r,\theta = \pi) &= \epv \vec{E}\cdot \vec{e}_\theta \big|_{\theta = 0} = \epv E\left(1 - \frac{a^2}{r^2}\right) \\
\sigma(r=a,\theta) &= \epv \vec{E} \cdot \vec{e}_r \big|_{r=a} = \epv E\sin \theta
\eea
\par


\end{ans}
\subsubsection{球形区域分离变量}
首先讨论简单的情况:假设空间具有绕$z$轴的选择不变性而使得电势不依赖于经度角$\varphi$,拉普拉斯方程的分离变量就相对简单
\beq
\frac{1}{r^2} \pfrac{}{r} (r^2 \pfrac{\phi}{r}) + \frac{1}{r^2\sin \theta} \frac{\partial^2 \phi}{\partial \theta^2} = 0
\eeq
分离变量的结果应该耳熟能详了
\beq
\phi(r,\theta) = \sum_{\ell = 0}^\infty \left(a_\ell r^\ell + b_\ell \frac{1}{r^{\ell + 1}}\right) P_\ell (\cos \theta)
\eeq
其中$P_\ell$为勒让德多项式
\beq
P_\ell (x) = \frac{1}{2^\ell \ell!} \frac{d^\ell}{dx^\ell}(x^2 - 1)^\ell
\eeq
勒让德多项式在区间$(-1,1)$上按
\beq
\int_{-1}^1 x P_n(x) P_m(x) dx = \frac{2}{2n + 1} \delta_{mn}
\eeq
正交归一化。此外,勒让德多项式还有著名的母函数公式
\beq
\sum_{\ell=0}^\infty  P_\ell(x) r^\ell  = \frac{1}{\sqrt{r^2 +1 -2rx}}, r<1
\eeq
\begin{eg}
空间中存在一个半径为$a$的球壳,其上的电势给定,为$V(\theta)$,其中$\theta$为纬度角。使讨论全空间的电势分布。
\end{eg}
\begin{ans}
当$r<a$时,分离变量通解为
\beq
\phi(r,\theta) = \sum_{\ell = 0}^\infty a_\ell \left(\frac{r}{a}\right)^\ell P_\ell (\cos \theta)
\eeq
有
\beq
V(\theta) = \sum_{\ell=0}^\infty a_\ell P_\ell (\cos \theta)
\eeq
两边乘以$P_m(\cos \theta) \sin \theta d \theta$积分,利用正交归一性有
\beq
a_\ell \ =\frac{2\ell + 1}{2} \int_{\theta = 0}^\pi V(\theta) P_\ell(\cos \theta) \sin \theta d\theta
\eeq
$r>a$的情形只需要将$(\frac{r}{a})^\ell$换成$(\frac{a}{r})^{\ell+1}$。
\par
讨论一阶近似的结果是有益的。这时有
\beq
\phi(r,\theta) = 
\begin{cases}
a_0 + a_1 \frac{r}{a} \cos \theta &, r<a \\
a_0 + a_1 \frac{a^2}{r^2} \cos \theta &, r>a
\end{cases}
\eeq
球外的电场为
\bea
\vec{E} &= -\pfrac{\phi}{r} \vec{e}_r - \frac{1}{r} \pfrac{\phi}{\theta} \vec{e}_\theta \\
&= \frac{2a_1 a^2}{r^3} \cos \theta \vec{e}_r + \frac{a_1 a^2}{r^3} \sin \theta \vec{e}_\theta
\eea
在电磁学中我们已经学过$z$轴上的偶极子$\vec{p} = p\vec{e}_z$在远处产生的电场为$\vec{E} = \frac{1}{4\pi \epv} \left( \frac{2p\cos \theta}{r^3} \vec{e}_r + \frac{p\cos \theta}{r^3} \vec{e}_\theta \right) $。从而我们可以看出,展开系数$a_1$对应球壳的偶极矩
\beq
p = 4\pi \epv a_1 a^2
\eeq
这个关系也可以先得到球面的面电荷密度再积分得到。球内的极化强度,即球体内的平均偶极矩为
\beq
P = \frac{p}{\frac{4\pi}{3}a^3} = 3\epv a_1
\eeq
而球内的电场为
\beq
\vec{E} = -\del \left(a_0 + a_1 \frac{r}{a} \cos \theta \right) = -\del \left(a_0 + a_1 \frac{z}{a} \right) = -\frac{a_1}{a} \vec{e}_z
\eeq
于是我们得到一阶近似的情况下,球内的电场与极化强度的关系
\beq
\vec{E} = - \frac{\vec P}{3\epv}
\eeq
\par
\end{ans}
\begin{eg}
将一个半径为$a$的接地金属球放入匀强电场$\vec E$中,求空间的电势分布,并求金属球的等效偶极矩。
\end{eg}
\begin{ans}
取球心为原点,电场方向沿$z$轴正向。在球内,电场和电势恒为零;在金属球表面,电场的法向分量为零;在远处,电势$\phi \to - Er\cos \theta$。边界条件为
\beq
\begin{cases}
\pfrac{\phi}{r}\big|_{r=a} = 0 \\
\phi \big|_{r=a} = 0 \\
\phi(r\gg a,\theta) \to - Er\cos \theta
\end{cases}
\eeq
设$u = \phi + Er\cos \theta$,代入得$u$的边界条件为
\beq
\begin{cases}
\pfrac{u}{r}\big|_{r=a} = E \cos \theta \\
u\big|_{r=a} = Ea \cos \theta \\
u(r\gg a,\theta) \to 0
\end{cases}
\eeq
$u$必然有分离变量形式解
\beq
u(r,\theta) = \sum_{\ell = 0}^\infty a_\ell \left(\frac{a}{r}\right)^{\ell+1} P_\ell (\cos \theta)
\eeq
代入第二个边界条件可知
\beq
a_0 + a_1 \cos \theta + \sum_{\ell \geq 2}^\infty a_\ell  P_\ell (\cos \theta) =  Ea \cos \theta
\eeq
只能有
\beq
a_1 = Ea
\eeq
其余$a_\ell$为零。
于是
$u(r,\theta) = E \frac{a^3}{r^2} \cos \theta $,球外电势分布为
\beq
\phi(r,\theta) = -E\left( r - \frac{a^3}{r^2}\right) \cos \theta
\eeq
由偶极子电势公式$\phi = \frac{1}{4\pi \epv} \frac{p\cos \theta}{r^2}$,等效偶极子为
\beq
\vec p = -4\pi \epv a^3 \vec E
\eeq
体积平均为
\beq
\vec P = - 3 \epv \vec E
\eeq
我们发现,极化偶极子与外电场成正比,且反向。这是我们讨论物质极化的一个重要模型。
\end{ans}

\par
当空间不再沿$z$轴旋转对称时,电势的分离变量解只能按球谐函数展开成如下形式
\beq
\phi(r,\theta,\varphi) =\sum_{l=0}^\infty \sum_{m=-l}^{l} \left( a_{lm} r^l + b_{lm} \frac{1}{r^{l+1}}\right) Y_{lm}(\theta,\varphi)
\eeq
其中球谐函数的定义是
\beq
Y_{lm}(\theta,\varphi) = \sqrt{ \frac{2l+1}{4\pi} \frac{(l-m)!}{(l+m)!}} P_l^m(\cos \theta) e^{im\varphi}
\eeq
而连带勒让德函数又定义为
\beq
P_l^m (x) = (-1)^m (1-x^2)^{\frac{m}{2}} \frac{d^m}{dx^m} P_l(x)
\eeq
球谐函数的正交归一化为
\beq
\int_{\varphi = 0}^{2\pi} \int_{\theta = 0}^\pi Y_{lm}(\theta,\varphi) Y_{l'm'}*(\theta,\varphi) \sin \theta d\theta d \varphi = \delta_{ll'} \delta{mm'}
\eeq
即在单位球面上对面积的积分正交归一。
\begin{eg}
如上例,但此时球壳上的电势$V=V(\theta,\varphi)$,并取前2阶近似讨论解的行为。
\end{eg}
\begin{ans}
当$r<a$时,形式上写出解为
\beq
\phi(r,\theta,\varphi) =\sum_{l=0}^\infty \sum_{m=-l}^{l} a_{lm} \left( \frac{r}{a} \right)^lY_{lm}(\theta,\varphi)
\eeq
有
\beq
V(\theta,\varphi) = \phi(r=a,\theta,\varphi) = \sum_{l=0}^\infty \sum_{m=-l}^{l} a_{lm} Y_{lm}(\theta,\varphi)
\eeq
两边乘以$\sin \theta d\theta d \varphi$对整个球面积分,利用正交归一性得到
\beq
a_{lm} = \int \sin \theta d \theta d \varphi V(\theta,\varphi) Y_{lm}(\theta,\varphi)
\eeq
球面外的解只需要将$\left(\frac{r}{a}\right)^l$替换成$\left( \frac{a}{r}\right)^{l+1}$。
 
 
\end{ans}


\subsection{电像法与格林函数}
电像法可以用来求解点电荷与对称性边界的问题。\par
格林函数的起点为积分公式
\bea
\int_V d^3 \vec{x} (u \nabla^2 v -v \nabla^2 u) &= \int_V d^3 \vec{x} \del \cdot( u \del v -v \del u) \\
&= \int_{\partial V} dS \vec n \cdot ( u \del v -v \del u) = \int_{\partial V} dS ( u \pfrac{v}{n} -v \pfrac{u}{n})
\eea
接下来考虑边界问题
\beq
\nabla^2 u = -\frac{\rho(\vec x)}{\epsilon_0}
\eeq
\beq
\begin{cases}
u\big|_{S} = V(\vec x) ;\\
\text{或} \\
\pfrac{u}{n}\big|_{S} = \frac{\sigma(\vec{x})}{\epsilon_0}
\end{cases}
\eeq
我们考虑同样边界的点源问题
\beq
\nabla^2 G = -\delta(\vec x - \vec{x}')
\eeq
\beq
\begin{cases}
G\big|_{S} = 0 ;\\
\text{或} \\
\pfrac{G}{n}\big|_{S} = 0
\end{cases}
\eeq
这一解记作$G(\vec x; \vec{x}')$,显然有$G(\vec x; \vec{x}') = G(\vec{x}'; \vec{x}) $。假如$G$已经由其他方法求得,就可以代入一开始的积分公式。左边项
\bea
\int_V d^3 \vec{x} (u \nabla^2 G -G \nabla^2 u)  &= \int_V d^3 \vec{x} (-u\delta(\vec x - \vec{x}') + G \frac{\rho(\vec x)} {\epsilon_0}) \\
&= -u(\vec{x}') + \int_V d^3 \vec{x} G(\vec x; \vec{x}') \frac{\rho(\vec x)} {\epsilon_0}\\
\eea
\par
对第一类边界条件,右边项
\beq
\int_{\partial V} dS ( u \pfrac{G}{n} -G \pfrac{u}{n}) = \int_{\partial V} dS V(\vec x)  \pfrac{G}{n} 
\eeq
于是
\beq
u(\vec{x}') = \int_V d^3 \vec{x} G(\vec x; \vec{x}') \frac{\rho(\vec x)} {\epsilon_0} - \int_{\partial V} dS  \pfrac{G}{n} V(\vec x) 
\eeq
习惯上用$\vec{x}$表示场点;$\vec{x}'$表示源点,交换得到
\beq
u(\vec{x}) = \int_V d^3 \vec{x}' G(\vec x; \vec{x}') \frac{\rho(\vec{x}')} {\epsilon_0} - \int_{\partial V} dS' \pfrac{G(\vec x; \vec{x}')}{n'} V(\vec x')  
\eeq
对第二类边界条件,右边项
\beq
\int_{\partial V} dS ( u \pfrac{G}{n} -G \pfrac{u}{n}) = -\int_{\partial V} dS  \frac{\sigma(\vec{x})}{\epsilon_0}G(\vec x; \vec{x}' )
\eeq
于是
\beq
u(\vec{x}') = \int_V d^3 \vec{x} G(\vec x; \vec{x}') \frac{\rho(\vec x)} {\epsilon_0} + \int_{\partial V} dS \frac{\sigma(\vec{x})} {\epsilon_0}G(\vec x; \vec{x}' )
\eeq
或
\beq
u(\vec{x}) = \int_V d^3 \vec{x}' G(\vec x; \vec{x}') \frac{\rho(\vec{x}')} {\epsilon_0} - \int_{\partial V} dS' \frac{\sigma(\vec{x}')}{\epsilon_0} G(\vec x; \vec{x}' )
\eeq
\par
特别地,当空间中无自由电荷,而给定空间中某一表明上的电势分布$V(\vec x)$时,可以得到
\beq
u(\vec{x}) = - \int_{S} dS' \pfrac{G(\vec x; \vec{x}')}{n'} V(\vec x')  
\eeq
当$V(\vec x ) = \delta ( \vec x- \vec x ')$时,得到的空间电势分布称为电势格林函数$G_V$,有
\beq
G_V(\vec{x};\vec{x}') = -\pfrac{G(\vec x; \vec{x}')}{n'}
\eeq

\subsection{多极展开}
多极展开研究电荷远处的电势。我们从自由空间静场解
\beq
\phi(\vec{x}) = \frac{1}{4\pi \epv} \int d^3 \vec{x}' \frac{\rho(\vec{x}')}{\left|\vec{x} - \vec{x}'\right|}
\eeq
出发,假定电荷分布在一个原点附近$\sim a$的范围内,我们考虑$|\vec x| \gg a$的电势。记
\beq
r = \left|\vec{x} - \vec{x}'\right| = \sqrt{(x_k - x_k') (x_k - x_k')}
\eeq
我们计算$\frac{1}{r}$ 按$x_i'$的展开系数。首先有
\beq
\pfrac{r}{x_i'} = \frac{-2\delta_{ik}(x_k - x_k')}{2\sqrt{(x_k - x_k') (x_k - x_k')}}  = \frac{x_i' - x_i}{r}
\eeq
于是
\beq
\pfrac{}{x_i'}\left( \frac{1}{r}\right) = -\frac{1}{r^2} \pfrac{r}{x_i'} = \frac{x_i - x_i'}{r^3} \sim O(\frac{1}{r^2})
\eeq
\bea
\frac{\partial^2}{\partial x_i' \partial x_j'} \left( \frac{1}{r}\right) &= \pfrac{}{x_j'} \pfrac{}{x_i'}\left( \frac{1}{r}\right) \\
&= \pfrac{}{x_j'} \left(\frac{x_i - x_i'}{r^3}\right) \\
&= \frac{-\delta_{ij}}{r^3} + (x_i - x_i')\left( - \frac{3}{r^4}\right) \pfrac{r}{x_j'} \\
&= \frac{-\delta_{ij}}{r^3} + (x_i - x_i')\left( - \frac{3}{r^4}\right) \frac{x_j' - x_j}{r} \\
&= \frac{-\delta_{ij}}{r^3} + \frac{3(x_i - x_i')(x_j - x_j')}{r^5}\sim O(\frac{1}{r^3})
\eea
于是$\frac{1}{r}$展开到$\sim \frac{1}{r^3}$项就是
\bea
\frac{1}{r} &\simeq \frac{1}{r}\big|_{\vec{x}'=0} + \pfrac{r}{x_i'}\big|_{\vec{x}'=0} + \frac{1}{2} \frac{\partial^2}{\partial x_i' \partial x_j'} \left( \frac{1}{r}\right)\big|_{\vec{x}'=0} x_i' x_j' \\
&= \frac{1}{r} + \frac{x_i x_i'}{r^3} + \frac{1}{2} \frac{3x_i x_j - \delta_{ij} r^2}{r^5}x_i' x_j'
\eea
其中重新引入了记号$r = \sqrt{|\vec{x}'|}$。在远处的电势
\bea
\phi(\vec{x}) &=   \frac{1}{4\pi \epv} \int d^3 \vec{x}' \frac{\rho(\vec{x}')}{\left|\vec{x} - \vec{x}'\right|}\\ &\simeq \frac{1}{4\pi \epv} \int d^3 \vec{x}' \left(\frac{1}{r} + \frac{x_i x_i'}{r^3} +  \frac{1}{2}\frac{3x_i x_j - \delta_{ij} r^2}{r^5}x_i' x_j'\right) \rho(\vec{x}') \\
&= \frac{1}{4\pi \epv} \int d^3 \vec{x}' \left(\frac{1}{r} + \frac{x_i x_i'}{r^3} + \frac{1}{2} \frac{3x_i' x_j' - \delta_{ij} r'^2 }{r^5}x_i x_j\right) \rho(\vec{x}')
\eea
最后一个等号将$\delta_{ij}$重新分配到了$\sim \vec{x}'$项上,很容易验证它们之间是相等的。定义
\beq
Q =  \int d^3 \vec{x}' \rho(\vec{x}') 
\eeq
\beq
p_i = \int d^3 \vec{x}' \rho(\vec{x}') x_i'
\eeq
\beq
D_{ij} =\int d^3 \vec{x}' \rho(\vec{x}') (3x_i' x_j' - \delta_{ij} r'^2 )
\eeq
分别为零极、偶极和四极矩,就有
\beq
\phi(\vec{x}') \simeq \frac{1}{4\pi \epv} \left( \frac{Q}{r} + \frac{p_i x_i}{r^2} +\frac{1}{2}  \frac{D_{ij} x_i x_j}{r^3}\right)
\eeq
四级矩$D_{ij}$显然是一个对称无迹张量,一个对称无迹张量的独立分量数为5($SO(3)$群的分解$9 = 5\oplus 3 \oplus 1$)。各独立分量为
\beq
D_{xx} = \int (2x^2 - y^2 - z^2 ) \rho(\vec{x}) d^3 \vec{x} 
\eeq
\beq
D_{yy} = \int (2y^2 - x^2 - z^2 ) \rho(\vec{x}) d^3 \vec{x} 
\eeq
\beq
D_{xy} = \int3xy \rho(\vec{x}) d^3 \vec{x} 
\eeq
\beq
D_{yz} = \int  3yz \rho(\vec{x})d^3 \vec{x}
\eeq
\beq
D_{zx} = \int  3zx \rho(\vec{x})d^3 \vec{x}
\eeq
其他分量$D_{zz} = - D_{xx} - D_{yy};\ D_{yx} = D_{xy}, \ D_{zy} = D_{yz}, \ D_{xz} = D_{zx}$。
\begin{eg}
估算一个均匀带电的半径为$a$的圆盘在远处产生的电势和这个圆盘的电容。
\end{eg}
\begin{ans}
设圆盘上的面电荷密度为$\sigma$,空间电荷分布$\rho(\vec x) = \sigma \delta(z)$。零极矩就是总电荷$Q =\sigma \pi a^2$,偶极矩显然可以根据对称性判断为$0$。四级矩
\beq
D_{xx} = \int ( 2x^2 - y^2 ) \sigma dxdy = \int_{\theta=0}^{2\pi} d\theta \int_{r=0}^a (2r^2 \cos^2 \theta - r^2 \sin^2 \theta) rdr = \frac{\sigma \pi a^4}{4}
\eeq
\beq
D_{yy} =  \int ( 2y^2 - x^2 ) \sigma dxdy = D_{xx} = \frac{\sigma \pi a^4}{4}
\eeq
\beq
D_{zz} = -D_{xx} - D_{yy} = - \frac{\sigma \pi a^4}{2}
\eeq
非对角项$D_{xy}$根据对称性判断为$0$;$D_{zy}$和$D_{zx}$由于$\rho(\vec x) \propto \delta(z)$的限制只能为$0$。于是远场电势
\beq
\phi(\vec x) \simeq \frac{1}{4\pi \epv} \left( \frac{Q}{r} + \frac{\sigma \pi a^4}{8r^5} (x^2 + y^2 - 2z^2)\right)
\eeq
或者用球坐标表示为
\beq
\phi(r,\theta) \simeq \frac{Q}{4\pi \epv} \left( \frac{1}{r} + \frac{a^2}{8r^3} (\sin^2 \theta - 2\cos^2 \theta)\right)
\eeq
电容可以这样估计:圆盘边缘的电势为
\beq
\phi(r=a,\theta = \frac{\pi}{2}) \sim \frac{Q}{4\pi \epv} \frac{9}{8a}
\eeq
电容为
\beq
C = \frac{Q}{\phi(r=a,\theta = \frac{\pi}{2})} \sim \frac{32a\pi \epv}{9}
\eeq
可以想象,因为多极展开只适用于$r\gg a$的情况,因此上式对电容的估计是十分粗糙的。
\end{ans}
\par
虽然磁场和电场很大程度上是对偶的,上面的讨论一般都可以直接套用到磁场上。但需要注意的是,对于磁场,由于不存在磁荷,磁矢势的多极展开是从偶极项开始的,磁矩$\vec m$产生的磁矢势为
\beq
\vec A = \frac{\mu_0}{4\pi} \frac{\vec m \times \vec x}{x^3}
\eeq










\section{介质}
\subsection{电介质的基本图像}
在外场下,一个实际的物质会发生极化,产生出偶极矩。
\beq
\vec{P} = \vec P ( \vec E,\vec B)
\eeq
偶极子势
\beq
\phi(\vec r) = \frac{1}{4\pi \epv} \frac{\vec p \cdot \vec r}{r^3}
\eeq
给定电荷分布$\rho(\vec x)$和偶极分布$\vec P (\vec x)$,空间中的电势
\beq
\phi(\vec x) =  \frac{1}{4\pi \epv}  \int \left(\frac{\rho(\vec{x}')}{|\vec x - \vec{x}'|} + \frac{\vec P(\vec{x}') \cdot (\vec{x} - \vec{x}')}{|\vec x - \vec{x}'|^3} \right) d^3 \vec{x}'
\eeq
注意到
\beq
\frac{\vec x  - \vec{x}'}{|\vec x - \vec{x}'|^3} = \del' \left( \frac{1}{|\vec x - \vec{x}'|}\right)
\eeq
于是
\bea
\int d^3 \vec{x}' P(\vec{x}') \frac{\vec x  - \vec{x}'}{|\vec x - \vec{x}'|^3} &= \int d^3 \vec{x}' P(\vec{x}')\cdot \del' \left( \frac{1}{|\vec x - \vec{x}'|}\right) \\
&= \int d^3 \vec{x}' \left[ \del' \cdot \left( \frac{1}{|\vec x - \vec{x}'|P(\vec{x}')}\right)  - \frac{1}{|\vec x - \vec{x}'|} \del' P(\vec{x}')\right]\\
&=\oint d\vec S \frac{1}{|\vec x - \vec{x}'|}P(\vec{x}')- \int d^3 \vec{x}' \frac{1}{|\vec x - \vec{x}'|} \del' P(\vec{x}')
\eea
也就是说,极化矢量$\vec P$的作用等价于一个分布在介质的面电荷密度$\sigma_b = \vec P \cdot \vec n$以及分布在介质当中的体电荷密度$\rho_b = -\del \cdot \vec P$。这从图像上来讲是清楚的:比如一块均匀极化的长方体介质,其内部的正负电荷略微分离,把介质分成一层一层的离散小片,某一层向下一层分量出的正电荷总能和下一层剩余的负电荷抵消,从而在介质内部无净极化电荷;唯独表明层向外贡献的正电荷不能再被抵消,存在表面电荷。
\par
在外场不强的情况下,一般假设极化是均匀且线性的
\beq
\vec P = (\epsilon - \epv) \vec E
\eeq
这时,所有静电场规律只需要将原来的$\epv$换成$\epsilon$。
\subsection{磁介质的基本图像}
磁介质的基本图像和电介质稍有差别。磁矩$\vec m$产生的矢势近似为
\beq
\vec A = \frac{\mu_0}{4\pi} \frac{\vec m \times \vec x}{x^3}
\eeq
于是对于一个磁矩分布$\vec M$,有
\beq
\vec A (\vec x) = \int d^3 \vec{x}' \frac{\mu_0}{4\pi} \frac{\vec M(\vec{x}') \times ( \vec x - \vec{x}')}{|\vec x - \vec{x}'|^3}
\eeq
我们按同样的套路整理右边项:为了清楚起见,把叉乘写成指标的形式
\beq
A_i (\vec x) =\epsilon_{ijk}  \int d^3 \vec{x}' \frac{\mu_0}{4\pi} \frac{M_j (\vec{x}') ( \vec x - \vec{x}')_k}{|\vec x - \vec{x}'|^3}
\eeq
同样利用
\beq
\frac{\vec x  - \vec{x}'}{|\vec x - \vec{x}'|^3} = \del' \left( \frac{1}{|\vec x - \vec{x}'|}\right)
\eeq
有
\bea
A_i (\vec x) &=\epsilon_{ijk}  \int d^3 \vec{x}' \frac{\mu_0}{4\pi} M_j (\vec{x}') \partial'_k\left( \frac{1}{|\vec x - \vec{x}'|}\right) \\
&= \frac{\mu_0}{4\pi}  \epsilon_{ijk}  \int d^3 \vec{x}'  \partial'_k \left(M_j (\vec{x}')\frac{1}{|\vec x - \vec{x}'|}\right)- \frac{\mu_0}{4\pi}  \epsilon_{ijk}  \int d^3 \vec{x}' \left( \partial'_k M_j (\vec{x}')\right)\frac{1}{|\vec x - \vec{x}'|}\\
&=  \frac{\mu_0}{4\pi}  \epsilon_{ijk}  \oint dS_k'  M_j (\vec{x}')\frac{1}{|\vec x - \vec{x}'|}+ \frac{\mu_0}{4\pi}  \epsilon_{ijk}  \int d^3 \vec{x}' \left( \partial'_j M_k (\vec{x}')\right)\frac{1}{|\vec x - \vec{x}'|}\\
&=  \frac{\mu_0}{4\pi} \oint dS'  \epsilon_{ijk}  M_j n_k (\vec{x}')\frac{1}{|\vec x - \vec{x}'|}+ \frac{\mu_0}{4\pi}  \int d^3 \vec{x}' \left( \epsilon_{ijk}  \partial'_j M_k (\vec{x}')\right)\frac{1}{|\vec x - \vec{x}'|}\\
 \eea
 再把指标语言换成矢量语言就清楚了
 \beq
 \vec A (\vec x) = \frac{\mu_0}{4\pi} \oint dS' ( \vec M \times \vec n )\frac{1}{|\vec x - \vec{x}'|} + \frac{\mu_0}{4\pi} \int d^3 \vec{x}' (\del' \times \vec M) \frac{1}{|\vec x - \vec{x}'|}
 \eeq
 立即看出,等效的面电流密度和体电流密度分别为
 \beq
 \vec K_{\vec M} = \vec M \times \vec n
 \eeq
 \beq
 \vec{j}_{\vec M} = \del \times \vec M
 \eeq
 面电流密度这项的图像特别清楚:由于$\vec M$可以看出介质中许多小电流圈的分别,在均匀极化的情况下,介质内部的这些电流圈在相切处的电流方向相反,全部抵消;仅仅在表面产生如\cref{mag_pol}的面电流。
 \cpicn{0.2}{mag_pol}{磁极化的表面效应}
 \par
 在外场不强的情况下,一般假设磁化是均匀且线性的
 \beq
 \vec M = \left( \frac{1}{\mu_0} - \frac{1}{\mu} \right) \vec B
 \eeq
 这时,所有的静磁场规律只要将原来的$\mu_0$换成$\mu$。
 \subsection{欧姆定律}
 上面都考虑的是理想的介质:在外场下,电子会稍微偏离原子核,但仍然被束缚在原所在位置(原子核)附近(对应量子力学的Stark效应)。而对于一些不太理想的介质,电子可能被剥离原子而在介质中自由运动。在外场不强的情况下,我们可以假设这部分电子产生的电流正比于外场
 \beq
 \vec j = \sigma \vec E
 \eeq
 \subsection{介质中的边值问题}
 我们先考虑介质中的麦克斯韦方程。出发点依然是真空中的麦克斯韦方程。注意到,麦克斯韦方程右边的源项是空间中所有的电荷(电流),也即是自由电荷(人为添加的电荷)和束缚电荷(介质对外场的相应)之和。
 \bea
 \rho_t &= \rho_f + \rho_b  = \rho_f - \del \cdot \vec P\\
\vec{j}_t &= \vec{j}_f + \vec{j}_b = \vec{j}_f + \del \times \vec{M}
\eea
此时麦克斯韦方程组中的含源方程变为
\bea
\del \cdot \vec E &= \frac{1}{\epv}(\rho_f - \del \cdot \vec P) \\
\del \times \vec B &= \mu_0 ( \vec{j}_f + \del \times \vec{M})
\eea
移项就得到
\bea
\del \cdot \left(\vec P  + \epv \vec E \right) &= \rho_f \\
\del \times \left( \frac{1}{\mu_0} \vec B- \vec M\right) &= \vec{j}_f
\eea
定义电位移矢量$\vec D$和磁场强度$\vec H$
\bea
\vec D &= \epv \vec E + \vec P \\
\vec H &= \frac{1}{\mu_0} \vec B - \vec M
\eea
在线性极化的情况下
\bea
\vec D &= \epsilon \vec E\\
\vec H &= \frac{1}{\mu} \vec B
\eea
此时麦克斯韦方程组化为
\bea
\del \cdot \vec D &= \rho_f \\
\del \times \vec E &= - \frac{\partial B}{\partial t} \\
\del \cdot \vec B &= 0 \\
\del \times \vec H &= \vec{j}_f
\eea




\section{动场}
\subsection{自由无界空间中的场}
如果空间中没有自由电荷,且介质是线性的,把麦克斯韦方程按
\bea
\partial_0 \equiv \frac{\partial}{\partial_t} &\to i\omega \\
\del &\to -i\vec k
\eea
傅立叶变换,一组用于确定自由无界空间中的场的方程就是
\bea
\vec k \cdot \vec D &= 0 \\
\vec k \times \vec E &= \omega \vec B \\
\vec k \cdot \vec B &= 0 \\
\vec k \times \vec H &= -\omega \vec D + \vec{j} \\
D_i &= \epsilon_{ij} E_j \\
B_i &= \mu_{ij} H_j \\
j_i &= \sigma_{ij} E_j
\eea
上面一共有17个方程(5个矢量方程和2个标量方程),却只有15个未知量,因此这组方程事实上是过定的。另外我们发现,因为不存在源,这组方程是齐次的。对过定的线性齐次方程,有非零解(不是所有场都等于0)的条件就是方程非满秩。
\par
换句话说,我们不能从上面的方程解出场本身,但可以解出系数之间的约束关系。我们现在来着手做这件事:把后三个方程代入前4个,全部采用指标语言\footnote{$\epsilon_{ij}$代表介电常数张量,而$\epsilon_{ijk}$代表全反称符号。根据指标个数判断符号意义应该不至于产生混淆。},有
\bea
\epsilon_{ij}k_i E_j &= 0 \\
\epsilon_{ijk}k_j E_k &= \omega \mu_{ij} H_j \\
\mu_{ij} k_i H_j &=0\\
\epsilon_{ijk} k_j H_k &=-\omega \epsilon_{ij} E_j + \sigma_{ij} E_j
\eea
第2和第4两式做内积得到
\beq
\epsilon_{ijk}\epsilon_{ilm} k_j k_l E_k H_m = \omega \mu_{im}(-\omega \epsilon_{ik} + \sigma_{ik}) H_m E_k
\eeq
抽掉$E_k H_m$得到
\beq
(\delta_{jl} \delta_{km} - \delta_{jm} \delta_{kl}) k_j k_l  = \omega \mu_{im}(-\omega \epsilon_{ik} + \sigma_{ik}) 
\eeq
\beq
\vec{k}^2 \delta_{km} - k_m k_k = -\omega^2 \epsilon_{ik} \mu_{im} + \omega \sigma_{ik}	\mu_{im}
\eeq



















\end{document}
