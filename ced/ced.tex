\documentclass[a4paper,11pt]{ctexart}

\usepackage{amsmath}
\usepackage{color}
\usepackage{mathrsfs}
\usepackage[colorlinks,
            linkcolor=blue,
		 urlcolor=black]{hyperref}
\usepackage{graphicx}
\usepackage{cleveref}
\usepackage{float}

\crefname{equation}{}{}
\crefname{figure}{图}{图}
\crefname{footnote}{注释}{注释}

%\cpic{<尺寸>}{<文件名>}}用于生成居中的图片。
\newcommand{\cpic}[2]{
\begin{center}
\includegraphics[scale=#1]{#2}
\end{center}
}

%\cpicn{<尺寸>}{<文件名>}{<注释>}用于生成居中且带有注释的图片,其label为图片名。
\newcommand{\cpicn}[3]
{
\begin{figure}[H]
\cpic{#1}{#2}
\caption{#3\label{#2}}
\end{figure}
}

\newtheorem{eg}{例}[section]
\newtheorem{ans}{解答}[section]

\newcommand{\beq}{\begin{equation}}
\newcommand{\eeq}{\end{equation}}
\newcommand{\bea}{\begin{equation}\begin{aligned}}
\newcommand{\eea}{\end{aligned}\end{equation}}
\newcommand{\red}{\color{red}}
\newcommand{\lag}{\mathcal{L}}
\newcommand{\diag}{\mathrm{diag}}
\newcommand{\del}{\vec{\nabla}}
\newcommand{\epv}{\epsilon_0}
\newcommand{\pfrac}[2]{\frac{\partial #1}{\partial #2}}
\newcommand{\emptyline}{\vspace{0.7cm} \\}
\newcommand{\dist}{\left| \vec{x} - \vec{x}' \right|}
\newcommand{\sgn}{\mathrm{sgn}}
\newcommand{\res}{\mathrm{res\ }}


\title{Classical Electrodynamic}
\author{Probfia}
\date{\today}


\begin{document}
\maketitle
\tableofcontents



\section{无边界的经典电磁场}
\subsection{麦克斯韦方程}
真空中与源耦合的麦克斯韦方程具有如下形式
\bea \label{maxeq}
\del \cdot \vec{E} &= \frac{\rho}{\epv} \\
\del \times \vec{E} &= -\pfrac{\vec{B}}{t} \\
\del \cdot \vec{B} &= 0 \\ 
\del \times \vec{B} &= \pfrac{\vec{E}}{t}+ \mu_0 \vec{j}
\eea
为了简化上面的方程,引入势函数$\phi$和$\vec{A}$使得
\bea
\vec{E} &= -\del \phi - \pfrac{\vec{A}}{t} \\
\vec{B} & = \del \times \vec{A}
\eea
这使得麦克斯韦方程中的$\del \times \vec{E}$和$\del \cdot \vec{B}$式自动满足。势函数的选取依然存在不定性,可以验证,若$(\phi,\vec A)$给出的物理场是$( \vec E , \vec B )$,对任意的标量场$\lambda (\vec{x},t)$,$(\phi -\partial_t \lambda, \vec A - \del \lambda )$。这种不变性称为势函数的规范不变性。规范给出对势函数的一个具体限制,我们常用洛伦兹规范$\partial_t \phi + \del \cdot \vec A = 0$。
\par
把势函数和规范要求带入麦克斯韦方程的剩下两式就得到
\bea \label{maxpot}
\frac{\partial^2 \phi}{\partial t^2} - \nabla^2 \phi &= \frac{\rho }{\epv} \\
\frac{\partial^2 \vec A}{\partial t^2} - \nabla^2 \vec A &= \mu_0 \vec j 
\eea
推导第二式需要将原等式右边的$\vec E$用势函数表示,并用洛伦兹规范条件与左边的$\del \cdot ( \del \cdot \vec A)$抵消。
\par
当源$(\rho, \vec j )$不存在时,势函数$\phi$和$\vec A$的各个分量满足自由空间中的波动方程。
\subsection{格林函数}
当空间中存在源$(\rho,\vec{j})$时,容易看出$\phi \sim \rho$,$A_i \sim j_i$。方程形式的类似性告诉我们为了具体讨论源对势函数的影响,只用讨论$\phi$对$\rho$的影响。
\par
由于电磁场的叠加原理,我们只需要考虑点源的影响。考虑方程
\beq
\frac{\partial^2 \phi}{\partial t^2} - \nabla^2 \phi = \delta(t-t_0,\vec{x}-\vec{x}_0)
\eeq
两边乘$e^{-i\omega t + i \vec{k} \cdot \vec{x}} dt d^3 \vec{x}$做积分,得到傅立叶空间中的方程为
\beq
(\vec{k}^2 - \omega^2 ) \phi = e^{-i\omega t_0 + i \vec{k} \cdot \vec{x}_0}
\eeq
傅立叶逆变换为
\beq
\phi = \int \frac{d\omega}{2\pi} \int \frac{d^3 \vec{k}}{(2\pi)^3} \frac{1}{\vec{k}^2-\omega^2} e^{-i\omega t_0 + i \vec{k} \cdot \vec{x}_0} e^{i\omega t - i \vec{k} \cdot \vec{x}}
\eeq
这个积分可以这样进行,首先,由于傅立叶空间中$\vec{x}$是一个给定的量,记$\vec{x}-\vec{x}_0$的长度为$r$,$\vec{x}-\vec{x}_0$与$\vec{k}$的夹角为$\theta$(随$\vec{k}$变化而变化,从而被对$d^3 \vec{k}$的积分积掉。第二个积分因此可以改到在球坐标下进行。球坐标下的体积元$d^3 \vec{k} = k^2 \sin \theta dk d\theta d\phi$,因此有
\beq
\phi = \int \frac{d\omega}{2\pi} e^{i\omega ( t - t_0)} \int_{\phi = 0}^{2\pi} d\phi\int \frac{k^2 dk}{(2\pi)^3} \frac{1}{k^2 - \omega^2} \int_{\theta = 0}^\pi e^{ikr\cos \theta } \sin \theta d\theta 
\eeq
最后两个个角向积分很容易进行
\bea
\int_{\theta = 0}^\pi e^{ikr \cos \theta} \sin \theta d\theta &= -\int_{\theta = 0}^{\pi} e^{ikr \cos \theta} d(\cos \theta) \\
&= \int_{x = -1}^1 e^{ikrx}dx = \left. \frac{e^{ikrx}}{ikr} \right|_{-1}^1 = 2\frac{\sin kr}{kr}
\eea
现在可以进行对$k$的积分,首先为了利用复变函数积分技巧,有
\beq
\frac{2}{r} \int \frac{k dk}{(2\pi)^3} \frac{1}{k^2 - \omega^2}\sin kr = \frac{2}{r} \mathrm{Im} \int \frac{k dk}{(2\pi)^3} \frac{1}{k^2 - \omega^2} e^{ikr} 
\eeq
由于$r>0$,在上半复平面积分可以让$e^{ikr}$在上半围道消失,但极点$k = \pm \omega$出现在围道上,为了解决这个困境,考虑
\beq
\int \frac{k dk}{(2\pi)^3} \frac{1}{k^2 - \omega^2} e^{ikr}  = \lim_{\epsilon \to 0} \int \frac{k dk}{(2\pi)^3} \frac{1}{k^2 - \omega^2 + i \epsilon} e^{ikr} 
\eeq
这时被积函数的极点就出现在$k = \pm \sqrt{\omega^2 - i\epsilon} \simeq \pm \omega \mp i\frac{\epsilon}{2\omega}$,极点$-\omega + i \frac{\epsilon}{2\omega} $出现在上半复平面,对积分有贡献。由于
\beq
\frac{k}{k^2 - \omega^2} e^{ikr} = \frac{1}{k-(-\omega)} \frac{k}{k-\omega}e^{ikr}
\eeq
该点的留数为$\frac{1}{2(2\pi)^3} e^{-i\omega r}$,积分的最终结果因此为$\frac{1}{(2\pi)^2 r} \cos \omega r$。\par
最后进行对$\omega$的积分,有
\bea
\phi &= 2\pi \int_{\omega = -\infty}^\infty \frac{d\omega}{2\pi} e^{i\omega (t-t_0) }\frac{ \cos \omega r }{(2\pi)^2r} \\
& = \frac{1}{(2\pi)^2 r}  \int_{\omega = -\infty}^\infty d\omega \cos \omega(t-t_0) \cos \omega r \\
&= \frac{1}{2(2\pi)^2 r} \int_{\omega = -\infty}^\infty d\omega \left[ \cos \omega(t-t_0 -r) + \cos \omega(t-t_0 + r)\right] \\
&= \frac{1}{4 \pi r} \left[\delta (t-t_0-r) + \delta (t-t_0+r) \right]
\eea
这就是势函数对电源的响应,也即源对势的格林函数。
\subsection{推迟势}
根据格林函数的公式,对一般的源$\epv \rho(\vec{x},t)$有
\bea
\phi(\vec{x},t) &=\frac{1}{4\pi \epv}  \int dt'd^3 \vec{x}' \frac{1}{|\vec{x}-\vec{x}'|} \rho(\vec{x}',t') \left[\delta (t-t_0-r) + \delta (t-t_0+r) \right]\\
&= \frac{1}{4\pi \epv} \int d^3 \vec{x}' \frac{1}{|\vec{x}-\vec{x}'|} \left[\rho(\vec{x}',t-|\vec{x}-\vec{x}'|) + \rho(\vec{x},t+|\vec{x}'-\vec{x}'|) \right]
\eea
这里发现,势函数$\phi$可以分解为推迟势
\beq
\phi_{ret} (\vec{x},t) = \frac{1}{4\pi \epv} \int d^3 \vec{x}' \frac{1}{|\vec{x}-\vec{x}'|} \rho(\vec{x}',t-|\vec{x}-\vec{x}'|) 
\eeq
和超前势
\beq
\phi_{adv} (\vec{x},t) = \frac{1}{4\pi \epv} \int d^3 \vec{x}' \frac{1}{|\vec{x}-\vec{x}'|} \rho(\vec{x}',t+|\vec{x}-\vec{x}'| )
\eeq
可以验证,推迟势和超前势都各自满足有源的势函数波动方程,但超前势却面临严重的因果律问题(未来时间对现在世界的影响),因此,我们以后将只考虑推迟势。势对源的响应因此概括为
\bea \label{retp}
\phi(\vec x,t) &= \frac{1}{4\pi \epv} \int d^3 \vec{x}' \frac{1}{|\vec{x}-\vec{x}'|} \rho(\vec{x}',t-|\vec{x}-\vec{x}'|)  \\
\vec A (\vec x,t) &= \frac{\mu_0 }{4\pi } \int d^3 \vec{x}' \frac{1}{|\vec{x}-\vec{x}'|} \vec j (\vec{x}',t-|\vec{x}-\vec{x}'|) 
\eea
如果要放回所有的光速$c$,则是
\bea
\phi(\vec x,t) &= \frac{1}{4\pi \epv} \int d^3 \vec{x}' \frac{1}{|\vec{x}-\vec{x}'|} \rho(\vec{x}',t-\frac{1}{c}|\vec{x}-\vec{x}'|)  \\
\vec A (\vec x,t) &= \frac{\mu_0 }{4\pi } \int d^3 \vec{x}' \frac{1}{|\vec{x}-\vec{x}'|} \vec j (\vec{x}',t-\frac{1}{c}|\vec{x}-\vec{x}'|) 
\eea
推迟势的物理意义在于否定了超距作用:一个位于$\vec{x}'$点的电子必须经过$\sim \frac{1}{c} \left| \vec{x}' - \vec{x} \right|$的时间后才能感受到位于$\vec{x}$处的电子发生的变化,信息以光速传递。此外可以看出,$\phi \sim \frac{1}{r}$,即势在无穷远处消散。这都是符合我们的物理直觉的。
\subsection{规范场的作用量}

\section{静场}
\subsection{拉普拉斯方程}
假设电荷分布和电流分布都不显含时间$t$,则从\cref{retp}可以看出,势函数同样不显含时间。这样不随时间演化的场称为静场。此时\cref{maxpot}简化为
\bea
\nabla^2 \phi &= - \frac{\rho}{\epv}\\
\nabla^2 \vec A &= - \mu_0 \vec j
\eea
即势函数的各个分量都分别满足柏松方程。同样的,由于它们相仿的形式,我们暂且将目光局限在决定$\phi$的方程上。在无界空间中,$\phi$的解可以直接套用推迟势公式\cref{retp},此时$\rho$无时间依赖
\beq
\phi(\vec{x}) = \frac{1}{4\pi \epv} \int d^3 \vec{x}' \frac{\rho(\vec{x}')}{\left|\vec{x} - \vec{x}'\right|}
\eeq
但实际问题中,空间中存在着各种介质和导体,因此,电势$\phi$需要满足一些有限空间的边界条件。
\par
边界条件分为两类,第一类边界条件给出边界上$\phi$取值的直接限制
\beq
\phi \big|_{S} = f(\vec{x})
\eeq
第二类边界条件给出$\phi$梯度的法向分量,即电场垂直于边界面分量的限制
\beq
\del \phi \cdot \vec{n} \big|_S= f(\vec{x})
\eeq
静场问题归结于在边界条件下解泊松方程的问题。好的一点在于,若给定电荷分布$\rho(\vec{x})$,我们总能做变换$\psi = \phi - \frac{1}{4\pi \epv} \int d^3 \vec{x}' \frac{\rho(\vec{x}')}{\left|\vec{x} - \vec{x}'\right|}$。此时,$\psi$满足拉普拉斯方程
\beq
\nabla^2 \psi = 0
\eeq
而$\psi$的第一类(第二类)边界条件相当于原来$\phi$的边界条件减去对于自由解在边界上(的法向分量)的取值。也就是说,我们总能靠减去自由解的方法把泊松方程问题转化为拉普拉斯方程问题。

\subsection{分离变量解}
\subsubsection{直角区域分离变量}
在直角区域内,一个函数可以按正余弦函数展开。对于有界空间,电势表达为一系列傅立叶级数形式;而对无界空间,电势表达为傅立叶积分。我们分别举一个有界和无界的例子表明直角区域静电问题的解法。\begin{eg}
在正方形区域$0<x,y<L$中,除顶边$0<x<L, y=L$的电势为$V(x)$外,其余三边接地。求该正方形区域内的电势分布。
\end{eg}
\begin{ans}
设电势$\phi( x,y) = X(x) Y(y)$,带入拉普拉斯方程有
\beq
\frac{1}{X} \frac{d^2 X}{d x^2} = - \frac{1}{Y} \frac{d^2 Y}{dy^2} = -k^2
\eeq
右边取$-k^2$可以让$X(x)$取震荡形式从而满足两边的零边界条件,从而有分离变量解
\begin{equation}
X_n = a_n \sin \frac{n\pi x}{L}, n=0,1,2,\cdots
\end{equation}

\begin{equation}
Y_n = b_n \sinh \frac{n\pi y}{L}, n=0,1,2,\cdots
\end{equation}
于是写出
\begin{equation}
\phi(x,y) = \sum_{n=1}^\infty c_n \sin \frac{n\pi x}{L} \sinh \frac{n\pi y}{L}
\end{equation}
将上式代入边界条件$u(x,L) =V(x)$,得到
\begin{equation}
\sum_{n=1}^\infty c_n \sin \frac{n\pi x}{L} \sinh n\pi = V(x)
\end{equation}
上式两边同乘$\sin \cfrac{m\pi x}{L}$并对$x$从0到$L$积分(利用函数集$\{ \sin \cfrac{n \pi x}{L}\}$的正交性,相当于求$V$的傅里叶正弦展开系数),得到
\beq
\frac{L}{2}  c_n \sinh n \pi=  \int_{x=0}^L V(x) \sin \frac{n \pi x}{L} dx
\eeq
于是
\beq
\phi(x,y) = \frac{2}{L} \sum_{n=1}^\infty \frac{1}{\sinh n \pi}  \left( \int_{x=0}^L V(x) \sin \frac{n \pi x}{L} dx \right)
\sin \frac{n\pi x}{L} \sinh \frac{n\pi y }{L}
\eeq
当$V(x)=V$为一常数时
\begin{equation}
c_n = \frac{2V[1-(-1)^n]}{n\pi \sinh n \pi} \end{equation}
故正方形区域内的电势分布为
\begin{equation}
\phi(x,y) = 4V\sum_{n=1}^{\infty} \frac{1}{n\pi \sinh n \pi}  \sin \frac{n\pi x}{L} \sinh \frac{n\pi y}{L}
\end{equation}
取$n=1000$的截断作出的图像如\cref{es2}。
\cpicn{0.4}{es2}{本例题中方形区域内的电势分布}
\end{ans}
\par
下面一个例子说明无界空间中的分离变量。
\begin{eg}
在三维空间中的$xy$平面上有一个无限大平板,其上的电势给定为$V(x,y)$。求全空间的电势分布和板两侧的电荷分布。
\end{eg}
\begin{ans}
根据对称性,先分离$z$变量。设$\phi(x,y,z) = \psi(x,y) Z(z)$,有
\beq
\frac{1}{\psi} \nabla^2 \psi  = - \frac{1}{Z} \frac{d^2 Z}{dz^2} =- k^2
\eeq
右边选择$-k^2$是因为$Z(z)$必然在足够远处按指数形式$\sim e^{-k|z|}$消散而不能持续震荡。因此
\beq
\frac{d^2 Z}{dz^2} - k^2 Z = 0
\eeq
根据$z$的正负选取合适的解
\beq
Z(z) \sim \begin{cases} e^{-kz} &,z>0; \\ e^{kz} &,z<0 \end{cases}
\eeq
进一步分离变量,设$\psi(x,y) = X(x)Y(y)$,有
\bea
\frac{d^2 X}{dx^2} + k_1^2 X &= 0 \\
\frac{d^2 Y}{dy^2} + k_2^2 Y &= 0
\eea
其中$k_1^2 + k_2^2 = k^2$。$xy$平面的无界性给出模式$(k_1,k_2)$的连续谱,$X\sim e^{ik_1 x}$,$Y \sim e^{ik_2 y}$。解形式上为各个模式的线性组合
\beq
\phi(x,y,z) = \int \frac{dk_1 dk_2}{(2\pi)^2} c(k_1,k_2) e^{i(k_1 x + k_2 y)} e^{-\sqrt{k_1^2 + k_2^2} |z|}
\eeq
为了确定系数$c(k_1,k_2)$,令$z=0$有
\beq
V(x,y) = \int \frac{dk_1 dk_2}{(2\pi)^2} c(k_1,k_2) e^{i(k_1 x + k_2 y)}
\eeq
可以看出$V(x,y)$就是$c(k_1,k_2)$的傅立叶逆变换,于是
\beq
c(k_1,k_2) = \int dx dy V(x,y) e^{-i(k_1 x + k_2 y)}
\eeq
由电磁学的高斯定理知道表面面电荷分布$\sigma = \epv \vec{E} \cdot \vec{n}$,于是需要知道电场的$z$分量
\beq
E_z = - \pfrac{\phi}{z} =\sgn(z) \int \frac{dk_1 dk_2}{(2\pi)^2}\frac{c(k_1,k_2)}{\sqrt{k_1^2 + k_2^2}} e^{i(k_1 x + k_2 y)} e^{-\sqrt{k_1^2 + k_2^2} |z|}
\eeq
符号函数的出现是因为对$z$的绝对值求导。于是面电荷分布
\bea
\sigma_\pm &= \left. \epv\sgn(z) \int \frac{dk_1 dk_2}{(2\pi)^2}\frac{c(k_1,k_2)}{\sqrt{k_1^2 + k_2^2}} e^{i(k_1 x + k_2 y)} e^{-\sqrt{k_1^2 + k_2^2} |z|} \vec{e}_z \cdot (\pm \vec{e}_z) \right|_{z=0} \\ &= \epv \int \frac{dk_1 dk_2}{(2\pi)^2}\frac{c(k_1,k_2)}{\sqrt{k_1^2 + k_2^2}} e^{i(k_1 x + k_2 y)}
\eea
正负号的出现是因为上下表面的法向量方向不同,这个正负号于$\sgn(z)$抵消。我们发现上下表面带等量同种电荷,这其实从对称性可以很容易看出。\par
下面去特例对结果进行讨论。当$V(x,y) = V$为一常数时,有
\beq
c(k_1,k_2) = \int dx dy V e^{-i(k_1 x + k_2 y)} = (2\pi)^2 \delta(k_1) \delta(k_2)
\eeq
于是
\bea
\phi(x,y,z) &= \int \frac{dk_1 dk_2}{(2\pi)^2} (2\pi)^2 V \delta(k_1) \delta(k_2) e^{i(k_1 x + k_2 y)} e^{-\sqrt{k_1^2 + k_2^2} |z|} \\ 
&= V
\eea
即全空间的电势为一个常数,这乍看不符合直觉,但显然$\phi = V$是$\nabla^2 \phi = 0$符合给定边界条件的一个解,而且唯一性定理保证了它就是唯一的解。\par
再例如$V(x,y) = \frac{1}{1+x^2}$。有
\beq
c(k_1,k_2) = \int dx dy \frac{1}{1+x^2} e^{-i(k_1 x + k_2 y)} = 2\pi \delta(k_2) \int dx \frac{1}{1+x^2} e^{-ik_1 x}
\eeq
积分可以用复变函数的方法进行,注意到被积函数$f(z) = \frac{1}{1+z^2} e^{-ik_1 z}$下有两个极点$\res f(i) = \frac{e^{-k_1}}{2i}$,$\res f(-i) = - \frac{e^{k_1}}{2i}$。当$k_1 > 0$,积分在上半围道进行,结果为$\pi e^{-k_1}$;当$k_1 < 0 $,积分在下半围道进行,这时在实轴上$x$从正无穷走向负无穷,因此需要在$2\pi i \res f(-i)$上再添一个符号才是原积分的结果,为$\pi e^{k_1}$。两种情况积分可以统一表示成$\pi e^{-|k_1|}$。故
\beq
c(k_1,k_2) = 2\pi^2 \delta(k_2) e^{-|k_1|}
\eeq
于是全空间电势分布
\bea
\phi(x,y,z) &= \int \frac{dk_1 dk_2}{(2\pi)^2} 2\pi^2 \delta(k_2) e^{-|k_1|} e^{i(k_1 x + k_2 y)} e^{-\sqrt{k_1^2 + k_2^2} |z|} \\ 
&= \int_{k_1 = -\infty}^\infty dk_1 \frac{1}{2} e^{-|k_1|} e^{ik_1 x} e^{-|k_1| |z|} \\
&= \mathrm{Re}\ \int_{k_1 = 0}^\infty dk_1 e^{-k_1(1+|z|-ix)}  \\ 
&= \mathrm{Re}\ \frac{1}{1+z-ix} = \frac{1+|z|}{(1+|z|)^2 + x^2}
\eea
电场分布
\beq
\vec{E} = -\pfrac{\phi}{x}\vec{e}_x - \pfrac{\phi}{z} \vec{e}_z  = \sgn z\left ( \frac{2 x (z+1)}{\left(x^2+(z+1)^2\right)^2}\vec{e}_x +\left[\frac{2 (z+1)^2}{\left(x^2+(z+1)^2\right)^2}-\frac{1}{x^2+(z+1)^2}\right] \vec{e}_z \right)
\eeq
表面电荷分布
\beq
\sigma(x,y) = \left. \epv \vec{E} \cdot \vec{e}_z \right|_{z=0} = \epv \frac{2-x^2}{x^4}
\eeq
\par
容易计算的情形是$V(x,y) = V(\sqrt{x^2 + y^2}) \equiv V(r)$的情形。给定$\vec{k} = (k_1,k_2)$,记$\theta$为$\vec{k}$和$\vec{r}$之间的夹角。容易想到,现在$c(k_1,k_2)$也只依赖于$\vec{k}$的模长$k$。在极坐标下积分,有
\bea
c(k) &= \int_{r=0}^\infty rdr \int_{\theta = 0}^{2\pi} V(r) e^{-ikr\cos \theta} d\theta \\
&= 2\pi \int_{r=0}^\infty rdr V(r) J_0(kr)
\eea
其中引入了贝塞尔函数$J_0$及其积分表达式$J_0(x) = \frac{1}{2\pi}\int_{-\pi}^\pi  e^{ix\cos \theta} d\theta$。\par
全空间的电场于是为
\bea
\phi(r,z) &= \int_{k=0}^\infty \frac{kdk}{(2\pi)^2} \int_{\varphi=0}^{2\pi} d\varphi \left(2\pi \int_{r'=0}^\infty r'dr' V(r') J_0(kr) \right) e^{ikr\cos \varphi} e^{-kz} \\
&= \int_{k=0}^\infty kdk e^{-kz} J_0 (kr)  \int_{r'=0}^\infty r' dr' V(r') J_0(kr')
\eea
此处必须知道$V(r)$的具体表达式,否则无法继续再化简。例如当$V(r) = e^{-\frac{r^2}{2a^2}}$,后一个积分等于$a^2 e^{-\frac{1}{2} a^2 k^2}$,再带入积分就得到
\beq
\phi(r,z) = a^2 \int_{k=0}^\infty kdk e^{-kz-\frac{1}{2} a^2 k^2} J_0 (kr)
\eeq
这里还是化简不了,不过我们依然是得到了一个积分表达式。可以用这个表达式进行数值计算观察势的行为。结果如\cref{2_2p1}。
\cpicn{0.3}{2_2p1}{本例中电势随$(r,z)$的分布}
可以看到,在$r$方向,电势大致呈高斯型,特别是在$r=0$处电势的导数为0,体现出原有电势分布的对称性;而在$z$方向,电势大致是按$\sim e^{-z/a}$下降的,这点可以从$z=0$处电势的导数非零看出。



\end{ans}

\subsubsection{扇形区域分离变量}
采用柱坐标在扇形或圆形区域分离变量时,解的形式依赖于$z$轴依赖是否存在。当电势不依赖于$z$时,可以假设$\phi(r,\theta) = R(r) \Theta(\theta)$,进而有
\beq
\Theta \frac{1}{r} \frac{d}{dr} \left( r \frac{dR}{dr}\right) + R \frac{1}{r^2} \frac{d^2 \Theta}{d \theta^2} = 0
\eeq
进一步整理得到
\beq
\frac{r}{R} \frac{d}{dr} \left( r \frac{dR}{dr} \right) = - \frac{1}{\Theta} \frac{d^2 \Theta}{d \theta^2} = \nu^2
\eeq
角向方程的解写成
\beq
\Theta(\theta) = c_\nu \cos \nu \theta + s_\nu \sin \nu \theta
\eeq
当$\nu=0$时,由于二阶常微分线性方程总有两个线性无关解,$\Theta = c_0$不足以构成完备解,但容易验证,$\Theta = c_0 + s_0 \theta$也满足$\nu=0$时的方程。
$\nu$的取值与边界条件有关,当形区域为圆形区域时,需要添加自然周期性边界条件(PBC)$\Theta(\theta + 2\pi) = \Theta(\theta)$,这要求$\nu$只能取整数,且$s_0=0$。
\par
径向方程整理为
\beq
r^2 \frac{d^2 R}{dr^2} + r \frac{dR}{dr} - \nu^2 R = 0
\eeq
这个方程可以简单求解如下:假设$R \sim r^p$,有
\beq
p(p-1) + p - \nu^2 = 0
\eeq
当$\nu \not= 0$时,解为$R \sim r^\nu + r^{-\nu}$;当$\nu=0$,由于二阶常微分线性方程总有两个线性无关解,$R \sim \mathrm{const.}$不足以构成完备解,但容易验证,$R \sim \ln r$也满足$\nu=0$时的方程。
\par
于是,无$z$依赖时,扇形区域分离变量的形式解写成
\beq
\phi(r,\theta) = (a_0 + b_0 \ln r)(c_0 + s_0 \theta) + \sum_{\nu} \left(a_\nu r^\nu + b_\nu \frac{1}{r^\nu} \right) \left( c_\nu \cos \nu \theta + s_\nu \sin \nu \theta \right)
\eeq
当考虑的空间区域包括原点时,电势在原点不发散要求$b$项全部为0;剔除原点时,所有项都可能对最终表达式做贡献。
\begin{eg}
空间内有一半径为$a$的无限长圆柱面,其轴与$z$轴重合。圆柱面上的电势$V(\theta)$给定且只依赖于极角。求全空间的电势分布。
\end{eg}
\begin{ans}
按之前的讨论,在$r<a$范围内,解按
\beq
\phi(r,\theta) = \sum_{m=0}^\infty \left( \frac{r}{a} \right)^m \left( c_m \cos m \theta + s_m \sin m \theta \right)
\eeq
展开。令$r=a$得到
\beq
V(\theta) = \sum_{m=0}^\infty  \left( c_m \cos m \theta + s_m \sin m \theta \right)
\eeq
于是
\bea
c_m &= \frac{1}{\pi} \int_{\theta=0}^{2\pi} V(\theta) \cos m \theta d\theta \\
s_m &= \frac{1}{\pi} \int_{\theta=0}^{2\pi} V(\theta) \sin m \theta d\theta
\eea
圆柱面以外的解只需要将$\frac{r}{a}$替换为$\frac{a}{r}$。
\par
例如,当$V(\theta) = V_0 \sin 2\theta$时,展开系数只有$s_2 = V_0$非零,故解为
\beq
\phi(r,\theta) = V_0 \min \{ \left( \frac{r}{a}\right)^2 ,\left( \frac{a}{r} \right)^2\}  \sin 2\theta
\eeq
对应的图像如$\cref{2_3p1}$。
\cpicn{0.12}{2_3p1}{$V(\theta) = V_0 \sin 2\theta$时全空间的电势分布}
\end{ans}
\par
下面一个例子说明扇形区域下的解法。
\begin{eg}
空间中一扇形区域边界为两个半无限大接地金属板,其张角为$\beta$。求扇形区域内电势分布和金属板上电荷分布的通解,并讨论$\beta \to 0$,$\beta = \pi$和$\beta \to 2\pi$的行为。
\end{eg}
\begin{ans}
由于原点包括在内,形式上的分离变量解为
\beq
\phi(r,\theta) = a_0 + \sum_\nu (a_\nu \cos \nu \theta + b_\nu \sin \nu \theta) r^\nu
\eeq
边界条件为
\bea
\phi(r,\theta=0) &= 0 \\
\phi(r,\theta=\beta) &= 0
\eea
可知$a_\nu = 0$,$\nu = n\pi \theta /\beta$。于是通解为
\beq
\phi(r,\theta) = a_0 + \sum_{n=1}^\infty b_n r^{\frac{n\pi }{\beta}} \sin \frac{n\pi \theta}{\beta}
\eeq
取一阶近似得到
\beq
\phi(r,\theta) \simeq a_0 + a_1 r^\frac{\pi }{\beta} \sin \frac{\pi \theta}{\beta}
\eeq
于是电场为
\bea
\vec{E} &\simeq -\pfrac{\phi}{r} \vec{e}_r - \frac{1}{r} \pfrac{\phi}{\theta} \vec{e}_\theta \\
&=-\frac{\pi }{\beta}a_1 r^{\frac{\pi }{\beta}-1} \sin \frac{\pi  \theta}{\beta} \vec{e}_r-\frac{\pi }{\beta} a_1 r^{\frac{\pi }{\beta}-1} \cos \frac{\pi \theta }{\beta} \vec{e}_\theta
\eea
电场大小$E \sim r^{\frac{\pi }{\beta}-1}$,板上的面电荷密度为
\beq
\sigma = \epv \vec{E} \cdot \vec{e}_\theta  = -\frac{\pi }{\beta} a_1 r^{\frac{\pi }{\beta}-1}\sim r^{\frac{\pi}{\beta} - 1}
\eeq
\par
当$\theta \to 0$,图像上对应一个带狭缝的无限长金属棒。狭缝上的电荷分布$\sigma \sim r^\infty$,即在原点处,狭缝边缘几乎没有电荷,跟容易想出这是由于该处电荷间的强烈排斥导致的。
\par
当$\theta = \pi$,图像上对应一个无限大金属板。此时的电场分布为
\beq
\vec{E} \simeq - a_1 \sin \theta \vec{e}_r - a_1 \cos \theta \vec{e}_\theta = -a_1 \vec{e}_y
\eeq
对应一个匀强电场。从电磁学知识我们已经知道,一个无限大带电金属板对应的电场就是一个匀强电场,于是系数$a_1=-E$。再看面电荷分布,有
\beq
\sigma = - \epv a_1 =\epv E
\eeq
与熟知的结果$E = \frac{\sigma}{\epv}$一致。
\par
当$\theta \to 2\pi$,图像上对应一个二维劈尖,面电荷分布和电场大小$\sigma \sim E\sim \frac{1}{\sqrt{r}}$,即在尖端处,电场和面电荷都非常大,之后随远离尖端而衰减,符合尖端放电的物理事实。
\end{ans}
\subsubsection{锥形区域分离变量}
\subsubsection{电荷格林函数}

























\end{document}
