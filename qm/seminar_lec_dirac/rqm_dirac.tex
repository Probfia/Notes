\documentclass[aspectratio=1610,14pt,matheuler]{beamer}
\batchmode
%\input{macros.tex}
\usepackage{xeCJK}
\setCJKmainfont{STKaiti}
\usepackage{beamerthemesplit}
\usetheme{CambridgeUS}
\useoutertheme{smoothbars}
\usecolortheme{beaver}

\usepackage{xcolor}
\usepackage{amsmath}
\usepackage{amssymb}
\usepackage{graphicx}
\usepackage{eufrak}
\usepackage{color}
\usepackage{slashed}
\usepackage{tcolorbox}

%\def\bch{\begin{CJK}{UTF8}{gbsn}}
%\def\ech{\end{CJK}}
\newcommand{\bch}{}
\newcommand{\ech}{}
\def\bcenter{\begin{center}}
\def\ecenter{\end{center}}
\def\skipline{{\vskip0.1in}}
\def\skiplines{{\vskip0.2in}}
\def\tbox#1{\begin{tcolorbox}#1\end{tcolorbox}}

\newcommand{\field}{\mathscr{F}}

\newcommand{\reals}{\mathbb{R}}
\newcommand{\complexs}{\mathbb{C}}
\newcommand{\ints}{\mathbb{Z}}
%\newcommand{\dim}{\mathrm{dim\ }}
\newcommand{\diag}{\mathrm{diag \ }}
\newcommand{\up}{\uparrow}
\newcommand{\down}{\downarrow}
\newcommand{\su}{\mathfrak{su}}
\newcommand{\so}{\mathfrak{so}}
\DeclareMathOperator{\tr}{tr}
\newcommand{\card}{\mathrm{card \ }}
\newcommand{\mani}{\mathcal{M}}
\newcommand{\lag}{\mathcal{L}}
\newcommand{\ham}{\mathcal{H}}
\def\secpage#1#2{\begin{frame}\bch\bcenter{\bf \Huge #1} \skipline \tbox{#2}\ecenter\ech\end{frame}}
\newcommand{\mat}[1]{\begin{pmatrix}#1\end{pmatrix}}

\newcommand{\red}[1]{{\color{red} #1}}
\def\green#1{{\color[rgb]{0.1,0.7,0.2}#1}}
\newtheorem{thm}{定理}[section]
\newtheorem{axm}{公理}[section]
\newtheorem{dfn}{定义}[section]

%\cpic{<尺寸>}{<文件名>}}用于生成居中的图片。
\newcommand{\cpic}[2]{
\begin{center}
\includegraphics[scale=#1]{#2}
\end{center}
}

%\cpicn{<尺寸>}{<文件名>}{<注释>}用于生成居中且带有注释的图片,其label为图片名。
\newcommand{\cpicn}[3]
{
\begin{figure}[h!]
\cpic{#1}{#2}
\caption{#3\label{#2}}
\end{figure}
}

\title{Relativistic Quantum Mechanics: Dirac Eq (III)}
\substitle{Spin, $g$-factor and $\gamma$ Matrix Trace Techniques.}
  \date{\today}


\begin{document}

\begin{frame}
 
\titlepage
\begin{center}
\tableofcontents
\end{center}

\end{frame}

\section{Spin, Helicity and Chirality}
\secpage{自旋角动量}{$$\vec J = \vec L + \frac{1}{2} \vec \varSigma$$}

\begin{frame}{复习:狄拉克方程和旋量 }
\frametitle{\bch 复习:狄拉克方程和旋量 \ech}
\bch
\begin{itemize}
\item 回忆:我们把KG方程的二次算符$H^2=\vec p^2 + m^2$强行拆成两个一次算符$H=\vec \alpha \cdot \vec p + \beta m$得到了狄拉克方程,因此它首先具有我们希望的相对论色散关系。
\item 狄拉克表象下的
$$
\vec \alpha = \mat{0 & \vec \sigma \\ \vec \sigma & 0},\ \beta = \mat{1 & 0 \\ 0 & -1}
$$
\item
狄拉克方程的波函数是一个4分量的列向量$\psi$,称为旋量。其中的两个自由度描述正能量解,对应正粒子;另外两个自由度描述负能量解,对应反粒子。
\item
我们还被告知,正能量解包含的2个自由度描述了一个自旋$1/2$粒子,但为什么呢?这是我们这节课要回答的问题。
\end{itemize}
\ech
\end{frame}

\begin{frame}
\frametitle{\bch 守恒量 \ech}
\bch
\begin{itemize}
\item 哈密顿量
$$
H = \vec \alpha \cdot \vec p + \beta m
$$
同时包括了外部时空坐标的算符($\vec p$,连续谱)和内部旋量空间的算符($\vec \alpha,\ \beta$,离散谱)。
\item 我们知道,量子力学中与哈密顿量对易的可观测量算符是守恒量。例如,对薛定谔的$H = \frac{\vec p^2}{2m}$,轨道角动量$\vec L = \vec x \times \vec p$守恒,$[\vec  L, H ] =0$。
\item 在狄拉克的$H$中,$\vec L$还守恒吗?\includegraphics[scale=0.12]{speechless}

\end{itemize}
\ech
\end{frame}

\begin{frame}
\frametitle{\bch $[\vec L,H] \not= 0$ \ech}
\bch
\begin{itemize}
\item 我们开始了:
$$
[L_i,H] = [\epsilon_{ijk} x_j p_k,\alpha_l p_l + \beta m] = \epsilon_{ijk}[x_j p_k,p_l]\alpha_l
$$
对易子可以用基本对易关系化简
$$
[x_j p_k,p_l]  = x_j[p_l,p_k]+[x_j,p_l]p_k = i\delta_{jl}p_k
$$
于是
$
[L_i,H] = i \epsilon_{ijk} \delta_{jl}p_k \alpha_l =  i \epsilon_{ijk} \alpha_j p_k
$
$$
[\vec L,H] = i \vec \alpha \times \vec p \not= 0
$$
\cpic{0.2}{oops}
\item 讨论:轨道角动量算符不守恒来自外部坐标空间的算符非对易性。
\end{itemize}
\ech
\end{frame}

\begin{frame}
\frametitle{\bch 自旋角动量算符 \ech}
\bch
\begin{itemize}
\item
定义自旋角动量算符
$$\vec \varSigma = \mat{\vec \sigma & 0 \\ 0 & \vec \sigma} = \gamma^5 \vec \alpha$$
其中$\gamma^5 = \mat{0 & 1 \\ 1 &0}$是狄拉克表象下的$\gamma^5$。
\item
先证明:
$$[\varSigma_i,\alpha_j] = 2i\epsilon_{ijk} \alpha_k$$
\end{itemize}
\ech
\end{frame}

\begin{frame}
\frametitle{\bch $[\vec \varSigma, H ] \not= 0$ \ech}
\bch
\begin{itemize}
\item
$[\varSigma_i,\alpha_j] = \mat{\sigma_i & 0 \\ 0  & \sigma_i} \mat{0 & \sigma_j \\ \sigma_j &0}- \mat{0 & \sigma_j \\ \sigma_j &0} \mat{\sigma_i & 0 \\ 0 & \sigma_i} = \mat{0 & \sigma_i \sigma_j \\ \sigma_i \sigma_j &0}-\mat{0 & \sigma_j \sigma_i \\ \sigma_j \sigma_i & 0} = \mat{0 & [\sigma_i,\sigma_j] \\ [\sigma_i,\sigma_j] & 0}$
\item 利用泡利矩阵的著名对易关系$[\sigma_i,\sigma_j] = 2i \epsilon_{ijk} \sigma_k$就有
$$[\varSigma_i,\alpha_j] = 2i\epsilon_{ijk} \mat{0 &\sigma_k \\ \sigma_k &0 } = 2i\epsilon_{ijk} \alpha_k$$
\item
于是
$
[\varSigma_i ,H] = [\varSigma_i,\alpha_j p_j + \beta m]=  [\varSigma_i,\alpha_j]p_j = 2i\epsilon_{ijk} p_j \alpha_k
$
$$
[\vec \varSigma, H] = 2i \vec p \times \vec \alpha \not= 0
$$
\item 自旋角动量也不守恒,来源于内部旋量空间算符的非对易性。
\end{itemize}
\ech
\end{frame}

\begin{frame}
\frametitle{\bch $[\vec J, H ] = 0$ \ech}
\bch
\begin{itemize}
\item
我们有$[\vec L , H] = i \vec \alpha \times \vec p$和$[\vec \varSigma, H] = 2i\vec p \times \vec \alpha$。
\item
于是$$[\vec L + \frac{1}{2} \vec \varSigma,H] = 0$$
\item
定义总角动量算符
{\color{blue}
$$ \vec J = \vec L + \frac{1}{2} \vec \varSigma$$}
则$$[\vec J,H]=0$$
\item 总角动量算符守恒!并且可以看出,狄拉克方程描述了自旋$1/2$粒子。
\end{itemize}
\ech
\end{frame}

\begin{frame}
\frametitle{\bch 螺旋度也是守恒量 \ech}
\bch
\begin{itemize}
\item
因为$[\vec \varSigma, H] = 2i\vec p \times \vec \alpha$,$[\vec p,H]=0$(对自由粒子)。
\item
于是$[\vec \varSigma\cdot \vec p,H] = 0$,即自旋(的两倍)在动量方向的投影为一个守恒量。
\item
定义螺旋度算符
{\color{blue}
$$ h = \frac{\vec \varSigma \cdot \vec p}{|\vec p|}$$}
则$$[h,H]=0$$
\item $h$的本征值为$\pm 1$,分别为右手粒子和左手粒子。
\cpic{0.2}{helicity}
\end{itemize}
\ech
\end{frame}

\begin{frame}
\frametitle{\bch 课堂小练习:Chirality and Helicity \ech}
\bch
\begin{itemize}
\item
设质量为$m$的粒子沿$z$轴匀速运动,$\vec p = p \vec e_z$,显式写出狄拉克的哈密顿量$H=\vec \alpha \cdot \vec p + \beta m$和螺旋度算符$h$。
\item
用{\color{red}Mathematica}算出这个哈密顿量的本征值和本征矢为
$$
+E:\mat{E+m \\ 0 \\-p \\ 0},\ \mat{0 \\ E+m \\0 \\-p};\quad -E: \mat{m-E\\0\\-p\\0},\ \mat{0\\m-E\\0 \\p}
$$
验证上面这4个本征矢是$h$的本征矢并计算对应的本征值。
\item 证明:对一般的狄拉克哈密顿量,手性算符$\gamma_5=\mat{0&1\\1&0}$与$H$仅当$m=0$时对易。
\item 计算$\gamma_5$对上面四个$H$本征矢的作用,证明:{\color{blue} 仅当$m=0$时,$h$的本征矢和$\gamma_5$的本征矢重合。}
\end{itemize}
\ech
\end{frame}

\begin{frame}
\frametitle{\bch Chirality and Helicity \ech}
\bch
\begin{itemize}
\item
$h$和哈密顿量对易,因此一个自由粒子的螺旋度是不变的;但对有质量粒子$\gamma_5$和哈密顿量不对易,因此{\color{blue}一个有质量自由粒子的手性会不断震荡。}
\item
$\gamma_5$只依赖于内部旋量空间,因此空间洛伦兹变换不会改变手性;但对{\color{blue}有质量粒子$h$非洛伦兹不变},因为我总可以找到一个比粒子快的坐标系使得动量反向(自旋不变,从而$h$变号)。
\item {\color{blue}对无质量粒子,手性和螺旋度重合},这样,粒子的手性也不震荡了;螺旋度也洛伦兹不变了。
\item 总结:
\begin{tabular}{l|l|l}

 & 有质量粒子 & 无质量粒子 \\ \hline
$\gamma_5$本征态:手性 & \red{不守恒}但\green{洛伦兹不变} & \green{守恒且洛伦兹不变} \\ \hline
$h$本征态:螺旋度 & \green{守恒}但\red{非洛伦兹不变} & \green{守恒且洛伦兹不变}\\
\end{tabular}
\end{itemize}
\ech
\end{frame}


\section{EM fields, $g$-factor, LL and Anomalous Transport}
\secpage{电磁场中的粒子:朗道能级,$g$因子和QCD反常运输}{$$E = \sqrt{m^2 + p_z^2 + 2neB}$$}
\begin{frame}
\frametitle{\bch 引入电磁场 \ech}
\bch
\begin{itemize}
\item
引入电磁场的最小耦合标准手续是$\vec p \to \vec p - q \vec A$,相当于把导数变成协变导数。
\item
引入电磁场后,狄拉克的哈密顿量是
$$
H = \vec \alpha \cdot (\vec p - q \vec A) + \beta m + q\phi
$$
注意这里的$\vec p$都是算符$-i\vec \nabla$。
\item
设旋量波函数$\psi = \mat{u \\ v}$,我们关注前两个代表正粒子的分量$u$。
\end{itemize}
\ech
\end{frame}

\begin{frame}
\frametitle{\bch 电磁场中的定态狄拉克方程 \ech}
\bch
\begin{itemize}
\item
代入狄拉克表象下的$\alpha$和$\beta$,有
$$
\mat{m + q\phi & -\vec \sigma \cdot( i\vec \nabla + q\vec A) \\ -\vec \sigma \cdot( i\vec \nabla + q\vec A) & -m + q\phi} \mat{u \\ v} = E\mat{u \\v}
$$
\item
耦合的方程组
$$
\begin{cases}
(m + q\phi - E) u - \vec \sigma \cdot( i\vec \nabla + q\vec A) v = 0\\
-\vec \sigma \cdot( i\vec \nabla + q\vec A) u + (-m - E + q\phi)v = 0
\end{cases}
$$
\item 因为我们只关注正能量解$u$,因此我们可以从第2式中解出$$v = \frac{-\vec \sigma \cdot( i\vec \nabla + q\vec A)}{m+E - q\phi}u$$
\end{itemize}
\ech
\end{frame}


\begin{frame}
\frametitle{\bch 电磁场中的定态狄拉克方程 \ech}
\bch
\begin{itemize}
\item
代入第一式就得到
$$
[(E - q\phi)^2-m^2]u = \left[\vec \sigma \cdot( i\vec \nabla + q\vec A)\right]^2 u
$$
处理右边项的时候要特别小心:因为这是个算符。
\item
$\left[\vec \sigma \cdot( i\vec \nabla + q\vec A)\right]^2 u = \sigma_i (i\partial_i + qA_i) \sigma_j (i\partial_j + qA_j) u = \sigma_i \sigma_j (i\partial_i + qA_i)(i\partial_ju + qA_j u) = (\delta_{ij} + i \epsilon_{ijk}\sigma_k) ({\color{cyan}-\partial_i \partial_j u} +{\color{orange} iq(\partial_i A_j) u }+ {\color{cyan} iqA_j \partial_i u + iqA_i \partial_j u + q^2 A_i A_j u})$

\item {\color{cyan}青色}的项是全对称的,$\epsilon_{ijk}$对它们的作用为0,因此
$$
\left[\vec \sigma \cdot( i\vec \nabla + q\vec A)\right]^2 u = (-\nabla^2 + 2iq \vec A \cdot \vec \nabla + q^2 \vec A^2)u - q (\vec \nabla \times \vec A)\cdot \vec \sigma u$$
\end{itemize}
\ech
\end{frame}


\begin{frame}
\frametitle{\bch 磁场的直接作用 \ech}
\bch
\begin{itemize}
\item
因为$\vec B = \vec \nabla \times \vec A$,我们发现,狄拉克方程中,磁场直接出现在方程里
$$
[(E - q\phi)^2-m^2]u = (-\nabla^2 + 2iq \vec A \cdot \vec \nabla + q^2 \vec A^2)u -{\color{red} q \vec B \cdot \vec \sigma u}$$

\item 这是狄拉克方程中独有的性质:在KG方程和薛定谔方程中这一项都不会自然出现(必须人为引入),而且我们发现,这一项于$\vec \sigma$相关,因此事实上描述了自旋和磁场的耦合。
\end{itemize}
\ech
\end{frame}


\begin{frame}
\frametitle{\bch 匀强磁场 \ech}
\bch
\begin{itemize}
\item
假设我们只有匀强磁场$\vec B = B \vec e_z$,选取$\vec A = Bx \vec e_y$,那么我们的方程化为
$$
(E^2 - m^2) u = (-\nabla^2 + 2iq Bx \partial_y + (qBx)^2)u - q B \sigma_3 u
$$
\item
方程不显含$y,z$,我们依然按套路$u \sim e^{ip_y y + ip_z z}$分量变量,有
$$
(E^2 - m^2) u = (- \frac{d^2}{dx^2} + p_y^2 + p_z^2  - 2q Bp_y x + (qBx)^2)u - q B \sigma_3 u
$$
或
$$
(E^2 - m^2 -p_z^2 + qB \sigma_3 )u = \left(- \frac{d^2}{dx^2} +(qBx - p_y)^2\right) u
$$
\end{itemize}
\ech
\end{frame}

\begin{frame}
\frametitle{\bch 分离上下旋 \ech}
\bch
\begin{itemize}
\item
我们还依稀记得$u$是一个2分量的旋量,进一步分解$u = \mat{u_+\\u_-}$,发现这个方程是自然解耦的
$$
\begin{cases}
(E^2 - m^2 -p_z^2 + qB)u_+ = \left(- \frac{d^2}{dx^2} +(qBx - p_y)^2\right) u_+ \\
(E^2 - m^2 -p_z^2 - qB)u_- = \left(- \frac{d^2}{dx^2} +(qBx - p_y)^2\right) u_-
\end{cases}
$$
\item 合写成
$$
(E^2 - m^2 -p_z^2 \pm qB)u_\pm = \left(- \frac{d^2}{dx^2} +(qBx - p_y)^2\right) u_\pm
$$
\end{itemize}
\ech
\end{frame}

\begin{frame}
\frametitle{\bch 对比谐振子 \ech}
\bch
\begin{itemize}
\item
继续以前的套路,我们对比谐振子方程
$$
Eu = \left(- \frac{1}{2m} \frac{d^2}{dx^2} + \frac{1}{2} m\omega^2 x^2\right)u \Rightarrow E= \left(n+\frac{1}{2}\right) \omega
$$
和方程
$$
(E^2 - m^2 -p_z^2 \pm qB)u_\pm = \left(- \frac{d^2}{dx^2} +(qBx - p_y)^2\right) u_\pm
$$
的系数,简单变形:
$
\frac{E^2 - m^2 -p_z^2 \pm qB}{2m} u_\pm = \left(- \frac{1}{2m} \frac{d^2}{dx^2} +\frac{1}{2} m\left(\frac{qB}{m}\right)^2\left(x - \frac{p_y}{qB}\right)^2\right) u_\pm
$
\end{itemize}
\ech
\end{frame}

\begin{frame}
\frametitle{\bch 朗道能级 \ech}
\bch
\begin{itemize}
\item
于是我们发现,磁场中的狄拉克粒子相当于一个频率为$\omega = \frac{qB}{m}$,中心位于$x_B = \frac{p_y}{qB}$的谐振子。能级为
$$
\frac{E^2 - m^2 -p_z^2 \pm qB}{2m} = \left(n+ \frac{1}{2}\right) \frac{qB}{m}
$$
解出
$$
E^{(\pm)}_n(p_z) = \sqrt{m^2 + p_z^2 + (2n+1)qB \mp qB}
$$
\item
讨论:朗道能级依赖于3个量$n,p_z,m_s$,而在KG方程那里我们得到过$E_n(p_z) = \sqrt{m^2 + p_z^2 + (2n+1)qB}$,{\color{blue}狄拉克方程的区别在于出现了自旋与磁场相互作用的附加能量}。
\end{itemize}
\ech
\end{frame}


\begin{frame}
\frametitle{\bch 电子$g$因子 \ech}
\bch
\begin{itemize}
\item
上旋粒子的最低能量为$$E^{(+)}_0(p_z) = \sqrt{m^2 + p_z^2} \simeq m+ \frac{p_z^2}{2m}$$
\item
下旋粒子的最低能量为$$E^{(-)}_0(p_z) = \sqrt{m^2 + p_z^2 + 2qB} \simeq m+ \frac{p_z^2}{2m} + \frac{qB}{m}$$
\item
上下自旋的磁矩分别为$\mu_z = g \mu_B m_s = \pm \frac{g}{2} \frac{q}{2m}$,能量差为$\Delta E = \vec B \cdot \Delta \vec \mu = g \frac{qB}{2m}$,和狄拉克方程朗道能级的结果对比就得到了狄拉克粒子的朗德因子$$g=2$$
\end{itemize}
\ech
\end{frame}

\begin{frame}
\frametitle{\bch 理论的美 \ech}
\bch
\begin{itemize}
\item
我们只是把KG方程的二阶导数拆成了两个一阶算符的积得到了狄拉克方程,就自然地得到了自旋$1/2$和电子$g$因子,而这些在非相对论量子力学中都只能是实验测得的参数。
\item
QED的耦合常数不是一个常数,而有重整化带来的修正,这又可以得到电子的反常磁矩$g = 2\left(1 + \frac{\alpha}{2\pi} + O(\alpha^2)\right)$。
\item
根据场论里的自旋-统计定理又可以知道自旋$1/2$粒子只能是费米子,而这在非相对论性量子力学中又只能是一个实验事实。
\end{itemize}
\ech
\end{frame}

\begin{frame}
\frametitle{\bch 特定自旋粒子朗道能级的简并度 \ech}
\bch
\begin{itemize}
\item
假设我们的空间有限,在$x$,$y$方向的长度分别是$L_x,L_y$,那么,$p_y$只能取离散值$p_y = \frac{n\pi}{L_y}$。
\item
谐振子中心为$x_c = \frac{p_y}{qB} = \frac{n\pi}{qB L_y}$,它同样不能超过盒子,于是有
$$
\frac{n\pi}{qB L_y} < L_x
$$
\item
于是我们得到对特定自旋粒子,朗道能级的简并度
$$
g^{(\pm)}(n) = \frac{qBA}{\pi}
$$
\end{itemize}
\ech
\end{frame}

\begin{frame}
\frametitle{\bch 朗道能级的简并度 \ech}
\bch
\begin{itemize}
\item
我们发现,上下旋粒子的朗道能级都可以写成
$$
E = \sqrt{m^2 +p_z^2 + 2nqB}
$$
但{\color{blue} $n=0$的朗道能级(最低朗道能级,LLL)只有自旋向上的粒子占据,而高朗道能级可以被两种自旋的粒子占据}。
\item
LLL只容纳自旋向上粒子的事实非常重要!
\item
简并度是(不考虑其他量子数)能量相同的所有态的数目,于是我们发现朗道能级的简并度准确地应该写成
$$
g(n) = 
\begin{cases}
\frac{qBA}{\pi}, n = 0 \\
2g(0), n \geq 1
\end{cases}
$$
\end{itemize}
\ech
\end{frame}


\begin{frame}
\frametitle{\bch QGP的LLL近似 \ech}
\bch
\begin{itemize}
\item 重离子非对心碰撞形成的夸克-胶子等离子体(QGP)拥有极强的磁场$eB \sim 10^{19} \mathsf{\ Gs}$。
\cpic{0.4}{mag_field}
\item 
有限温度和强磁场下的QGP,高朗道能级的粒子数都被玻尔兹曼因子$\sim e^{-\beta \sqrt{2neB}}$压低,因此,我们认为所有粒子都在LLL。这带来了{\color{blue}量子反常运输}现象。
\item 我们这里讨论{\color{blue}手征磁效应}和{\color{blue}手征分离效应}两种量子反常运输。

\end{itemize}
\ech
\end{frame}

\begin{frame}
\frametitle{\bch CME和CSE \ech}
\bch
{\small 量子反常运输源于LLL只容许自旋向上的正粒子和自旋向下的反粒子。}
\cpic{0.22}{anom_trans}
\ech
\end{frame} 
\begin{frame}

\frametitle{\bch 手征磁效应(CME) \ech}
\bch
\begin{itemize}
\item
LLL上的这群正粒子都自旋向上,负粒子因为带电相反,都自旋向下(保证低能量)。
\item
假设我们有一堆左手粒子和右手粒子,但右手粒子因为某些原因更多。因为LLL只容纳自旋向上的正粒子,右手正粒子全部往上跑;因为LLL只容纳自旋向下的负粒子,右手负粒子都往下跑,因此,我们有一个平行于外磁场方向的电流。
\item
估计电流的大小:正比于电量$e$,正比于LLL简并度$\sim eB$,正比于多出来的右手粒子数,以轴化学势$\mu_A$表征:{\color{blue}
$$
\vec j_V \sim e^2  \mu_A \vec B
$$
}
\item
这就是手征磁效应的电流。
\end{itemize}
\ech
\end{frame}

\begin{frame}
\frametitle{\bch 手征分离效应(CSE) \ech}
\bch
\begin{itemize}
\item
CME中,正负粒子一样多,而RH粒子多于LH粒子,产生一个电流$j_V \sim \mu_A$。
\item
对偶效应:LH粒子和RH粒子一样多,但正粒子多于负粒子,对偶地,它将产生一个手性流$j_A\sim \mu_V$。
\item
原因:在LLL下,正的RH粒子往上跑,正的LH粒子往下跑。等效于有一个向上运输的正螺旋($h$)流。
\item
估计手征流的大小:正比于电量$e$,正比于LLL简并度$\sim eB$,正比于多出来的正粒子数,以粒子化学势$\mu_V$表征:{\color{blue}
$$
\vec j_A \sim e^2  \mu_V \vec B
$$
}
\item
这就是手征分离效应的流。
\end{itemize}
\ech
\end{frame}

\begin{frame}
\frametitle{\bch 手征磁波(CMW) \ech}
\bch
\begin{itemize}
\item 我们看到,多出来的正电荷带来手性的向上流动;而多出来的手性又带来电荷的向上流动。
\item $j_A \sim \mu_V$,$j_V \sim \mu_A$,{\color{blue}两者交替激发形成手征磁波。}
\item 在强磁场极限下,CMW以光速传播(D. E. Kharzeev, H.-U. Yee, \emph{PRD}, 2010, {\color{olive}\url{https://arxiv.org/abs/1012.6026v2}})
\end{itemize}
\ech
\end{frame}

\begin{frame}
\frametitle{\bch 其他QCD反常运输 \ech}
\bch
\cpic{0.21}{at}
\ech
\end{frame}

\begin{frame}
\frametitle{\bch $\mu_V$和$\mu_A$的来源 \ech}
\bch
\begin{itemize}
\item 我们看到,QGP反常运输现象与手性不平衡和电荷不平衡密切相关;但为什么会凭空多出右手粒子/正粒子?
\item QCD真空拓扑涨落,$SU(3)$群独特的拓扑结构。
\cpic{0.26}{mu_a}
{\tiny For more on anomalous transport, see, e.g. D. E. Kharzeev, et al. {\color{olive}\url{http://arxiv.org/abs/arXiv:1511.04050}}, X.-G. Huang,{\color{olive} \url{https://arxiv.org/abs/1509.04073v3}}.}
\end{itemize}
\ech
\end{frame}


\section{$\gamma$ trace techniques}
\secpage{$\gamma$矩阵求迹技巧}{以后会有用的}


\begin{frame}
\frametitle{\bch 矩阵求迹的一般性质 \ech}
\bch
迹内部的矩阵乘积可以{\color{blue}轮换}顺序
$$
\tr ABC = \tr CAB = \tr BCA
$$
$$
\tr ABCD = \tr BCDA = \tr CDAB = \tr DABC
$$
但{\color{red}不成立
$$
\tr AB = \tr A \tr B
$$
和
$$
\tr ABC = \tr ACB
$$
等}
\ech
\end{frame}

\begin{frame}
\frametitle{\bch $\gamma^5$的一般性质 \ech}
\bch
\begin{itemize}
\item 我们之后的讨论都不依赖于具体的表象,而只利用基本定义$$\{\gamma^\mu,\gamma^\nu\} = 2\eta^{\mu \nu}$$
\item 定义$$\gamma^5 = i \gamma^0 \gamma^1 \gamma^2 \gamma^3$$
\item 证明:\begin{enumerate}
	\item $$(\gamma^5)^2 = 1$$
	\item $$\{\gamma^\mu ,\gamma^5\} = 0$$
	\item $$\tr \gamma^5 = 0$$
	\end{enumerate}
\end{itemize}
\ech
\end{frame}

\begin{frame}
\frametitle{\bch $(\gamma^5)^2 = 1$ \ech}
\bch
\begin{equation*}
\begin{aligned}
(\gamma^5)^2 &=- \gamma^0 \gamma^1 \gamma^2 \gamma^3\gamma^0 \gamma^1 \gamma^2 \gamma^3 \\
&= - \gamma^0 \gamma^1 \gamma^2( \{\gamma^3,\gamma^0 \}-\gamma^0\gamma^3)\gamma^1 \gamma^2 \gamma^3 \\
&=\gamma^0 \gamma^1 \gamma^2\gamma^0\gamma^3\gamma^1 \gamma^2 \gamma^3\\
&= {\color{teal}\dots(\text{再移两次}) }\\
&= \gamma^0\gamma^0\gamma^1 \gamma^2\gamma^3\gamma^1 \gamma^2 \gamma^3 \quad{\color{teal}(\gamma^0 \gamma^0 = \frac{1}{2}\{\gamma^0,\gamma^0\} = \frac{1}{2} 2\eta^{00} = 1)} \\
&= \gamma^1\gamma^1\gamma^2\gamma^3\gamma^2\gamma^3 \quad{\color{teal}(\gamma^1 \gamma^1 = \frac{1}{2}\{\gamma^1,\gamma^1\} = \frac{1}{2} 2\eta^{11} = -1)}\\
&= - \gamma^2\gamma^3\gamma^2\gamma^3	\\
&= \gamma^2\gamma^2\gamma^3\gamma^3\\
&= 1
\end{aligned}
\end{equation*}
\ech
\end{frame}

\begin{frame}
\frametitle{\bch 其他两个 \ech}
\bch
\begin{itemize}
\item
$$
\{\gamma^\mu,\gamma^5\} = i\gamma^\mu \gamma^0 \gamma^1 \gamma^2 \gamma^3 - i\gamma^0 \gamma^1 \gamma^2 \gamma^3 \gamma^\mu 
$$
{\small 把第一项的第一个$\gamma^\mu$往后移到第二个$\gamma^\mu$前,要移$\mu$次,添符号$(-1)^\mu$;把第二项的第二个$\gamma^\mu$移到第一个$\gamma^\mu$后,添符号$(-1)^{3-\mu} = -(-1)^\mu$,于是}
$$
\{\gamma^\mu,\gamma^5\} = i(-1)^\mu(\gamma^0 \dots \gamma^\mu \gamma^\mu \dots \gamma^3 - \gamma^0 \dots \gamma^\mu \gamma^\mu \dots \gamma^3) = 0
$$
\item
利用$(\gamma^0)^2 = 1$,$$\gamma^5 = \gamma^5 \gamma^0 \gamma^0 = - \gamma^0 \gamma^5 \gamma^0$$
两边取迹得到
$$
\tr \gamma^5 = - \tr \gamma^0 \gamma^5 \gamma^0 = -\tr \gamma^5 \gamma^0 \gamma^0 = -\tr \gamma^5
$$
于是
$$
\tr \gamma^5 = 0$$
\end{itemize}
\ech
\end{frame}

\begin{frame}
\frametitle{\bch 零迹定理 \ech}
\bch
\begin{itemize}
\item
定理:{\color{blue} 奇数个$\gamma^\mu$乘积的迹为0}。
\item
证明:
利用$(\gamma^5)^2=1$
$$\gamma^{\mu_1} \dots \gamma^{\mu_n} = \gamma^5\gamma^5 \gamma^{\mu_1} \dots \gamma^{\mu_n}$$
因为$\gamma^5$和所有$\gamma^\mu$都对易,把第二个$\gamma^5$移到最后,填符号$(-1)^n$
$$\gamma^{\mu_1} \dots \gamma^{\mu_n} = (-1)^n\gamma^5\gamma^{\mu_1} \dots \gamma^{\mu_n}\gamma^5$$
两边取迹,在右边的迹中把最后一个$\gamma^5$轮换到第一个消掉,得到
$$
\tr \gamma^{\mu_1} \dots \gamma^{\mu_n} = (-1)^n \tr \gamma^{\mu_1} \dots \gamma^{\mu_n}
$$
$n$为奇数时给出
$$
\tr \gamma^{\mu_1} \dots \gamma^{\mu_n}=0
$$
\end{itemize}
\ech
\end{frame}

\begin{frame}
\frametitle{\bch 思考 \ech}
\bch
如何处理偶数个$\gamma$矩阵乘积的迹?
\cpic{0.4}{dina}
\ech
\end{frame}

\begin{frame}
\frametitle{\bch 课堂小练习 \ech}
\bch
证明:$\tr \gamma^{\mu_1} \dots \gamma^{\mu_n} \gamma^5 = 0$,$n$为奇数
\cpic{0.4}{eat}
\ech
\end{frame}


\begin{frame}
\frametitle{\bch $\gamma$矩阵求迹公式列举 \ech}
\bch
记号:$\mu,\nu,\dots=0,1,2,3$,$i,j,\dots = 1,2,3$,$a,b,\dots = 1,2,3,5$;$d$为旋量空间的维数(最小且默认等于4)。
\begin{enumerate}
\item $\tr 1 = d (= 4)$;
\item $\tr \gamma^\mu \gamma^\nu = d\eta^{\mu \nu}$;
\item $\tr \gamma^a = 0$;
\item $\tr \gamma^\mu \gamma^\nu \gamma^5 = 0$;
\item $\tr \gamma^\mu \gamma^\nu \gamma^\rho \gamma^\sigma = d(\eta^{\mu \nu} \eta^{\rho \sigma} - \eta^{\mu \rho} \eta^{\nu \sigma} + \eta^{\mu \sigma} \eta^{\nu \rho})$.
\end{enumerate}
\ech
\end{frame}


\begin{frame}
\frametitle{\bch 后两个公式的证明 \ech}
\bch
\begin{itemize}
\item $$ \gamma^\mu \gamma^\nu \gamma^5 =(\{\gamma^\mu,\gamma^\nu\} - \gamma^\nu \gamma^\mu) \gamma^5 = 2\eta^{\mu \nu} \gamma^5 - \gamma^\nu \gamma^\mu \gamma^5$$
两边取迹得到
$$
\tr \gamma^\mu \gamma^\nu \gamma^5 = 2\eta^{\mu \nu} \tr \gamma^5 - \tr \gamma^\nu \gamma^\mu \gamma^5 = -\tr \gamma^\mu \gamma^\nu \gamma^5
$$
即$\tr \gamma^\mu \gamma^\nu \gamma^5 =0$
\item $\tr \gamma^\mu \gamma^\nu \gamma^\rho \gamma^\sigma = \tr (2\eta^{\mu \nu} - \gamma^\nu \gamma^\mu) \gamma^\rho \gamma^\sigma = 2\eta^{\mu \nu}  \tr \gamma^\rho \gamma^\sigma  - \tr  \gamma^\nu \gamma^\mu \gamma^\rho \gamma^\sigma = 2d\eta^{\mu \nu}\eta^{\rho \sigma} - \tr \gamma^\nu ( 2\eta^{\mu \rho} - \gamma^\rho \gamma^\mu) \gamma^\sigma = 2d\eta^{\mu \nu}\eta^{\rho \sigma}  - 2d \eta^{\mu \rho}\eta^{\nu \sigma} + \tr \gamma^\nu\gamma^\rho \gamma^\mu\gamma^\sigma = 2d\eta^{\mu \nu}\eta^{\rho \sigma}  - 2d \eta^{\mu \rho}\eta^{\nu \sigma}+ \tr \gamma^\nu\gamma^\rho (2\eta^{\sigma \mu} - \gamma^\sigma \gamma^\mu) = 2d(\eta^{\mu \nu}\eta^{\rho \sigma }- \eta^{\mu \rho}\eta^{\nu \sigma} + \eta^{\sigma \mu}\eta^{\nu \rho}) - \tr \gamma^\mu \gamma^\nu \gamma^\rho \gamma^\sigma$
于是
$$\tr \gamma^\mu \gamma^\nu \gamma^\rho \gamma^\sigma = d(\eta^{\mu \nu} \eta^{\rho \sigma} - \eta^{\mu \rho} \eta^{\nu \sigma} + \eta^{\mu \sigma} \eta^{\nu \rho})$$
\end{itemize}
\ech
\end{frame}

\begin{frame}
\frametitle{\bch Feynman $\slash$ 符号 \ech}
\bch
\begin{itemize}
\item
对任意4维矢量$a_\mu$,定义$\slashed{a} = a_\mu \gamma^\mu$。
\item
证明:$\slashed a \slashed b = ab - i a_\mu b_\nu \sigma^{\mu \nu}$,其中$\sigma^{\mu \nu} = \frac{i}{2}[\gamma^\mu ,\gamma^\nu]$,其中$ab = a_\mu b^\mu$是四维矢量的内积。
\end{itemize}
\ech
\end{frame}


\begin{frame}
\frametitle{\bch slash内积的公式的证明 \ech}
\bch
我们有
\begin{equation*}
\begin{aligned}
\slashed a \slashed b &= a_\mu \gamma^\mu b_\nu \gamma^\nu \\
&= a_\mu b_\nu \frac{1}{2}(\{\gamma^\mu,\gamma^\nu\} + [\gamma^\mu,\gamma^\nu])\\
&= a_\mu b_\nu \frac{1}{2} (2\eta^{\mu \nu} - 2i\sigma^{\mu \nu}) \\
&= a_\mu b^\mu - ia_\mu b_\nu \sigma^{\mu \nu}
\end{aligned}
\end{equation*}
\ech
\end{frame}


\begin{frame}
\frametitle{\bch Feynman $\slash$ 符号的迹 \ech}
\bch
利用之前求得的$\gamma$矩阵求迹公式和slash符号内积公式进一步证明
\begin{enumerate}
\item $\tr \slashed a = 0$;
\item $\tr \slashed a \slashed b = d ab$;
\item $\tr \slashed a \slashed b \slashed c \slashed d = d[(ab)(cd) - (ac)(bd) + (ad)(bc)]$;
\item 奇数个Feynman slash乘积的迹为零。
\end{enumerate}
\ech
\end{frame}

\begin{frame}
\frametitle{\bch 下标$\gamma$矩阵 \ech}
\bch
形式上定义$\gamma_\mu = \eta_{\mu \nu} \gamma^\nu$,证明
\begin{enumerate}
\item $\gamma_\mu \gamma^\nu = \delta_\mu^\nu -i \sigma_\mu^{\ \nu}$;
\item $\gamma_\mu \gamma^\mu = 4$;
\item $\slashed a = a^\mu \gamma_\mu$;
\end{enumerate}
第一个等式的证明:{\color{white} $\gamma_\mu \gamma^\nu = \eta_{\mu \rho} \gamma^\rho \gamma^\nu = \eta_{\mu \rho} \frac{1}{2} ( \{\gamma^\rho,\gamma^\nu\} + [\gamma^\rho,\gamma^\nu]) = \eta_{\mu \rho} \frac{1}{2} (2\eta^{\rho \nu} - 2i \sigma^{\rho \nu}) = \delta_\mu^\nu -i \sigma_\mu^{\ \nu}$}
\ech
\end{frame}


\begin{frame}
\frametitle{\bch 下标$\gamma$矩阵 \ech}
\bch
形式上定义$\gamma_\mu = \eta_{\mu \nu} \gamma^\nu$,证明
\begin{enumerate}
\item $\gamma_\mu \gamma^\nu = \delta_\mu^\nu -i \sigma_\mu^{\ \nu}$;
\item $\gamma_\mu \gamma^\mu = 4$;
\item $\slashed a = a^\mu \gamma_\mu$;
\end{enumerate}
第一个等式的证明:{ $\gamma_\mu \gamma^\nu = \eta_{\mu \rho} \gamma^\rho \gamma^\nu = \eta_{\mu \rho} \frac{1}{2} ( \{\gamma^\rho,\gamma^\nu\} + [\gamma^\rho,\gamma^\nu]) = \eta_{\mu \rho} \frac{1}{2} (2\eta^{\rho \nu} - 2i \sigma^{\rho \nu}) = \delta_\mu^\nu -i \sigma_\mu^{\ \nu}$}
\ech
\end{frame}

\begin{frame}
\frametitle{\bch slash记号和$\gamma$矩阵的积 \ech}
\bch
证明:
\begin{enumerate}
\item $\gamma_\mu \slashed a \gamma^\mu = -2 \slashed a$;
\item $\gamma_\mu \slashed a \slashed b \gamma^\mu = 4ab$;
\item $\gamma_\mu \slashed a \slashed b \slashed c \gamma^\mu = -2 \slashed c \slashed b \slashed a$
\end{enumerate}
证明方法:反复移位。
\ech
\end{frame}


















\end{document}