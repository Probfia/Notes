
\documentclass[a4paper,11pt]{ctexart}

\usepackage{amsmath}
\usepackage{color}
\usepackage{mathrsfs}
\usepackage[colorlinks,
            linkcolor=blue,
		 urlcolor=black]{hyperref}
\usepackage{graphicx}
\usepackage{cleveref}
\usepackage{float}

\crefname{equation}{}{}
\crefname{figure}{图}{图}
\crefname{footnote}{注释}{注释}

\newcommand{\beq}{\begin{equation}}
\newcommand{\eeq}{\end{equation}}
\newcommand{\bea}{\begin{equation}\begin{aligned}}
\newcommand{\eea}{\end{aligned}\end{equation}}
\newcommand{\red}{\color{red}}
\newcommand{\grandz}{\mathcal{Z}}
\newcommand{\lagden}{\mathcal{L}}
\newcommand{\lag}{\mathcal{L}}
\newcommand{\ham}{\mathcal{H}}
\newcommand{\diag}{\mathrm{diag}}
\newcommand{\emptyline}{\\ \ \\}
\newcommand{\op}{\hat{\mathcal{O}}}
\newcommand{\unit}[1]{\mathrm{\ #1}}
\newcommand{\tr}{\mathrm{tr \ }}

%\cpic{<尺寸>}{<文件名>}}用于生成居中的图片。
\newcommand{\cpic}[2]{
\begin{center}
\includegraphics[scale=#1]{#2}
\end{center}
}

%\cpicn{<尺寸>}{<文件名>}{<注释>}用于生成居中且带有注释的图片,其label为图片名。
\newcommand{\cpicn}[3]
{
\begin{figure}[H]
\cpic{#1}{#2}
\caption{#3\label{#2}}
\end{figure}
}

\title{Physics of quark-gluon plasma and high-energy heavy-ion collisions}
\author{Probfia}
\date{\today}

\begin{document}
\maketitle
\tableofcontents
\section{Introduction}
粒子物理标准模型中,有三代费米子共12个,规范玻色子共4个,以及希格斯玻色子。数学上讲,标准模型是$U(1)\times SU(2) \times SU(3)$。\par
采用自然单位制$\hbar = c = k_B = 1$。根据$\hbar c = 197\unit{MeV \cdot fm}$就可以得到$1\unit{fm} = \frac{1}{197}\unit{MeV^{-1}}$等自然单位制下的数值。\par
粒子物理实验中常用到散射截面的概念,它反应了粒子数随角度的分布
\beq
\frac{d \sigma}{d \varOmega} = \frac{dN}{d\varOmega dt}/nv
\eeq
方位角微元为$d\varOmega = \sin \theta d \theta d \phi$。
\subsection{群论简介}
群是一个带乘法运算的集合,与对称性密切相关。一个群可以有自己的表示,例如,考虑反射群$\{ I,P \}$,其中$P$是镜面反射算符满足$P^2 = I$。另外,考虑$Z_2 = \{ 1, -1\}$和自然乘法构成的群,它同样满足$(-1)^2 = 1$。可以建立两个群元素间的一一对应关系,因此说,$Z_2$群是反射群的一个表示。
\par
$SO(3)$群是三维空间中的旋转群,一般元素可以表示为$R_{\vec{n}} (\psi)$,意义为绕$\vec{n}$方向的轴进行$\psi$角度旋转。为了得到$SO(3)$群的表示,考虑一个小的角度$\delta \psi$,显然有
\beq
R_{\vec{n}} (\delta \psi) \vec{x}= \vec{x} + \delta \psi \vec{n} \times \vec{x} + o(\delta \psi)
\eeq
或者写成
\beq
R_{\vec{n}} (\delta \psi)_{ik} = \delta_{ik} -i (i \delta \psi \epsilon_{ijk} n_j )
\eeq
选取$n_j$为三维空间的正交基,就得到了三个表征$SO(3)$的生成元$(J_k)_{ij} = i \epsilon_{ikj}$
\beq
J_{1}=\left(\begin{array}{ccc}{0} & {0} & {0} \\ {0} & {0} & {-\mathrm{i}} \\ {0} & {\mathrm{i}} & {0}\end{array}\right) \quad, \quad J_{2}=\left(\begin{array}{ccc}{0} & {0} & {\mathrm{i}} \\ {0} & {0} & {0} \\ {-\mathrm{i}} & {0} & {0}\end{array}\right) \quad, \quad J_{3}=\left(\begin{array}{ccc}{0} & {-\mathrm{i}} & {0} \\ {\mathrm{i}} & {0} & {0} \\ {0} & {0} & {0}\end{array}\right)
\eeq
有限角度的旋转可以由无数个无穷小转动相乘而成,于是
\beq
R_{\vec{n}} (\psi) = \left[ R_{\vec{n}} \left( \frac{\psi}{N} \right) \right]^N \equiv e^{- i \psi \vec{n} \cdot \vec{J}}
\eeq
$\vec{J}$构成$SO(3)$的李代数$so(3)$。基本李括号是$[J_i,J_j] = i\epsilon_{ijk} J_k$。
\par
$SU(2)$群是二维特殊酉群,满足$U^\dagger U = I$和$\det U = 1$。同样考虑它的李代数,即考虑$U = I + i \epsilon A$。容易算出来$A$必须是厄米的。在$SO(3)$的例子中,因为各个$J$是反称的,对任意的$SO(3)$元,$\det R = \det e^{-i \psi \vec{n} \cdot \vec{J} } e^{-i \psi \tr \vec{n} \cdot \vec{J}} = 1$自然满足,但对于$SU(2)$,由于允许矩阵元是复数,必须外加条件$\tr A = 0$。以实数为域,$su(2)$的维数为变量数8减去约束方程的个数5,下面的三个泡利矩阵可以作为$su(2)$的基:
\beq
\sigma_1 = \begin{pmatrix} 0 &1 \\ 1 &0\end{pmatrix},\ \sigma_2 = \begin{pmatrix} 0 & -i \\ i & 0 \end{pmatrix},\ \sigma_3 = \begin{pmatrix} 1 & 0 \\  0 & -1 \end{pmatrix}
\eeq
$\sigma$矩阵间满足关系$\sigma_i \sigma_j = \delta_{ij}+i \epsilon_{ijk} \sigma_k$。为了保持$i\epsilon_{ijk}$为$su(2)$的结构常数,最好让$\sigma_i /2$作为基,这样,基本李括号才是
\beq
\left[\frac{1}{2}\sigma_i,\frac{1}{2}\sigma_j\right] = i\epsilon_{ijk} \frac{1}{2} \sigma_k
\eeq
同样地,作为李群的$SU(2)$中的元素就可以仿照$SO(3)$的情形写成
\beq
R_{\vec{n}} (\psi) = e^{-i \psi \vec{n} \cdot \frac{\vec{\sigma}}{2}}
\eeq
可以看到,在这种表示下$SU(2)$是以$4\pi$为周期的,因此也有$SO(3) \simeq SU(2)/Z_2$。

\subsection{夸克的味对称性与色荷}
核力事实上是夸克间强相互作用的剩余,而实验表明,核力并不对核子的种类作区分,即$p-p,\ p-n,\ n-n$间的核力都是差不多的,因此可以说,质子态和中子态间有旋转对称性,这一旋转同样用$SU(2)$群表征。\par
夸克模型建立后,发现,虽然$u,d,t$三种夸克的质量不相同,但强相互作用依然对它们几乎不作区分,它们之间具有$SU(3)$旋转对称性。\par
$SU(3)$的李代数维数为8,盖尔曼矩阵为它的一组基
\bea
&\lambda_{1}=\left(\begin{array}{lll}{0} & {1} & {0} \\ {1} & {0} & {0} \\ {0} & {0} & {0}\end{array}\right) \quad \lambda_{2}=\left(\begin{array}{rrr}{0} & {-i} & {0} \\ {i} & {0} & {0} \\ {0} & {0} & {0}\end{array}\right) \quad \lambda_{3}=\left(\begin{array}{rrr}{1} & {0} & {0} \\ {0} & {-1} & {0} \\ {0} & {0} & {0}\end{array}\right) \\
&\lambda_{4}=\left(\begin{array}{lll}{0} & {0} & {1} \\ {0} & {0} & {0} \\ {1} & {0} & {0}\end{array}\right) \quad \lambda_{5}=\left(\begin{array}{ccc}{0} & {0} & {-i} \\ {0} & {0} & {0} \\ {i} & {0} & {0}\end{array}\right)\\
&\lambda_{6}=\left(\begin{array}{lll}{0} & {0} & {0} \\ {0} & {0} & {1} \\ {0} & {1} & {0}\end{array}\right) \quad \lambda_{7}=\left(\begin{array}{lll}{0} & {0} & {0} \\ {0} & {0} & {-i} \\ {0} & {i} & {0}\end{array}\right) \quad  \lambda_{8}=\frac{1}{\sqrt{3}}\left(\begin{array}{ccc}{1} & {0} & {0} \\ {0} & {1} & {0} \\ {0} & {0} & {-2}\end{array}\right)
\eea
定义夸克态
\beq
u = (1,0,0)^T,\quad d = (0,1,0)^T,\quad s = (0,0,1)^T
\eeq
以及反夸克态
\beq
\bar{u} = (-1,0,0)^T ,\quad \bar{d} = (0,-1,0)^T,\quad \bar{s} = (0,0,-1)^T
\eeq
对于8个盖尔曼矩阵中的两个对角阵,定义
\beq
I_3 = \frac{1}{2} \lambda_3,\quad Y = \frac{1}{\sqrt{3}} \lambda_8
\eeq
夸克和反夸克态都是这两个矩阵的本征态,把对应的本征值画在$(I_3,Y)$平面上,就得到了所谓的夸克的权重图,如\cref{wd}。这称作夸克的味。此外,可以像对自旋那样定义3组(6个)升降阶算符,实现夸克味间的转换。
\cpicn{0.8}{wd}{夸克三种味的weight diagram.}
\par
但在历史上,发现一些重子和介子由3个或2个完全相同的夸克组成,例如$\Delta^{++} = (uuu)$。这违背了泡利不相容原理。为了解决这一问题,只能给夸克额外增加一个称为色的自由度,色有三种,分别为红,绿,蓝三色(及其反色)。
\par
引入色后,为了构造强相互作用的理论,类比电到电荷的过程,将色推广到色荷,最终可以写出量子色动力学的拉氏量
\beq
\mathcal{L}_{\mathrm{QCD}}=\overline{\psi}_{i}\left(i \gamma^{\mu}\left(D_{\mu}\right)_{i j}-m \delta_{i j}\right) \psi_{j}-\frac{1}{4} G_{\mu \nu}^{a} G_{a}^{\mu \nu}
\eeq
量子色动力学有几个基本特性,其中一个是色荷禁闭,即我们没法观察到非色中性物质的存在。色荷禁闭的特征尺度大致随能标增大而增大,也就是说,在高能标下会发生去禁闭的现象。这一现象可以由熵与稳定的关系表征。在低温下,$S/T^3 \sim O(1)$;而在高温下,$S/T^3 \sim O(N^2)$,这说明高温下存在额外自由度,这就是由去禁闭的色荷贡献的。
\par
在$(\rho,T)$平面上作出QCD物质的存在状态,可以得到所谓的QCD相图如\cref{qcdpd}。
\cpicn{0.9}{qcdpd}{QCD相图示意}
可以看到,为了实现去禁闭,可以通过高温和高压两种手段实现,高温可以通过重离子碰撞实现,也是早期宇宙对应的情形;而高压情形则存在于致密天体中。

\section{Relativistic quantum mechanics and field theory}
\subsection{相对论性量子力学}
回忆非相对论性量子力学基本方程-薛定谔方程的建立:首先我们有色散关系
\beq
E = \frac{\vec{p}^2}{2m}
\eeq
此后,将可观测量分别用算符代替,$E \to i \frac{\partial}{\partial t}$,$\vec{p} \to -i \vec{\nabla}$,两边作用波函数$\psi$,就得到了薛定谔方程
\beq
-\frac{1}{2m} \vec{\nabla}^2 \psi = i \frac{\partial \psi}{\partial t}
\eeq
薛定谔方程显然不是相对论协变的,为了得到一个相对论性的量子力学方程,最简单的考虑是利用相对论色散关系
\beq
E^2 = \vec{p}^2 + m^2
\eeq
利用之前的替换原则就得到
\beq
-\frac{\partial^2 \psi}{\partial t^2} = - \vec{\nabla}^2 \psi + m^2 \psi
\eeq
这称为KG方程(克莱因-高登方程),但它存在一些问题,首先态矢本身不足以确定系统的动力学演化,必须加入其一阶导数;其次,$|\psi|^2$非正定,使得概率诠释失效。
\par
为了解决这些问题,狄拉克提出,让$E = \vec{\alpha} \cdot \vec{p} + \beta m$,且上式作平方时能够恢复到相对论色散关系,即
\beq
E^2 = \alpha^i p_i \alpha^j p_j + (\alpha^i \beta + \beta \alpha^i)p_i + \beta^2 m^2 \equiv p_i p_j \delta^{ij} + m^2
\eeq
于是有
\bea
\{ \alpha^i,\alpha^j \} &= 2\delta^{ij} \\
\{\alpha^i,\beta\} &= 0 \\
\beta^2 &= 1
\eea















\end{document}


