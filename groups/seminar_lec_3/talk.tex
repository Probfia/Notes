\documentclass[CJK]{beamer}
\input{macros.tex}


\newcommand{\field}{\mathscr{F}}

\newcommand{\reals}{\mathbb{R}}
\newcommand{\complexs}{\mathbb{C}}
\newcommand{\ints}{\mathbb{Z}}
%\newcommand{\dim}{\mathrm{dim\ }}
\newcommand{\diag}{\mathrm{diag \ }}
\newcommand{\up}{\uparrow}
\newcommand{\down}{\downarrow}
\newcommand{\su}{\mathfrak{su}}
\newcommand{\so}{\mathfrak{so}}
\newcommand{\tr}{\mathrm{tr\ }}
\newcommand{\card}{\mathrm{card \ }}
\newcommand{\mani}{\mathcal{M}}
\newcommand{\lag}{\mathcal{L}}
\newcommand{\ham}{\mathcal{H}}
\def\secpage#1#2{\begin{frame}\bch\bcenter{\bf \Huge #1} \skipline \tbox{#2}\ecenter\ech\end{frame}}


\newtheorem{thm}{定理}[section]
\newtheorem{axm}{公理}[section]
\newtheorem{dfn}{定义}[section]

%\cpic{<尺寸>}{<文件名>}}用于生成居中的图片。
\newcommand{\cpic}[2]{
\begin{center}
\includegraphics[scale=#1]{#2}
\end{center}
}

%\cpicn{<尺寸>}{<文件名>}{<注释>}用于生成居中且带有注释的图片,其label为图片名。
\newcommand{\cpicn}[3]
{
\begin{figure}[h!]
\cpic{#1}{#2}
\caption{#3\label{#2}}
\end{figure}
}

\title{Group Theory\\ -Spin, Compactness, Symplectic Groups and QFT}
  \author{}
  \date{}


\begin{document}

\begin{frame}
 
\begin{center}
\begin{Large}
\bch
{\bf Group Theory}

{\vskip 0.3in}

Spin, Compactness, Symplectic Groups and QFT

\ech
\end{Large}
\end{center}


\end{frame}

\section{Spin}
\begin{frame}
\frametitle{\bch 电子波函数 \ech}
\bch
电子自旋是1/2意味着波函数$\varPsi$有两个分量,遵循变换:
$$
\varPsi \rightarrow e^{i \vec{\varphi} \cdot \vec{\sigma} / 2} \varPsi
$$
对于一个绕z轴$2\pi$的转动:
$$
\varPsi \rightarrow e^{i(2 \pi) \sigma_{3} / 2} \varPsi=\left(\begin{array}{cc}{e^{i \pi}} & {0} \\ {0} & {e^{-i \pi}}\end{array}\right) \varPsi=-\varPsi
$$
为什么一个$2\pi$的转动会使得波函数异号?
\ech
\end{frame}



\begin{frame}
\frametitle{\bch 自旋进动 \ech}
\bch
\begin{itemize}
\item 考虑静止的满足薛定谔方程的电子,将其放置在不随时变化的外磁场$\vec{B}$下,其哈密顿量为:$H=\mu \vec{B} \cdot \frac{\vec{\sigma}}{2}$,代入薛定谔方程可以得到解:$\varPsi(t)=e^{-i \mu \vec{B} \cdot \frac{\vec{\sigma}}{2} t} \varPsi(0)$
,假设磁场沿着3的方向,上式可化为:
$$
\varPsi(t)=\left(\begin{array}{cc}{e^{-i \frac{\mu B t}{2}}} & {0} \\ {0} & {e^{i \frac{\mu B t}{2}}}\end{array}\right) \varPsi(0)
$$
\item 自旋$\vec{S}=\varPsi^{\dagger} \frac{\vec{\sigma}}{2} \varPsi$,各个自旋分量的变化为:
$$
\begin{array}{l}{\vec{S}_{1}(t)=\cos \mu B t \vec{S}_{1}(0)-\sin \mu B t \vec{S}_{2}(0)} \\ {\vec{S}_{2}(t)=\sin \mu B t \vec{S}_{1}(0)+\cos \mu B t \vec{S}_{2}(0)} \\ {\vec{S}_{3}(t)=\vec{S}_{3}(0)}\end{array}
$$
\item 当$t=2 \pi(\mu B)^{-1}$时,上式有什么变化?
\end{itemize}
\ech
\end{frame}

\begin{frame}
\frametitle{\bch 时间反演 \ech}
\bch
\begin{itemize}
\item 考虑自由粒子的薛定谔方程$i \frac{\partial}{\partial t} \varPsi(t)=H \varPsi(t)$,假设在变换$t \rightarrow t^{\prime}=-t$下,方程依然满足上诉形式,即:
$i \frac{\partial}{\partial t^{\prime}} \varPsi^{\prime}\left(t^{\prime}\right)=H \varPsi^{\prime}\left(t^{\prime}\right)$

\item 假设$\varPsi^{\prime}\left(t^{\prime}\right)=T \varPsi(t)$,T是时间反演算符。代入上式化简并且$HT=TH$,可以得到:
$$
T^{-1}(-i) T=i
$$

\item 假设$T=UK$,且$K\varphi=\varphi^*K$,  $K^2=I$,代入上式化简得到:$U^{-1}iU=i$,成立的条件是$U$是幺正矩阵,所以$T$的反幺正性就体现在$K$上

\item 可以验证对于自旋为0粒子的平面波解$\varPsi(t)=e^{i(\vec{p} \cdot \vec{x}-E t)}$,时间反演后 $\varPsi^{’}(t)=e^{-i(\vec{p} \cdot \vec{x}+E t)}$,空间朝着相反的方向,但是能量依然是正的。此时$T^{2}=UKUK=UU^*K^2=+1$



\end{itemize}
\ech
\end{frame}


\begin{frame}
\frametitle{\bch 自旋的时间反演 \ech}
\bch
\begin{itemize}
\item 时间反演算符作用在自旋$\vec{S}=\varPsi^{\dagger} \frac{\vec{\sigma}}{2} \varPsi$上,$\varPsi \rightarrow \varPsi^{’}=T\varPsi$,并且$T=UK$,作用后得到:$\vec{S'}=  \frac{1}{2} \varPsi^{\dagger} K U^{\dagger} \vec{\sigma} U K \varPsi$。显然时间反演后的自旋应该是$\vec{S’}=-\vec{S}$,因此满足:$K U^{\dagger} \vec{\sigma} U K =-\vec{\sigma} $

\item 能够找到一组解$U=\eta \sigma_2$使得上式成立
$$
K U^{\dagger} \vec{\sigma} U K=\eta^{*} \eta K \sigma_{2} \vec{\sigma} \sigma_{2} K=K\left(\begin{array}{c}{-\sigma_{1}} \\ {+\sigma_{2}} \\ {-\sigma_{3}}\end{array}\right) K=-\vec{\sigma}
$$

\item 将这个解带进$T=UK$ ,得到连续作用两次的时间反演:
$$
T^{2}=\eta \sigma_{2} K \eta \sigma_{2} K=\eta \sigma_{2} \eta^{*} \sigma_{2}^{*} K K=-1
$$
两次时间反演后不会得到原来的状态,这意味着?
\end{itemize}
\ech
\end{frame}

\begin{frame}
\frametitle{\bch Kramer 简并 \ech}
\bch
\begin{itemize}
\item Kramer简并:一个在电场中运动的电子,不管电场有多复杂,每一个能级都是二重简并的。
\item 证明:电场在时间反演下不变,$HT=TH$,$\varPsi$和$T\varPsi$有相同的能量。
假设两者表示同一个态。则$T\varPsi$正比于$\varPsi$,即$T\varPsi=e^{i\alpha}\varPsi$。
$$
T^{2} \varPsi=T(T \varPsi)=T e^{i \alpha} \varPsi=e^{-i \alpha} T \varPsi=e^{-i \alpha} e^{i \alpha} \varPsi=\varPsi
$$
与之前矛盾,所以$\varPsi$和$T\varPsi$代表不同的两个态

\end{itemize}
\ech
\end{frame}




\section{Compactness}
\secpage{群流形的紧致性}{有限群的连续推广}
\begin{frame}
\frametitle{\bch 幺正定理 \ech}
\bch
\begin{itemize}
\item 我们知道,对一个有限群,总有幺正定理成立:
\item {\color{blue} 如果$D$是有限群$G$的一个表示,那么$D$和幺正表示$U$之间总相差一个相似变换
$$
\forall g \in G,\ D(g) = S^{-1} U(g) S
$$
}
\item 其中,幺正表示的定义是
$$
\forall g \in G,\ U^\dagger(g) U(g) = I
$$
\item 对连续群是否有幺正定理成立呢?
\end{itemize}
\ech
\end{frame}

\begin{frame}
\frametitle{\bch 紧致性 \ech}
\bch
\begin{itemize}
\item 如果群流形上紧致的,那么,对连续群依然有幺正定理。
\item {\color{blue} 称连续群$G$紧致,如果对整个群流形$\mani$的测度积分有限
$$\int_\mani d\mu(g) < \infty$$}
\item 粗略来讲就是群流形的体积有限。
\item 因此对$SO(n)$和$SU(n)$都有幺正定理成立,而对洛伦兹群$SO(1,1)$(群元的表达式为$\varLambda (\varphi) = \begin{pmatrix} \cosh \varphi & \sinh \varphi \\ \sinh \varphi & \cosh \varphi \end{pmatrix}$)则不成立幺正定理。
\end{itemize}
\ech
\end{frame}


\section{Symplectic Groups}
\secpage{辛群}{正则变换是保辛变换}

\begin{frame}
\frametitle{\bch 哈密顿力学 \ech}
\bch
请一位同学来默写一下哈密顿正则方程。
\cpic{0.2}{eat}

\ech
\end{frame}

\begin{frame}
\frametitle{\bch 哈密顿力学就是辛流形上的几何 \ech}
\bch
哈密顿正则方程
$$
\dot{q}_a = \frac{\partial H}{\partial p_a} ,\ \dot{p}_b = - \frac{\partial H}{\partial q_b}
$$
引入记号$\xi = (q_1,\cdots,q_n,p_1,\cdots,p_n)^T$,上式可以合写成
$$
\dot{\xi} = J\frac{\partial H}{\partial \xi}
$$
其中$J$是一个$2n\times 2n$矩阵
{\color{blue}
$$
J_{2n} = \begin{pmatrix} 0 & I_n \\ -I_n & 0 \end{pmatrix}
$$
}
\ech
\end{frame}

\begin{frame}
\frametitle{\bch 正则变换 \ech}
\bch
回忆理论力学:正则变换是保持辛流形上柏松括号
$$
[f,g] = \frac{\partial f}{\partial q_a} \frac{\partial g}{\partial p_a} - \frac{\partial f}{\partial p_a} \frac{\partial g}{\partial q_a} = J_{\alpha \beta} \partial_\alpha f \partial_\beta g
$$
不变(从而保持哈密顿正则方程形式不变)的相空间坐标变换$\xi_\alpha \to \zeta_\alpha (\xi_\beta)$。代入上式有
\begin{equation*}
\begin{aligned}
&[ f, g ] = J_{\alpha \beta} \frac{\partial f}{\partial \xi_\alpha} \frac{\partial g}{\partial \xi_\beta} \\
&\to J_{\alpha \beta} \frac{\partial f}{\partial \zeta_\rho} \frac{\partial \zeta_\rho}{\partial \xi_\alpha} \frac{\partial g}{\partial \zeta_\sigma} \frac{\partial \zeta_\sigma}{\partial \xi_\beta} \\
&\equiv J_{\rho \sigma} \frac{\partial f}{\partial \zeta_\rho} \frac{\partial g}{\partial \zeta_\sigma}
\end{aligned}
\end{equation*}
\ech
\end{frame}

\begin{frame}
\frametitle{\bch 辛群 \ech}
\bch
上式相当于要求变换矩阵$R_{\beta \sigma} = \frac{\partial \zeta_\sigma}{\partial \xi_\beta}$满足
$$
R_{\alpha \rho} J_{\alpha \beta} R_{\beta \sigma} = J_{\rho \sigma}
$$
或
$$
R^T J R = J
$$
\begin{itemize}
\item 回忆一下:$O(n)$是保持欧式度规$\delta_{ij}$不变的矩阵集合;$O(1,n-1)$是保持闵氏度规$\eta_{\mu \nu}$不变的矩阵集合。
\item 相似地,{\color{blue} 所有保持辛流形上“度规”$J_{2n}$不变的矩阵集合构成辛群$Sp(2n)$}。
\end{itemize}

\ech
\end{frame}

\begin{frame}
\frametitle{\bch 辛几何 \ech}
\bch

\begin{itemize}
\item 所以哈密顿力学就是辛流形上的几何。
\item 辛几何的研究很困难,因为一般人吃不了那么辣。
\end{itemize}
\cpic{0.25}{hot}
\ech
\end{frame}




\begin{frame}
\frametitle{\bch 实辛群和复辛群 \ech}
\bch
\begin{itemize}
\item 满足条件$R^T J R = J$的实矩阵$R$构成群$Sp(2n,\reals)$。
\item 满足条件$C^T J C = J$的复矩阵$C$构成群$Sp(2n,\complexs)$(注意是$C^T$不是$C^\dagger$)。
\item 上面的定义式自动包含了条件$\det R = 1$($\det C = 1$)。证明思路:如果$z$是$R$的本征值,那么$1/z$也是。
\end{itemize}
\ech
\end{frame}

\begin{frame}
\frametitle{\bch 幺正辛群 \ech}
\bch
幺正的辛群称为幺正辛群$USp(2n)$。显然
$$
USp(2n) = U(2n) \cap Sp(2n,\complexs) = SU(2n) \cap Sp(2n,\complexs)
$$
\ech
\end{frame}


\begin{frame}
\frametitle{\bch 幺正辛群的代数 \ech}
\bch
\begin{itemize}
\item
按一般的套路:设$U = I + i\epsilon H \in USp(2n)$。证明:
$$
H = \begin{pmatrix} P & W^* \\ W & -P^T\end{pmatrix}
$$
其中$P^\dagger = P$为厄米阵,$W = W^T$为(复)对称阵。
\item
厄米阵的独立实分量为$n^2$,复对称阵的独立分量为$n(n+1)$,于是
$$
\dim USp(2n) = n(2n+1)
$$
\item 特别地我们注意到
$$\dim USp(2) = 3 = \dim SU(2) = \dim SO(3)$$ $$\dim USp(4) = 10 = \dim SO(5)$$等。
\end{itemize}
\ech
\end{frame}

\begin{frame}
\frametitle{\bch 用泡利矩阵表示幺正辛代数 \ech}
\bch
\begin{itemize}
\item
设我们有一个$n$阶实反称阵$A$和3个$n$阶实对称阵$S_1,S_2,S_3$,$USp(2n)$的生成元就可以写成
$$
H = \begin{pmatrix} iA + S_3 & S_1 - iS_2 \\ S_1 + iS_2 & iA - S_3 \end{pmatrix}
$$
\item
利用矩阵直积记号,上式也可以写成
$$
H = iA \otimes I + S_a \otimes \sigma_a
$$
\end{itemize}
\ech
\end{frame}

\begin{frame}
\frametitle{\bch $SU(2) \simeq USp(2)$ \ech}
\bch
\begin{itemize}
\item
刚才已经知道$SU(2)$和$USp(2)$是代数同维的。
\item
在$n=1$时,$A \equiv 0$,$S_a$就是一个数$s_a$,于是
$$
H = s_a \sigma_a
$$
\item
于是任意的$USp(2)$群元就是$SU(2)$群元
$$
U = e^{i s_a \frac{\sigma_a}{2}}
$$
\end{itemize}
\ech
\end{frame}

\section{QFT}
\secpage{量子场论速成}{$$\hat \phi \sim \hat a + \hat a^\dagger$$}

\begin{frame}
\frametitle{\bch 课程中止 \ech}
\bch
因为大家没学过量子场论,我们的课程可能需要就此中止。
\cpic{0.2}{sad}
\ech
\end{frame}

\begin{frame}
\frametitle{\bch 为了课程的延续 \ech}
\bch
为了让课程继续,Zee 一鸿决定带领大家一节速成QFT。
\cpic{0.2}{happy}
\ech
\end{frame}

\begin{frame}
\frametitle{\bch 一维谐振子的正则量子化 \ech}
\bch
\begin{itemize}
\item
单位质量的谐振子哈密顿量为
$$
H = \frac{1}{2} p^2 + \frac{1}{2} \omega^2 q^2
$$
\item
施加正则量子化条件$[\hat q, \hat p] = i$,定义产生湮灭算符
$$
\hat{a}^\dagger = \frac{1}{\sqrt{2\omega}} ( \omega \hat q - i \hat p),\ \hat{a} = \frac{1}{\sqrt{2\omega}} ( \omega \hat q + i \hat p)
$$
正则量子化条件等价为$[\hat a,\hat{a}^\dagger] = 1$。
\item
坐标算符用产生湮灭算符表示为
$$
\hat q = \frac{1}{\sqrt{2\omega}} ( \hat a + \hat{a}^\dagger)
$$
略去零点能,哈密顿算符用产生湮灭算符表示为
$$
\hat H = \omega \hat{a}^\dagger \hat{a}
$$
\end{itemize}
\ech
\end{frame}

\begin{frame}
\frametitle{\bch 海森堡绘景 \ech}
\bch
\begin{itemize}
\item
我们一般的非相对论量子力学都是在薛定谔绘景下描述的,这个绘景可以简单概括成{\color{blue} 算符不变态矢变},态矢的含时演化方程就是薛定谔方程
$$
i \frac{\partial}{\partial t} |\varPsi \rangle=\hat H | \varPsi \rangle
$$
\item
我们可以作幺正变换$|\varPsi \rangle \to |\varPsi, t\rangle = e^{i\hat H t} |\varPsi\rangle$,算符也要相应变换为$\hat O \to \hat O(t) = e^{-i \hat H t} \hat O e^{i \hat H t}$。这样变换后的绘景称为海森堡绘景,它可以简单概括成{\color{blue} 态矢不变算符变}。
\item
代人薛定谔方程验证:
$$
\frac{\partial }{\partial t} |\varPsi, t\rangle = 0 
$$
$$
\frac{\partial }{\partial t} \hat O(t) = i [\hat H,\hat O]
$$
\end{itemize}
\ech
\end{frame}

\begin{frame}
\frametitle{\bch 海森堡绘景下的谐振子 \ech}
\bch
\begin{itemize}
\item
利用我们刚才得到的$\hat H = \omega \hat a^\dagger \hat a$,变换到海森堡绘景,利用运动方程$\frac{\partial }{\partial t} \hat O(t) = i [\hat H,\hat O]$得到
$$
\hat a(t) = \hat a e^{-i\omega t},\ \hat a^\dagger = \hat a^\dagger (t) e^{i\omega t}
$$
\item
于是坐标算符的含时演化为
$$
\hat q(t) = \frac{1}{\sqrt{2\omega}} (\hat a e^{-i\omega t} + \hat a^\dagger e^{i \omega t})
$$
其中$\hat a \equiv \hat a(t=0)$。
\end{itemize}
\ech
\end{frame}

\begin{frame}
\frametitle{\bch 从经典力学到经典场论 \ech}
\bch
\begin{itemize}
\item
假设我们有许多个用弹簧相连的单位质量振子,以$q_a(t)$表示每个振子的位移,体系的拉格朗日函数就是
$$
L = \sum_a \left( \frac{1}{2} \dot{q}_a^2 - \frac{1}{2}\omega^2 (q_{a+1} - q_a)^2\right)
$$
\item
如果这些振子变得稠密,以至于整个体系变成连续的(想象:从一个一维无限振子链过渡到一根橡皮筋),那么,指标$a$应该用坐标$x$替换。其他东西也应该按规则替换
$$
\sum_a \to \int dx
$$
$$
\dot q \to \frac{\partial q}{\partial t}
$$
$$
q_{a+1} - q_a \to \frac{\partial q}{\partial x}
$$
\end{itemize}
\ech
\end{frame}

\begin{frame}
\frametitle{\bch 推广到高维 \ech}
\bch
\begin{itemize}
\item
于是一维橡皮筋的拉格朗日函数就是
$$
L =  \int dx\frac{1}{2} \left[ \left( \frac{\partial q}{\partial t}\right)^2 - \omega^2 \left( \frac{\partial q}{\partial x} \right)^2\right]
$$
积分号内的东西称为拉氏密度$\lag$。
\item
把$q$记作$\phi$,令$\omega = 1$,重新引入外势$\frac{1}{2}m^2 \phi^2$,就有
$$
\lag = \frac{1}{2} \left( \frac{\partial \phi}{\partial t}\right)^2 - \frac{1}{2} ( \vec \nabla \phi )^2 - \frac{1}{2}m^2 \phi^2
$$
\item
上式可以写成更相对论协变的形式
$$
\lag = \frac{1}{2} \partial_\mu \phi \partial^\mu \phi - \frac{1}{2}m^2 \phi^2
$$
\end{itemize}
\ech
\end{frame}

\begin{frame}
\frametitle{\bch 经典场的运动方程和哈密顿量 \ech}
\bch
\begin{itemize}
\item
作用量原理给出经典场的欧拉-拉格朗日方程
$$
\partial_\mu \frac{\partial \lag}{\partial (\partial_\mu \phi)} - \frac{\partial \lag}{\partial \phi}
$$
\item
场的广义动量定义为
$$
\pi = \frac{\partial \lag}{\partial \dot \phi}
$$
\item
哈密顿密度
$$
\ham = \pi \dot \phi - \lag
$$
哈密顿量
$$
H = \int d^3 \vec{x} \ \ham
$$

\end{itemize}
\ech
\end{frame}

\begin{frame}
\frametitle{\bch KG场的运动方程和哈密顿量 \ech}
\bch
请大家计算一下$\lag = \frac{1}{2} \partial_\mu \phi \partial^\mu \phi - \frac{1}{2}m^2 \phi^2$给出的运动方程和哈密顿密度。
\cpic{0.2}{eat}
\ech
\end{frame}


\begin{frame}
\frametitle{\bch 场的傅立叶分解 \ech}
\bch
\begin{itemize}
\item 对运动方程$(\partial^2 + m^2)\phi = 0$两边傅立叶变换得到色散关系$\omega_{\vec k} = \sqrt{ \vec k^2 + m^2} $。
\item 哈密顿量$H = \frac{1}{2}\int d^3 \vec x \ \left( \pi^2 + (\vec \nabla \phi)^2 + m^2 \phi^2 \right)$可以利用傅立叶变换写成
$$
H = \frac{1}{2} \int \frac{d^3 \vec k }{(2\pi)^3} \ \left( |\pi_{\vec k} |^2 + \omega_{\vec{k}}^2 | \phi_{\vec k} |^2\right)
$$
\item 哈密顿量可以看成无穷个谐振子的和
$$
H_{\vec k} = \frac{1}{2} \left( |\pi_{\vec k} |^2 + \omega_{\vec{k}}^2 | \phi_{\vec k} |^2\right)
$$
\end{itemize}
\ech
\end{frame}

\begin{frame}
\frametitle{\bch 从经典场论到量子场论 \ech}
\bch
\begin{itemize}
\item 上面做的工作都还是在经典场论的框架下进行的。
\item 但我们已经把场分解成了无数个谐振子的和,而对单个以$\vec k$标记的谐振子,我们都可以定义正则对易关系,引入产生湮灭算符$\hat a_{\vec k}$和$\hat{a}_{\vec k}^\dagger$。这就完成了场的量子化。
\item 总的哈密顿量用产生湮灭算符写出
$$
\hat H = \int \frac{d^3 \vec k}{(2\pi)^3} \omega_{\vec k} \hat{a}^\dagger_{\vec k} \hat{a}_{\vec k}
$$
\item 傅立叶变换回去,在海森堡绘景下得到场算符的含时演化
$$
\hat \phi (\vec x,t) = \int \frac{d^3 \vec k}{(2\pi)^3} \ \frac{1}{\sqrt{2\omega_{\vec k}}}\left( \hat{a}_{\vec k} e^{i\omega_{\vec k} t - i \vec k \cdot \vec x} + \hat{a}^\dagger_{\vec k} e^{-i\omega_{\vec k} t + i \vec k \cdot \vec x} \right)
$$
\end{itemize}
\ech
\end{frame}



\begin{frame}
\frametitle{\bch 从量子场到粒子 \ech}
\bch
\begin{itemize}
\item 我们定义了一堆产生湮灭算符$\hat{a}_{\vec k}^\dagger$和$\hat{a}_{\vec k}$。\item 仿照量子谐振子,真空定义为使得所有模式$\vec k$都有$\hat a_{\vec k} | 0 \rangle = 0$的态。
\item 动量$\vec p$粒子态定义为
$$
| \vec p \rangle = \hat{a}_{\vec p}^\dagger |0\rangle
$$
\end{itemize}
\ech
\end{frame}

\begin{frame}
\frametitle{\bch 真空涨落 \ech}
\bch
\begin{itemize}
\item 场算符可以分解成产生算符和湮灭算符的和
$$
\hat \phi \sim \hat a + \hat a^\dagger
$$
\item 我们写出的拉氏量$\lag \sim \phi^2$,也就是$\lag$包含$\sim \hat a \hat a^\dagger + \hat a^\dagger \hat a$的项。
\item 换句话说,KG场的真空会自己涨落出粒子然后经过一段时间后又湮灭掉。
\end{itemize}
\cpic{0.13}{scare}
\ech
\end{frame}

\begin{frame}
\frametitle{\bch 构造对称操作下不变的拉氏量 \ech}
\bch
\begin{itemize}
\item 如果拉氏量在群元$g\in G$的对称操作下不变,则称这个拉氏量的对称群是$G$。
\item 很多时候我们反过来,先猜一个对称群,然后再构造在这个对称群下不变的拉氏量。
\item 比如我们想考虑$U(1)$对称的拉氏量,最简单的可以引入一个复标量场$\varPsi$,$U(1)$变化是$\varPsi \to e^{i\alpha} \varPsi$。
\item 于是可以想到,最简单的拉氏量大概长这样
$$
\lag \sim \varPsi^\dagger \varPsi
$$
\end{itemize}
\ech
\end{frame}

\begin{frame}
\frametitle{\bch 复自由标量场 \ech}
\bch
\begin{itemize}
\item 我们把复数引入进来相当于多引入自由度。因此我们需要两组产生湮灭算符。
\item 自洽的定义方法是$\varPsi \sim \hat a + \hat b^\dagger$;$\varPsi^\dagger \sim \hat a^\dagger + \hat b$。
\item 拉氏量$\lag \sim \varPsi^\dagger \varPsi \sim \hat a^\dagger \hat a + \hat a^\dagger \hat b^\dagger + \hat b \hat b^\dagger + \hat b\hat a$
\item 复标量场真空包含的物理过程:
\begin{itemize}
\item 
$\hat a^\dagger \hat a $:湮灭一个$a$粒子再产生(和反过来);
\item
$ \hat a^\dagger \hat b^\dagger$:产生一个$a,b$粒子对;
\item
$ \hat b \hat b^\dagger$:产生一个$b$粒子再湮灭(和反过来);
\item
$\hat b\hat a$:湮灭一个$a,b$粒子对。
\end{itemize}
\end{itemize}
\ech
\end{frame}

\begin{frame}
\frametitle{\bch $SU(N)$场论 \ech}
\bch
\begin{itemize}
\item $SU(N)$对称的拉氏量:最简单的构造方法是$\lag_0 \sim \varPsi^i \varPsi_i$,真空物理过程和$U(1)$没有太大差异:这是自由场。
\item 次简单的定义方法:$\lag_1 \sim \varPsi_i \varPsi_j \eta^{ij}$,这是在$SU(N)$自由场下引入的可能的相互作用项。
\item 同样的,$\varPsi \sim \hat a + \hat a^\dagger$;$\eta \sim \hat b + \hat b^\dagger$,$\lag \sim \hat a^\dagger \hat a^\dagger \hat b + \cdots$:对应的物理过程为湮灭一个$b$粒子并产生两个$a$粒子等等。
\end{itemize}
\ech
\end{frame}




\end{document}



\end{document}