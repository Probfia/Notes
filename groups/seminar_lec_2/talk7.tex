\documentclass[CJK]{beamer}
\usepackage{CJKutf8}
\usepackage{beamerthemesplit}
\usetheme{Malmoe}
\useoutertheme[footline=authortitle]{miniframes}
\usepackage{amsmath}
\usepackage{amssymb}
\usepackage{graphicx}
\usepackage{eufrak}
\usepackage{color}
\usepackage{slashed}
\usepackage{simplewick}
\usepackage{tikz}
\usepackage{tcolorbox}
\graphicspath{{../figures/}}
%%figures
\def\lfig#1#2{\includegraphics[width=#1 in]{#2}}
\def\addfig#1#2{\begin{center}\includegraphics[width=#1 in]{#2}\end{center}}
\def\wulian{\includegraphics[width=0.18in]{emoji_wulian.jpg}}
\def\bigwulian{\includegraphics[width=0.35in]{emoji_wulian.jpg}}
\def\bye{\includegraphics[width=0.18in]{emoji_bye.jpg}}
\def\bigbye{\includegraphics[width=0.35in]{emoji_bye.jpg}}
\def\huaixiao{\includegraphics[width=0.18in]{emoji_huaixiao.jpg}}
\def\bighuaixiao{\includegraphics[width=0.35in]{emoji_huaixiao.jpg}}
\def\jianxiao{\includegraphics[width=0.18in]{emoji_jianxiao.jpg}}
\def\bigjianxiao{\includegraphics[width=0.35in]{emoji_jianxiao.jpg}}
%% colors
\def\blacktext#1{{\color{black}#1}}
\def\bluetext#1{{\color{blue}#1}}
\def\redtext#1{{\color{red}#1}}
\def\darkbluetext#1{{\color[rgb]{0,0.2,0.6}#1}}
\def\skybluetext#1{{\color[rgb]{0.2,0.7,1.}#1}}
\def\cyantext#1{{\color[rgb]{0.,0.5,0.5}#1}}
\def\greentext#1{{\color[rgb]{0,0.7,0.1}#1}}
\def\darkgray{\color[rgb]{0.2,0.2,0.2}}
\def\lightgray{\color[rgb]{0.6,0.6,0.6}}
\def\gray{\color[rgb]{0.4,0.4,0.4}}
\def\blue{\color{blue}}
\def\red{\color{red}}
\def\green{\color{green}}
\def\darkgreen{\color[rgb]{0,0.4,0.1}}
\def\darkblue{\color[rgb]{0,0.2,0.6}}
\def\skyblue{\color[rgb]{0.2,0.7,1.}}
%%control
\def\be{\begin{equation}}
\def\ee{\nonumber\end{equation}}
\def\bea{\begin{eqnarray}}
\def\eea{\nonumber\end{eqnarray}}
\def\bch{\begin{CJK}{UTF8}{gbsn}}
\def\ech{\end{CJK}}
\def\bitem{\begin{itemize}}
\def\eitem{\end{itemize}}
\def\bcenter{\begin{center}}
\def\ecenter{\end{center}}
\def\bex{\begin{minipage}{0.2\textwidth}\includegraphics[width=0.6in]{jugelizi.png}\end{minipage}\begin{minipage}{0.76\textwidth}}
\def\eex{\end{minipage}}
\def\chtitle#1{\frametitle{\bch#1\ech}}
\def\bmat#1{\left(\begin{array}{#1}}
\def\emat{\end{array}\right)}
\def\bcase#1{\left\{\begin{array}{#1}}
\def\ecase{\end{array}\right.}
\def\bmini#1{\begin{minipage}{#1\textwidth}}
\def\emini{\end{minipage}}
\def\tbox#1{\begin{tcolorbox}#1\end{tcolorbox}}
\def\pfrac#1#2#3{\left(\frac{\partial #1}{\partial #2}\right)_{#3}}
%%symbols
\def\bropt{\,(\ \ \ )}
\def\sone{$\star$}
\def\stwo{$\star\star$}
\def\sthree{$\star\star\star$}
\def\sfour{$\star\star\star\star$}
\def\sfive{$\star\star\star\star\star$}
\def\rint{{\int_\leftrightarrow}}
\def\roint{{\oint_\leftrightarrow}}
\def\stdHf{{\textit{\r H}_f}}
\def\deltaH{{\Delta \textit{\r H}}}
\def\ii{{\dot{\imath}}}
\def\skipline{{\vskip0.1in}}
\def\skiplines{{\vskip0.2in}}
\def\lagr{{\mathcal{L}}}
\def\hamil{{\mathcal{H}}}
\def\vecv{{\mathbf{v}}}
\def\vecx{{\mathbf{x}}}
\def\vecy{{\mathbf{y}}}
\def\veck{{\mathbf{k}}}
\def\vecp{{\mathbf{p}}}
\def\vecn{{\mathbf{n}}}
\def\vecA{{\mathbf{A}}}
\def\vecP{{\mathbf{P}}}
\def\vecsigma{{\mathbf{\sigma}}}
\def\hatJn{{\hat{J_\vecn}}}
\def\hatJx{{\hat{J_x}}}
\def\hatJy{{\hat{J_y}}}
\def\hatJz{{\hat{J_z}}}
\def\hatj#1{\hat{J_{#1}}}
\def\hatphi{{\hat{\phi}}}
\def\hatq{{\hat{q}}}
\def\hatpi{{\hat{\pi}}}
\def\vel{\upsilon}
\def\Dint{{\mathcal{D}}}
\def\adag{{\hat{a}^\dagger}}
\def\bdag{{\hat{b}^\dagger}}
\def\cdag{{\hat{c}^\dagger}}
\def\ddag{{\hat{d}^\dagger}}
\def\hata{{\hat{a}}}
\def\hatb{{\hat{b}}}
\def\hatc{{\hat{c}}}
\def\hatd{{\hat{d}}}
\def\hatN{{\hat{N}}}
\def\hatH{{\hat{H}}}
\def\hatp{{\hat{p}}}
\def\Fup{{F^{\mu\nu}}}
\def\Fdown{{F_{\mu\nu}}}
\def\newl{\nonumber \\}
\def\vece{\mathrm{e}}
\def\calM{{\mathcal{M}}}
\def\calT{{\mathcal{T}}}
\def\calR{{\mathcal{R}}}
\def\barpsi{\bar{\psi}}
\def\baru{\bar{u}}
\def\barv{\bar{\upsilon}}
\def\qeq{\stackrel{?}{=}}
\def\torder#1{\mathcal{T}\left(#1\right)}
\def\rorder#1{\mathcal{R}\left(#1\right)}
\def\contr#1#2{\contraction{}{#1}{}{#2}#1#2}
\def\trof#1{\mathrm{Tr}\left(#1\right)}
\def\trace{\mathrm{Tr}}
\def\comm#1{\ \ \ \left(\mathrm{used}\ #1\right)}
\def\tcomm#1{\ \ \ (\text{#1})}
\def\slp{\slashed{p}}
\def\slk{\slashed{k}}
\def\calp{{\mathfrak{p}}}
\def\veccalp{\mathbf{\mathfrak{p}}}
\def\Tthree{T_{\tiny \textcircled{3}}}
\def\pthree{p_{\tiny \textcircled{3}}}
\def\dbar{{\,\mathchar'26\mkern-12mu d}}
\def\erf{\mathrm{erf}}
\def\const{\mathrm{constant}}
\def\pheat{\pfrac p{\ln T}V}
\def\vheat{\pfrac V{\ln T}p}
%%units
\def\fdeg{{^\circ \mathrm{F}}}
\def\cdeg{^\circ \mathrm{C}}
\def\atm{\,\mathrm{atm}}
\def\angstrom{\,\text{\AA}}
\def\SIL{\,\mathrm{L}}
\def\SIkm{\,\mathrm{km}}
\def\SIyr{\,\mathrm{yr}}
\def\SIGyr{\,\mathrm{Gyr}}
\def\SIV{\,\mathrm{V}}
\def\SImV{\,\mathrm{mV}}
\def\SIeV{\,\mathrm{eV}}
\def\SIkeV{\,\mathrm{keV}}
\def\SIMeV{\,\mathrm{MeV}}
\def\SIGeV{\,\mathrm{GeV}}
\def\SIcal{\,\mathrm{cal}}
\def\SIkcal{\,\mathrm{kcal}}
\def\SImol{\,\mathrm{mol}}
\def\SIN{\,\mathrm{N}}
\def\SIHz{\,\mathrm{Hz}}
\def\SIm{\,\mathrm{m}}
\def\SIcm{\,\mathrm{cm}}
\def\SIfm{\,\mathrm{fm}}
\def\SImm{\,\mathrm{mm}}
\def\SInm{\,\mathrm{nm}}
\def\SImum{\,\mathrm{\mu m}}
\def\SIJ{\,\mathrm{J}}
\def\SIW{\,\mathrm{W}}
\def\SIkJ{\,\mathrm{kJ}}
\def\SIs{\,\mathrm{s}}
\def\SIkg{\,\mathrm{kg}}
\def\SIg{\,\mathrm{g}}
\def\SIK{\,\mathrm{K}}
\def\SImmHg{\,\mathrm{mmHg}}
\def\SIPa{\,\mathrm{Pa}}



\newcommand{\field}{\mathscr{F}}

\newcommand{\reals}{\mathbb{R}}
\newcommand{\complexs}{\mathbb{C}}
\newcommand{\ints}{\mathbb{Z}}
%\newcommand{\dim}{\mathrm{dim\ }}
\newcommand{\diag}{\mathrm{diag \ }}
\newcommand{\up}{\uparrow}
\newcommand{\down}{\downarrow}
\newcommand{\su}{\mathfrak{su}}
\newcommand{\so}{\mathfrak{so}}
\newcommand{\tr}{\mathrm{tr\ }}
\newcommand{\card}{\mathrm{card \ }}

\newtheorem{thm}{定理}[section]
\newtheorem{axm}{公理}[section]
\newtheorem{dfn}{定义}[section]

%\cpic{<尺寸>}{<文件名>}}用于生成居中的图片。
\newcommand{\cpic}[2]{
\begin{center}
\includegraphics[scale=#1]{#2}
\end{center}
}

%\cpicn{<尺寸>}{<文件名>}{<注释>}用于生成居中且带有注释的图片,其label为图片名。
\newcommand{\cpicn}[3]
{
\begin{figure}[h!]
\cpic{#1}{#2}
\caption{#3\label{#2}}
\end{figure}
}

\title{Group Theory\\ Talk 7 - Group Theory in a Quantum World}
  \author{}
  \date{}


\begin{document}

\begin{frame}
 
\begin{center}
\begin{Large}
\bch
{\bf Group Theory}

{\vskip 0.3in}

Talk 7 - Group Theory in a Quantum World

\ech
\end{Large}
\end{center}

\vskip 0.2in
\begin{center}
Han Gao
\vskip 0.1in
gaoh26@mail2.sysu.edu.cn
\vskip 0.2in
{\tiny \url{https://github.com/Probfia/Notes/blob/master/groups/seminar_lec_2/talk.pdf} }\\
\end{center}


\end{frame}

\section{Review and Overview}
\begin{frame}
\frametitle{\bch 这章特别水 \ech}
\bch
所以秉着“人人都拿公益时”的理念,莫宗霖同学给我们先对前一章作一个回顾:
\begin{itemize}
\item 群的表示:定义,可约与不可约;
\item 方特征标表等式$N(C) = N(R)$,$\sum_c n_c \chi^* (c) \chi(c) = N(G)$;
\item 实表示与复表示;
\item \dots
\end{itemize}
\ech
\end{frame}

\begin{frame}
\frametitle{\bch 大览 \ech}
\bch
\begin{itemize}
\item 群表示论与量子力学
\item 物理中的对称性:作用量
\end{itemize}
\ech
\end{frame}



\section{Quantum Mechanics and Representation}
\begin{frame}

\frametitle{\bch 薛定谔方程 \ech}
\bch
量子力学薛定谔方程
$$
H \psi = E \psi
$$
哈密顿量$H$是一个线性算符,$\psi$是一个能量为$E$的本征态。
\ech
\end{frame}

\begin{frame}
\frametitle{\bch 简并 \ech}
\bch
单个能量$E$可能对应一系列本征态$\psi^\alpha$。
$$
H \psi^\alpha = E \psi^\alpha ,\ \alpha = 1,2,\cdots , d
$$
这种情况称为$d$重简并。
\par
你能举一些简并的例子么?
\ech
\end{frame}

\begin{frame}
\frametitle{\bch 简并的来源 \ech}
\bch
从刚才举出来的例子中总结一下,量子力学中,简并的来源是什么?
\par
Spoiler: Symmetry.
\ech
\end{frame}

\begin{frame}
\frametitle{\bch 简并的来源 \ech}
\bch
设有一系列幺正变换$\hat T$,称哈密顿量在变换$\hat T$下不变,如果
$$
\hat{T}^\dagger H \hat T = H
$$
证明:
\begin{itemize}
\item 上面的等式等价于$[H,T] = 0$;
\item 所有的$T$构成群;
\item 若$\psi$是$H$能量为$E$的本征态,则$\hat T \psi$也是。
\end{itemize}

\ech
\end{frame}

\begin{frame}
\frametitle{\bch 糟糕的记号 \ech}
\bch
上面最后一个记号其实有问题:$T$其实并不能直接作用在态上,因为$T$是对物理空间里的矢量的某种操作(e.g. $SO(3)$旋转;镜面反射$x \to -x$等),而不是希尔伯特空间里的算符。
\par
不过我们还是暂且这样写。它的意思是,如果我们把空间中的矢量按一定规则变化,对应的态按照这个规则会怎么变。

\ech
\end{frame}

\begin{frame}
\frametitle{\bch 简并态构成子空间 \ech}
\bch
假设我们已经找到了$d$和能量$E$本征态$\psi_a,\ a = 1,2,\cdots,d$,并且它们是正交归一的$\langle \psi_a | \psi_b \rangle = \delta_{ab}$。
\par
线性组合
$$
\psi = c^a \psi_a
$$
也是一个能量为$E$的本征态。因此说简并的本征态生活在由$\psi_a$张成的线性空间$\langle \psi_a \rangle$中。
\ech
\end{frame}


\begin{frame}
\frametitle{\bch 简并态构成子空间 \ech}
\bch
因为$\hat T \psi_a \in \langle \psi_a \rangle$,可以按基展开,形式上写成
$$
\hat T \psi_a = D(T)_a^b \psi_b
$$
其中$D(T)$在给定操作$T$时,就是一个普通的系数(或者,$d$阶方阵)。
\par
$D$建立了空间对称变换和希尔伯特空间中态矢变换间的对应关系。
\ech
\end{frame}

\begin{frame}
\frametitle{\bch 你看出来我在说什么了么 \ech}
\bch
$T$是空间对称群的元素,$D(T)$是(希尔伯特空间中的)矩阵。
\par
这不就是群表示论么?
\par
证明:$D(T_1 T_2) = D(T_1) D(T_2)$,等式右边理解为矩阵乘法。
\ech
\end{frame}

\begin{frame}
\frametitle{\bch 简并度告诉我们对称性的信息 \ech}
\bch
$D$就是对称群$G = \{ T \}$的一个$d$维不可约表示。
\par
因此,假如我们从实验中得到了一个$d$重简并度,我们知道带来这个简并度的对称群至少有一个$d$维不可约表示。
\bea
d &=& \text{简并度(来源于实验) }\\ &=&\text{ 对称群的一个不可约表示维数(理论的追求)}
\eea
\ech
\end{frame}

\begin{frame}
\frametitle{\bch 总结一下 \ech}
\bch
\begin{itemize}
\item 物理空间中的对称群$G$中的元素$T$可以映射到一个希尔伯特空间中的线性算符$D(T)$;
\item 在这个希尔伯特空间中可以找一组正交基;
\item 哈密顿算符在这种正交基下的矩阵表示是一个分块纯量阵$H = \diag( E_{(1)} I_{d(1)},E_{(2)} I_{d(2)},\cdots,E_{(s)} I_{d(s)})$,自然地分出了简并态;
\item 每一个分块对应群$G$的一个不可约表示$D^{(r)}$,哈密顿算符在子空间中是纯量阵,与舒尔引理对应。
\end{itemize}

\ech
\end{frame}


\begin{frame}
\frametitle{\bch 回到现实 \ech}
\bch

\cpic{0.4}{think}
给定一个量子体系,只给你一个测量能量的装置,你能得到关于一个量子体系简并度的任何信息么?
\ech
\end{frame}

\begin{frame}
\frametitle{\bch 不可能 \ech}
\bch
当然不行。
\par
比如说,对氢原子,假如你不去测量$1s$电子的自旋,你永远不会知道基态其实是2重简并的。
\par
因此,需要更多的信息才能确定简并度。

\ech
\end{frame}

\begin{frame}
\frametitle{\bch 实验很重要 \ech}
\bch
只有做其他实验才能知道简并度的其他信息。
\begin{itemize}
\item 8种重子具有几乎相同的质量;
\item 怎么区分8种重子?电荷数和奇异数;
\item 质量=能量,$d=8$;
\item 强相互作用至少有一个8维不可约表示;
\item 8个盖尔曼矩阵,对应8种传递色相互作用的胶子。
\end{itemize}
\cpic{0.2}{gmm}
\ech
\end{frame}


\begin{frame}
\frametitle{\bch 实验真的很重要吗? \ech}
\bch
\begin{itemize}
\item
理论爱好者:不需要实验,我就坐在那里想,我就知道这个世界(至少)是$SO(3)$的;我还可以拿这个算出很多符合实验的结果;然后我还可以把对称群推到$SO(10),SU(20),SU(\infty) \cdots$。
\item
但这真的是完全想出来的么?如果我们去问一个刚出生的婴儿这个世界是不是$SO(3)$对称的,他会怎么回答?你又怎么知道把对称群推大就能够奏效呢?
\item
哲学问题:我们真的有与生俱来的知识么?人是否是自身经验的总和?
\end{itemize}
\ech
\end{frame}

\begin{frame}
\frametitle{\bch  \ech}
\bch
STOP
\ech
\end{frame}


\begin{frame}
\frametitle{\bch 宇称Parity对称性\ech}
\bch
一维粒子
$$
H = -\frac{1}{2m} \frac{d^2}{dx^2} + V(x)
$$
如果$V(x) = V(-x)$,哈密顿量在宇称变换$r : x \to -x$下不变。物理空间的一个对称群$Z_2 = \{I ,r\}$。
\ech
\end{frame}

\begin{frame}
\frametitle{\bch 宇称对称性的后果\ech}
\bch
我们知道$Z_2$有两个1维不可约表示
\begin{itemize}
\item 平凡表示:$I \to 1,\ r \to 1$,使得态(波函数)必须满足$\psi(-x) \equiv \hat r \psi(x) = D(r) \psi(x) = \psi(x)$;
\item 自身表示:$I \to 1,\ r \to -1$,使得态(波函数)必须满足$\psi(-x) \equiv \hat r \psi(x) = D(r) \psi(x) = -\psi(x)$。
\end{itemize}
群表示论直接告诉我们,当空间具有宇称对称性时,波函数只能是严格奇函数或偶函数。(qm作业第2题)

\ech
\end{frame}


\begin{frame}
\frametitle{\bch 周期平移对称性\ech}
\bch
一维粒子
$$
H = -\frac{1}{2m} \frac{d^2}{dx^2} + V(x)
$$
如果$V(x) = V(x+a)$,哈密顿量在平移变换$T : x \to x+a$下不变。物理空间的一个对称群$G = \{ \cdots T^{-1} , I, T , T^2 \cdots\}$。
\ech
\end{frame}


\begin{frame}
\frametitle{\bch 阿贝尔群\ech}
\bch
虽然对称群$G$很大,甚至是无限的;但显然$G$是一个阿贝尔群,而阿贝尔群只能有一维表示。
\par
(复习:
\par
方特征标表等式$N(C) = N(R)$和$\sum_r d_r^2 = N(G)$;阿贝尔群每个元素各成一类,$N(C) = N(G)$,只能所有$d_r = 1$。)
\\  \ \\
\par
于是可以设$D(T) = \xi$为一个任意复数,则有$\psi(x+a) \equiv \hat T \psi(x) = D(T) \psi(x) = \xi \psi(x)$。
\ech
\end{frame}

\begin{frame}
\frametitle{\bch 归一化条件 \ech}
\bch
$$1 = \int |\psi(x+a) |^2 dx  |\xi|^2 \int |\psi(x)|^2 dx = |\xi|^2$$
于是只能有$\xi \in U(1)$。将其写成$\xi = e^{ika}$,于是得到Bloch定理
$$
\psi(x+a) = e^{ika} \psi(x)
$$
\ech
\end{frame}

\begin{frame}
\frametitle{\bch 布里渊区 \ech}
\bch
因为$e^{ika} = e^{i(ka + 2\pi)} = e^{i(k + \frac{2\pi}{a} )}$,$k$的取值精确到周期$\frac{2\pi}{a}$,通常限制$$-\frac{\pi}{a} < k < \frac{\pi}{a}$$
这一限制就是布里渊区。
\ech
\end{frame}

\begin{frame}
\frametitle{\bch 更多的表示论 \ech}
\bch
我们已经看到表示论在量子力学中的强大作用。
\par
之后的章节中将讨论更广泛的对称群,例如:
\begin{itemize}
\item 空间$SO(3)$对称性的量子力学效应:它自然地引出自旋的概念;
\item 狄拉克方程中的4个$\gamma$矩阵其实就是洛伦兹群(狭义相对论的对称群)的一个表示论。
\end{itemize}
\par
不过这些东西我们今天都不讲。
\ech
\end{frame}

\section{Symmetry and Action}
\begin{frame}
\frametitle{\bch 运动方程与作用量 \ech}
\bch
我们好几年没有用过的
$$F = ma$$
和我们半年没有用过的
$$
S[q] = \int dt L(q,\dot{q}) ,\ L(q,\dot{q}) = \frac{1}{2}m\dot{q}^2 - V(q),\ \delta S = 0
$$
是等价的。
\ech
\end{frame}

\begin{frame}
\frametitle{\bch 对称性 \ech}
\bch
我们说一个理论具有对称性,如果
\begin{itemize}
\item 在对称操作下,运动方程左右两边按相同的方式变换;
\item 或:在对称操作下,作用量(或者,很多时候可以严格到拉氏量)不变。
\end{itemize}
这两种定义是等价的,但显然后一种定义更加好用:你只需要傻算就行了。
\ech
\end{frame}

\begin{frame}
\frametitle{\bch 例子:中心简谐势场 \ech}
\bch
三维空间中势场$V(\vec x ) = \frac{1}{2}k \vec x^2$对应的运动方程为
$$
m \frac{d^2 \vec x}{dt^2} = k \vec x
$$
拉氏量为
$$
L = \frac{1}{2} m \dot{\vec x}^2 - \frac{1}{2}k \vec x^2
$$
作旋转变换$\vec x \to R \vec x,\ R \in SO(3)$,试从两种角度证明中心简谐势场具有$SO(3)$对称性。
\ech
\end{frame}

\begin{frame}
\frametitle{\bch 场论作用量 \ech}
\bch
$\phi$场依赖于$D$个自变量$x_\mu$,作用量
$$S[\phi] = \int d^D x \mathcal{L} (\phi,\partial_\mu \phi)$$
$\delta S = 0$当$$\partial_\mu \frac{\partial \mathcal{L}}{\partial (\partial_\mu \phi)} - \frac{\partial \mathcal{L}}{\partial \phi} = 0$$

\ech
\end{frame}

\begin{frame}
\frametitle{\bch 例子 \ech}
\bch
QED拉氏量
$$
\mathcal{L}_{QED} = \bar{\psi} ( i \slashed{D} - m) \psi - \frac{1}{4} F^{\mu \nu} F_{\mu \nu}$$
其中$\bar{\psi} = \psi^\dagger \gamma^0$,$\slashed D = \gamma^\mu ( \partial_\mu - iq A_\mu)$,$F_{\mu \nu} = \partial_\mu A_\nu - \partial_\nu A_\mu$。
\par
试求$\psi$的运动方程:不包含$\psi^\dagger$的导数项,对$\psi^\dagger$变分就得到
$$
\partial_\mu \frac{\partial \mathcal{L}_{QED} }{\partial (\partial_\mu \psi^\dagger)} = 0 = \frac{\partial \mathcal{L}_{QED}}{\partial \psi^\dagger}$$
$$
\gamma^0 (i \slashed{D} - m) \psi = 0
$$
或
$$
(i\slashed{D} - m) \psi =0
$$

\ech
\end{frame}

\begin{frame}
\frametitle{\bch 我今天已经速成QED了? \ech}
\bch
{\Huge \centering tql}
\ech
\end{frame}

\begin{frame}
\frametitle{\bch 场论对称性与守恒流 \ech}
\bch
例如,复自由标量场
$$
\mathcal{L} = \partial_\mu \psi \partial^\mu \psi^* - M^2 \psi \psi^*
$$
和刚才提到的$QED$拉氏量在$U(1)$变换$\psi \to e^{i\lambda} \psi$下是对称的。
\par
每个对称性带来一个守恒流,$U(1)$对称性带来的守恒流是概率流守恒;二次量子化后可以重新诠释为正粒子数-反粒子数守恒。
\ech
\end{frame}











































\end{document}