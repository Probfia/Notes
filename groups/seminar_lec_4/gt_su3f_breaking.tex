\documentclass[aspectratio=1610,14pt,mathserif]{beamer}
\batchmode
%\usepackage{CJKutf8}
\usepackage{beamerthemesplit}
\usetheme{Malmoe}
\useoutertheme[footline=authortitle]{miniframes}
\usepackage{amsmath}
\usepackage{amssymb}
\usepackage{graphicx}
\usepackage{eufrak}
\usepackage{color}
\usepackage{slashed}
\usepackage{simplewick}
\usepackage{tikz}
\usepackage{tcolorbox}
\graphicspath{{../figures/}}
%%figures
\def\lfig#1#2{\includegraphics[width=#1 in]{#2}}
\def\addfig#1#2{\begin{center}\includegraphics[width=#1 in]{#2}\end{center}}
\def\wulian{\includegraphics[width=0.18in]{emoji_wulian.jpg}}
\def\bigwulian{\includegraphics[width=0.35in]{emoji_wulian.jpg}}
\def\bye{\includegraphics[width=0.18in]{emoji_bye.jpg}}
\def\bigbye{\includegraphics[width=0.35in]{emoji_bye.jpg}}
\def\huaixiao{\includegraphics[width=0.18in]{emoji_huaixiao.jpg}}
\def\bighuaixiao{\includegraphics[width=0.35in]{emoji_huaixiao.jpg}}
\def\jianxiao{\includegraphics[width=0.18in]{emoji_jianxiao.jpg}}
\def\bigjianxiao{\includegraphics[width=0.35in]{emoji_jianxiao.jpg}}
%% colors
\def\blacktext#1{{\color{black}#1}}
\def\bluetext#1{{\color{blue}#1}}
\def\redtext#1{{\color{red}#1}}
\def\darkbluetext#1{{\color[rgb]{0,0.2,0.6}#1}}
\def\skybluetext#1{{\color[rgb]{0.2,0.7,1.}#1}}
\def\cyantext#1{{\color[rgb]{0.,0.5,0.5}#1}}
\def\greentext#1{{\color[rgb]{0,0.7,0.1}#1}}
\def\darkgray{\color[rgb]{0.2,0.2,0.2}}
\def\lightgray{\color[rgb]{0.6,0.6,0.6}}
\def\gray{\color[rgb]{0.4,0.4,0.4}}
\def\blue{\color{blue}}
\def\red{\color{red}}
\def\green{\color{green}}
\def\darkgreen{\color[rgb]{0,0.4,0.1}}
\def\darkblue{\color[rgb]{0,0.2,0.6}}
\def\skyblue{\color[rgb]{0.2,0.7,1.}}
%%control
\def\be{\begin{equation}}
\def\ee{\nonumber\end{equation}}
\def\bea{\begin{eqnarray}}
\def\eea{\nonumber\end{eqnarray}}
\def\bch{\begin{CJK}{UTF8}{gbsn}}
\def\ech{\end{CJK}}
\def\bitem{\begin{itemize}}
\def\eitem{\end{itemize}}
\def\bcenter{\begin{center}}
\def\ecenter{\end{center}}
\def\bex{\begin{minipage}{0.2\textwidth}\includegraphics[width=0.6in]{jugelizi.png}\end{minipage}\begin{minipage}{0.76\textwidth}}
\def\eex{\end{minipage}}
\def\chtitle#1{\frametitle{\bch#1\ech}}
\def\bmat#1{\left(\begin{array}{#1}}
\def\emat{\end{array}\right)}
\def\bcase#1{\left\{\begin{array}{#1}}
\def\ecase{\end{array}\right.}
\def\bmini#1{\begin{minipage}{#1\textwidth}}
\def\emini{\end{minipage}}
\def\tbox#1{\begin{tcolorbox}#1\end{tcolorbox}}
\def\pfrac#1#2#3{\left(\frac{\partial #1}{\partial #2}\right)_{#3}}
%%symbols
\def\bropt{\,(\ \ \ )}
\def\sone{$\star$}
\def\stwo{$\star\star$}
\def\sthree{$\star\star\star$}
\def\sfour{$\star\star\star\star$}
\def\sfive{$\star\star\star\star\star$}
\def\rint{{\int_\leftrightarrow}}
\def\roint{{\oint_\leftrightarrow}}
\def\stdHf{{\textit{\r H}_f}}
\def\deltaH{{\Delta \textit{\r H}}}
\def\ii{{\dot{\imath}}}
\def\skipline{{\vskip0.1in}}
\def\skiplines{{\vskip0.2in}}
\def\lagr{{\mathcal{L}}}
\def\hamil{{\mathcal{H}}}
\def\vecv{{\mathbf{v}}}
\def\vecx{{\mathbf{x}}}
\def\vecy{{\mathbf{y}}}
\def\veck{{\mathbf{k}}}
\def\vecp{{\mathbf{p}}}
\def\vecn{{\mathbf{n}}}
\def\vecA{{\mathbf{A}}}
\def\vecP{{\mathbf{P}}}
\def\vecsigma{{\mathbf{\sigma}}}
\def\hatJn{{\hat{J_\vecn}}}
\def\hatJx{{\hat{J_x}}}
\def\hatJy{{\hat{J_y}}}
\def\hatJz{{\hat{J_z}}}
\def\hatj#1{\hat{J_{#1}}}
\def\hatphi{{\hat{\phi}}}
\def\hatq{{\hat{q}}}
\def\hatpi{{\hat{\pi}}}
\def\vel{\upsilon}
\def\Dint{{\mathcal{D}}}
\def\adag{{\hat{a}^\dagger}}
\def\bdag{{\hat{b}^\dagger}}
\def\cdag{{\hat{c}^\dagger}}
\def\ddag{{\hat{d}^\dagger}}
\def\hata{{\hat{a}}}
\def\hatb{{\hat{b}}}
\def\hatc{{\hat{c}}}
\def\hatd{{\hat{d}}}
\def\hatN{{\hat{N}}}
\def\hatH{{\hat{H}}}
\def\hatp{{\hat{p}}}
\def\Fup{{F^{\mu\nu}}}
\def\Fdown{{F_{\mu\nu}}}
\def\newl{\nonumber \\}
\def\vece{\mathrm{e}}
\def\calM{{\mathcal{M}}}
\def\calT{{\mathcal{T}}}
\def\calR{{\mathcal{R}}}
\def\barpsi{\bar{\psi}}
\def\baru{\bar{u}}
\def\barv{\bar{\upsilon}}
\def\qeq{\stackrel{?}{=}}
\def\torder#1{\mathcal{T}\left(#1\right)}
\def\rorder#1{\mathcal{R}\left(#1\right)}
\def\contr#1#2{\contraction{}{#1}{}{#2}#1#2}
\def\trof#1{\mathrm{Tr}\left(#1\right)}
\def\trace{\mathrm{Tr}}
\def\comm#1{\ \ \ \left(\mathrm{used}\ #1\right)}
\def\tcomm#1{\ \ \ (\text{#1})}
\def\slp{\slashed{p}}
\def\slk{\slashed{k}}
\def\calp{{\mathfrak{p}}}
\def\veccalp{\mathbf{\mathfrak{p}}}
\def\Tthree{T_{\tiny \textcircled{3}}}
\def\pthree{p_{\tiny \textcircled{3}}}
\def\dbar{{\,\mathchar'26\mkern-12mu d}}
\def\erf{\mathrm{erf}}
\def\const{\mathrm{constant}}
\def\pheat{\pfrac p{\ln T}V}
\def\vheat{\pfrac V{\ln T}p}
%%units
\def\fdeg{{^\circ \mathrm{F}}}
\def\cdeg{^\circ \mathrm{C}}
\def\atm{\,\mathrm{atm}}
\def\angstrom{\,\text{\AA}}
\def\SIL{\,\mathrm{L}}
\def\SIkm{\,\mathrm{km}}
\def\SIyr{\,\mathrm{yr}}
\def\SIGyr{\,\mathrm{Gyr}}
\def\SIV{\,\mathrm{V}}
\def\SImV{\,\mathrm{mV}}
\def\SIeV{\,\mathrm{eV}}
\def\SIkeV{\,\mathrm{keV}}
\def\SIMeV{\,\mathrm{MeV}}
\def\SIGeV{\,\mathrm{GeV}}
\def\SIcal{\,\mathrm{cal}}
\def\SIkcal{\,\mathrm{kcal}}
\def\SImol{\,\mathrm{mol}}
\def\SIN{\,\mathrm{N}}
\def\SIHz{\,\mathrm{Hz}}
\def\SIm{\,\mathrm{m}}
\def\SIcm{\,\mathrm{cm}}
\def\SIfm{\,\mathrm{fm}}
\def\SImm{\,\mathrm{mm}}
\def\SInm{\,\mathrm{nm}}
\def\SImum{\,\mathrm{\mu m}}
\def\SIJ{\,\mathrm{J}}
\def\SIW{\,\mathrm{W}}
\def\SIkJ{\,\mathrm{kJ}}
\def\SIs{\,\mathrm{s}}
\def\SIkg{\,\mathrm{kg}}
\def\SIg{\,\mathrm{g}}
\def\SIK{\,\mathrm{K}}
\def\SImmHg{\,\mathrm{mmHg}}
\def\SIPa{\,\mathrm{Pa}}

\usepackage{xeCJK}
\setCJKmainfont{STFangsong}
\usepackage{beamerthemesplit}
\usetheme{Boadilla}
%\useoutertheme{smoothbars}
\usecolortheme{beaver}

\usepackage{xcolor}
\usepackage{amsmath}
\usepackage{amssymb}
\usepackage{graphicx}
\usepackage{eufrak}
\usepackage{color}
\usepackage{slashed}
\usepackage{tcolorbox}

%\def\bch{\begin{CJK}{UTF8}{gbsn}}
%\def\ech{\end{CJK}}
\newcommand{\bch}{}
\newcommand{\ech}{}
\def\bcenter{\begin{center}}
\def\ecenter{\end{center}}
\def\skipline{{\vskip0.1in}}
\def\skiplines{{\vskip0.2in}}
\def\tbox#1{\begin{tcolorbox}#1\end{tcolorbox}}

\newcommand{\field}{\mathscr{F}}

\newcommand{\reals}{\mathbb{R}}
\newcommand{\complexs}{\mathbb{C}}
\newcommand{\ints}{\mathbb{Z}}
%\newcommand{\dim}{\mathrm{dim\ }}
\newcommand{\diag}{\mathrm{diag \ }}
\newcommand{\up}{\uparrow}
\newcommand{\down}{\downarrow}
\newcommand{\su}{\mathfrak{su}}
\newcommand{\so}{\mathfrak{so}}
\DeclareMathOperator{\tr}{tr}
\DeclareMathOperator{\diag}{diag}
\newcommand{\card}{\mathrm{card \ }}
\newcommand{\mani}{\mathcal{M}}
\newcommand{\lag}{\mathcal{L}}
\newcommand{\ham}{\mathcal{H}}
\def\secpage#1#2{\begin{frame}\bch\bcenter{\bf \Huge #1} \skipline \tbox{#2}\ecenter\ech\end{frame}}
\newcommand{\mat}[1]{\begin{pmatrix}#1\end{pmatrix}}
\newcommand{\mev}{\ {\rm MeV}}

\newcommand{\bea}{\begin{equation*}\begin{aligned}}
\newcommand{\eea}{\end{aligned}\end{equation*}}

\newcommand{\red}[1]{{\color{red} #1}}
\def\green#1{{\color[rgb]{0.1,0.6,0.3}#1}}
\newcommand{\purple}[1]{{\color{purple} #1}}
\newcommand{\orange}[1]{{\color{orange} #1}}
\newcommand{\blue}[1]{{\color{blue} #1}}
\newtheorem{thm}{定理}[section]
\newtheorem{axm}{公理}[section]
\newtheorem{dfn}{定义}[section]

%\cpic{<尺寸>}{<文件名>}}用于生成居中的图片。
\newcommand{\cpic}[2]{
\begin{center}
\includegraphics[scale=#1]{#2}
\end{center}
}

%\cpicn{<尺寸>}{<文件名>}{<注释>}用于生成居中且带有注释的图片,其label为图片名。
\newcommand{\cpicn}[3]
{
\begin{figure}[h!]
\cpic{#1}{#2}
\caption{#3\label{#2}}
\end{figure}
}

\title{Group Theory: Symmetry in the Particle World}
\substitle{$SU(3)_f$ Symmetry Breaking and Field Theory}
\author{Han G.}
  \date{\today}


\begin{document}

\begin{frame}
 
\titlepage

\end{frame}

\begin{frame}{内容提要}
\tableofcontents
\end{frame}

\section{Quantization of Scalar and Spinor}
\secpage{准备知识:复标量场和旋量场}{$$\lag = \bar{\psi}(i\slashed{\partial}-m)\psi$$}

\begin{frame}{自旋$0$粒子:复标量场}
\begin{itemize}
\item 自由标量场的拉氏量
$$
\lag = \frac{1}{2}(\partial \phi)^2 - \frac{1}{2}m^2 \phi^2
$$
\item 量子化后描述质量为$m$的自旋$0$粒子。
\end{itemize}
\end{frame}

\begin{frame}{自旋$\frac{1}{2}$粒子:旋量场}
\begin{itemize}
\item 自由旋量场的拉氏量
$$
\lag =  \bar{\psi}(i\slashed{\partial}-m)\psi
$$
\item
我们知道,旋量是$4$分量列向量,$\bar{\psi} =\psi^\dagger \gamma^0 $,$\slashed{\partial} = \gamma^\mu \partial_\mu$。这些在量子力学seminar上都讨论了。
\item 量子化后描述质量为$m$的自旋$1/2$粒子(区分正负)。
\end{itemize}
\end{frame}


\begin{frame}{如果有很多种粒子}
那就把这些粒子的拉氏量全加起来。
\cpic{0.4}{happy}
\end{frame}


\section{Flavor Symmetry Breaking}
\secpage{味对称性破缺}{$$m_s \gg m_{u,d}$$}


\begin{frame}{$n$重态的来源}
\begin{itemize}
\item
我们在之前两节的讨论中发现,介子8重态,重子8重态和10重态的来源都是三种夸克态
$$
u = \mat{1\\0 \\0},\quad v = \mat{0\\1\\0},\quad s = \mat{0\\0\\1}
$$
的$SU(3)$旋转对称性。这称作味对称性。
\item
如果$SU(3)_f$对称性是精确的,同种表示中的所有粒子都具有同样的质量(最多相差静电能的量级)。
\item
但是,我们发现,对介子8重态,这些介子质量差得离谱;重子8重态的重子质量也差$\sim 0.2$左右。
\item
来源:奇夸克$s$的质量比其他两个大得多!
$$
\green{m_u \approx 2.3\mev,\quad m_d \approx 4.8\mev,}\quad \red{m_s \approx 95\mev}
$$
\end{itemize}
\end{frame}

\begin{frame}{破缺的对称性}
\begin{itemize}
\item
粒子世界的$SU(3)_f$味对称性因为某种机制被破坏了。
\item
在一级近似下,我们认为$SU(3)_f$破缺成同位旋对称性$SU(2)_i$和超荷(奇异数)对称性$U(1)_s$。$SU(2)_i$对称性在实验上符合得很好。
\item
来源:$u,d$夸克差不多,象征奇异数的$s$夸克重很多。
\item
对称性破缺的其他粒子:原子的轨道量子数$j$在自由$SO(3)$空间中无限制:不可约表示的维数$2j+1$可以取到无穷;但假如把原子放到晶格里,$SO(3)$对称性被破坏,$j$的取值有限个。
\end{itemize}
\end{frame}




\section{Meson Eightfold}
\secpage{介子八重态}{$$4m_K^2=m_\pi^2 + 3m_\eta^2$$}

\begin{frame}{构造介子场}
\begin{itemize}
\item
介子8重态对应$SU(3)_f$的$(1,1)$张量表示,即伴随表示。也就是,介子场可以用盖尔曼矩阵张成
$$
M = \frac{1}{\sqrt 2} M_a \lambda_a = \frac{1}{2} \mat{ M_3 + \frac{1}{\sqrt 3} M_8 & M_1 - iM_2 & M_4 - iM_5 \\
M_1 + iM_2 & -M_3 + \frac{1}{\sqrt 3} M_8 & M_6 - iM_7 \\ M_4 + iM_5 & M_6 + iM_7 & -\frac{2}{\sqrt 3} M_8 }
$$
\item 重新线性组合:注意同位旋对称性没有被破坏,而$\green{(\lambda_{1},\lambda_2,\lambda_3)},\red{(\lambda_4,\lambda_5),(\lambda_6,\lambda_7)},\orange{\lambda_8}$各对应一组同位旋对称性$8 \to \green{3_0} \oplus \red{2_3 \oplus 2_{-3} }\oplus \orange{1_0}$:
$$
M = \mat{\green{\frac{1}{\sqrt 2} \pi^0 }\orange{+ \frac{1}{\sqrt 6} \eta} &\green{ \pi^+} & \red{K^+} \\
\green{ \pi^- }& \green{- \frac{1}{\sqrt 2} \pi^0 }\orange{+ \frac{1}{\sqrt 6} \eta } & \red{K^0} \\
\red{ K^- }& \red{\bar{K}^0}  & \orange{- \frac{2}{\sqrt 6} \eta}}
$$

\end{itemize}
\end{frame}

\begin{frame}{构造完全$SU(3)_f$对称的拉氏量}
\begin{itemize}
\item 最简单的,含动能项和质量项的,$SU(3)_f$不变的标量是
$$
\lag = \frac{1}{2}\tr \left[(\partial M)^2 - m^2 M^2 \right]
$$
\item 计算出$$\tr M^2 =\green{ (\pi^0)^2 + 2\pi^+ \pi^- } + \red{2(K^0 \bar{K}^0 + K^+ K^-)} + \orange{\eta^2}$$
\item 物理意义:介子场解耦合,所有同型(同一个同位旋)的介子质量相同,都是$m$。
\end{itemize}
\end{frame}

\begin{frame}{破坏对称性}
\begin{itemize}
\item 刚刚我们建立了完美$SU(3)_f$的介子场拉氏量,它的各个介子确实是等质量的。
\item 现在我们就来着手破坏这个美丽的对称性。
\item \emph{悲剧就是把美好的事物毁灭给人看。}——鲁迅
\end{itemize}
\end{frame}


\begin{frame}{开局一张图}
\begin{itemize}
\item 我们思考怎么去破坏对称性,为此我们先看熟悉的图
\cpic{0.16}{mesons}
\item 然后我们发现,质量差异与超荷(奇异数)$I_8$密切相关,这很容易理解,因为超荷由$s$夸克数决定,而它比其他两个夸克都重很多。我们又知道,描述超荷的盖尔曼矩阵是$\lambda_8$。
\end{itemize}
\end{frame}



\begin{frame}{后面全靠编}
\begin{itemize}
\item 然后我们发现,质量差异与超荷(奇异数)$I_8$密切相关,这很容易理解,因为超荷由$s$夸克数决定,而它比其他两个夸克都重很多。我们又知道,描述超荷的盖尔曼矩阵是$\lambda_8$。
\item 为此我们在原来的拉氏量中引入一个$\sim \lambda_8$的标量作为破坏对称性的项。这一项的一种可能的,也是最简单的非平凡取法是
$$
\tr M^2 \lambda_8
$$
\item 但$\lambda_8 = \frac{1}{\sqrt 3} I - \sqrt{3}\diag(0,0,1)$,前面的含单位阵的项可以吸收到原来的拉氏量中去,剩下$\sim \tr M^2\diag(0,0,1)$破坏对称性。
\item \blue{课堂小练习}:利用$
M = \mat{\green{\frac{1}{\sqrt 2} \pi^0 }\orange{+ \frac{1}{\sqrt 6} \eta} &\green{ \pi^+} & \red{K^+} \\
\green{ \pi^- }& \green{- \frac{1}{\sqrt 2} \pi^0 }\orange{+ \frac{1}{\sqrt 6} \eta } & \red{K^0} \\
\red{ K^- }& \red{\bar{K}^0}  & \orange{- \frac{2}{\sqrt 6} \eta}}
$
用介子场表示出$\tr M^2 \diag(0,0,1)$。

\end{itemize}
\end{frame}

\begin{frame}{继续编}
\begin{itemize}
\item 所以,我们现在引入对称性破缺后的拉氏量,称作有效拉氏量,为
$$
\lag_{eff} =  \frac{1}{2}\tr \left[(\partial M)^2 - m^2 M^2 \right] - \frac{\alpha}{2} \tr M^2 \diag(0,0,1)
$$
\item 代入刚才\blue{课堂小练习}的结果,有效拉氏量中的质量项是
$
\green{\frac{1}{2} m^2\left[ (\pi^0)^2 + 2\pi^+ \pi^- ]} + \red{(m^2 + \frac{\alpha}{2})(K^0 \bar{K}^0 + K^+ K^-)} + \orange{\frac{1}{2}\left(m^2 + \frac{2}{3} \alpha\right) \eta^2}$
\item
于是我们马上得到,引入$SU(3)_f$破缺后,各个介子的质量为
$$
\green{m_\pi^2 = m^2},\quad \red{m_K^2 = m^2 + \frac{\alpha}{2}},\quad \orange{m_\eta^2 = m^2 + \frac{2}{3} \alpha}
$$
\end{itemize}
\end{frame}


\begin{frame}{检查一下编得怎么样}
\begin{itemize}
\item
消去我们的模型参量$m$和$\alpha$,最后留下3个可观测质量的关系
$$
\blue{4m_K^2 = 3m_\eta^2 + m_\pi^2}
$$
这称为Gell-Mann--Okudo(没错,盖尔曼一人顶两)公式。
\item
代入数据$m_K \approx 496\mev, m_\eta \approx 548\mev, m_\pi \approx 138 \mev$,有
$$
LHS \approx 9.8\times 10^5\mev^2 , RHS \approx 9.2 \times 10^5 \mev^2
$$
还是比较靠谱。要知道,介子八重态的质量差异几乎是$\sim 300\%$,这里把差异降到了$\sim 0.06$,已经非常不错了。
\item
这也暗示了,引入的对称破缺的最低阶修正已经能够很好地解释现象。但是我们这里的模式其实是一个唯象模型:我们没办法第一性地直接计算出模型参数$\alpha$。
\end{itemize}
\end{frame}


\section{Meson Eightfold}
\secpage{重{\small(zh\`ong)}子八重{\small(ch\'ong)}态}{$$3m_\Lambda + m_\Sigma = 2(m_N + m_\Xi)$$}

\begin{frame}{构造重子场}
\begin{itemize}
\item
重子8重态同样对应$SU(3)_f$的$(1,1)$张量表示,即伴随表示。一样的套路:
$$
B = \frac{1}{\sqrt 2} B_a \lambda_a
$$
\item 但是和介子(自旋0波色子)不同,重子是自旋$1/2$费米子,它的场是旋量场。也就是,每个$B_a$是四分量的旋量。
\item 同样的套路,把同位旋相同的粒子写在同一组盖尔曼矩阵,重新线性组合:
$$
B = \mat{\green{\frac{1}{\sqrt 2} \Sigma^0 } + \orange{ \frac{1}{\sqrt 6} \Lambda} & \green{\Sigma^+} & \purple{p} \\
\green{\Sigma^-} & \green{- \frac{1}{\sqrt 2} \Sigma^0} +\orange{ \frac{1}{\sqrt 6} \Lambda} & \purple{n} \\
\red{\Xi^-} & \red{\Xi^0} & \orange{-\frac{2}{\sqrt 6} \Lambda }}
$$
\end{itemize}
\end{frame}

\begin{frame}{构造完全$SU(3)_f$对称的拉氏量}
\begin{itemize}
\item 仿照单粒子自由狄拉克场的拉氏量写出$SU(3)$不变的自由狄拉克场拉氏量
$$
\lag = \tr \bar{B} (i\slashed{\partial} - m) B
$$
\item 很容易想到,$\bar{B}$应该是对每个重子场取狄拉克共轭$\bar{B_a} = B_a^\dagger \gamma^0$后取矩阵转置
$$
\bar B = \mat{\green{\frac{1}{\sqrt 2} \bar \Sigma^0 } + \orange{ \frac{1}{\sqrt 6}\bar \Lambda} & \green{\bar \Sigma^-} & \red{\bar \Xi^-} \\
\green{\bar \Sigma^+} & \green{- \frac{1}{\sqrt 2} \bar \Sigma^0} +\orange{ \frac{1}{\sqrt 6} \bar \Lambda} &  \red{\bar \Xi^0}\\
\purple{\bar p} & \purple{\bar n} & \orange{-\frac{2}{\sqrt 6} \bar \Lambda }}
$
\end{itemize}
\end{frame}

\begin{frame}{自然解耦}
\begin{itemize}
\item 直接计算出
$$
\tr \bar B B =\green{ \bar \Sigma^0 \Sigma^0 + \bar \Sigma^+ \bar \Sigma^+ + \bar \Sigma^- \bar \Sigma^- }+\red{ \bar \Xi^- \Xi^- + \bar \Xi^0 \Xi^0} +\purple{ \bar p p + \bar n n }+\orange{ \bar \Lambda \Lambda}
$$
\item 所有重子自然解耦合,拉氏量中的质量项为
$$
m\left(\green{\sum_\Sigma \bar \Sigma \Sigma }+\red{ \sum_\Xi \bar \Xi \Xi}+\purple{\sum_N \bar N N }+\orange{\bar \Lambda \Lambda} \right)
$$
\item 在$SU(3)_f$对称下,4种重子的质量都是$m$。
\end{itemize}
\end{frame}

\begin{frame}{开局一张图}
\begin{itemize}
\item 我们又来看图思考怎么破坏对称性
\cpic{0.18}{bayons}
\item 明显,我们之前引入$\lambda_8$的套路依然可行。
\end{itemize}
\end{frame}


\begin{frame}{后面全靠编}
\begin{itemize}
\item 因为$\bar B$和$B$之间不对易,可以引入两种可能的对称破缺项
$$\sim \tr \bar B B \diag(0,0,1) = \purple{\bar p p + \bar n n }+\orange{ \frac{2}{3} \bar \Lambda \Lambda}$$
$$\sim \tr \bar B \diag(0,0,1) B = \red{\bar \Xi^- \Xi^- + \bar \Xi^0 \Xi^0  }+ \orange{ \frac{2}{3} \bar \Lambda \Lambda}$$
\item 上两项的$\alpha$倍和$\beta$倍减到原拉氏量中,有效拉氏量的质量项
\bea
\lag_{eff} = &(\mathrm{kinetic\ terms} )\\  &-\green{m\sum_\Sigma \bar \Sigma \Sigma } - \red{(m+\beta) \sum_\Xi \bar \Xi \Xi} -\purple{(m+\alpha)\sum_N \bar N N} \\ &-\orange{\left(m + \frac{2}{3}\alpha + \frac{2}{3}\beta\right)\bar \Lambda \Lambda}
\eea
\end{itemize}
\end{frame}

\begin{frame}{看下编得怎么样}
\begin{itemize}
\item $SU(3)_f$破缺理论中的4种核子质量为
$$
\green{m_\Sigma = m},\quad \red{m_\Xi = m+\beta},\quad \purple{m_N = m + \alpha} ,\quad \orange{m_\Lambda = m + \frac{2}{3}(\alpha + \beta)}
$$
\item 消去模型参数,得到重子八重态的Gell-Mann--Okudo公式
$$
\blue{3m_\Lambda + m_\Sigma = 2(m_N + m_\Xi)}
$$
\item 代入数值$m_N \approx 939\mev, m_\Sigma \approx 1193\mev, m_\Xi \approx 1318\mev, m_\Lambda \approx 1116\mev$得到
$$
LHS \approx 4541\mev, \quad RHS \approx 4514\mev
$$
\item
很好!
\end{itemize}
\end{frame}

\section{Representation}
\secpage{对称性破缺与表示论}{Throw away the 27!}


\begin{frame}{我们从群论角度看看我们做了什么}
\begin{itemize}
\item
我们的拉氏量包含了两个$(1,1)$张量相乘的项$$\varphi^i_j \varphi^k_l$$
\item
这个表示按道理可以分解成
$8\otimes 8 = 27\oplus 10\oplus 10^* \oplus 8 \oplus 8 \oplus 1$,但因为两个相乘的张量上一样的,所以事实上我们只能得到
$$
(8\otimes 8)_S = 27\oplus 8 \oplus 1
$$
\item
\blue{课堂小练习}:证明为什么$(8\otimes 8)_S = 27\oplus 8 \oplus 1$
\end{itemize}
\end{frame}

\begin{frame}{只保留伴随表示}
\begin{itemize}
\item
本来我们要求$\varphi^i_j \varphi^k_l$按张量变换
$$
\varphi^i_j \varphi^k_l \to U^i_p U^k_r (U^\dagger)^q_j (U^\dagger)^s_l  \varphi^p_q \varphi^r_s
$$
它的各个不可约表示分解也应该按相应规则变换。
\item
但现在我们加入了$\sim \varphi^2 \lambda_8$的项,相当于要求
$$
\varphi^i_m \varphi^m_k (\lambda_8)^k_j
$$
按$(1,1)$张量变换,这相当于仅仅保留了伴随表示8。
\end{itemize}
\end{frame}

\begin{frame}{上帝恩惠搞理论的}
\begin{itemize}
\item
不知道为什么,在物理中,我们从来都只用考虑几个最低价的表示,特别是基础表示和伴随表示。\emph{{\small
\item
George Zweig, who discovered quarks independently of Gell-Mann, recalled that he thought that the weak interaction currents of the strongly interacting particles should also be classified in representations of $SU(3)$, and that both the 8- and 27-dimensional representations were to be used. At the time, a decay process requiring the presence of the 27 had been seen by an experimentalist of high reputation working with a strong team using a well understood technique.
\item Zweig went to talk to Feynman, who liked the idea of applying $SU(3)$ to the decay, but kept saying “Throw away the 27!”
\item Feynman turned out to be right; the experimentalists were mistaken.}}
\end{itemize}
\end{frame}


\section{Pause}
\secpage{课程中止}{下学期继续}

\begin{frame}{回顾}

我们来回顾一下这学期的群论都学了什么:
\begin{itemize}
\item
对称性和群,群的定义和实例。
\item
有限群的表示论:可约与不可约,幺正定理,舒尔引理,特征标正交定理......
\item
连续群的表示论:$SO(N)$和$SU(N)$张量,基础表示和伴随表示,乘积表示的直和分解,群流形和$SU(2)$双覆盖......
\item
群论与物理:不可约表示与量子力学简并,内部对称性与外部对称性,同位旋对称性,$SU(3)_f$对称性与核子物理......
\end{itemize}
\end{frame}

\begin{frame}{谈谈感想}
大家来谈谈学习群论的感想。
\cpic{0.4}{good}
\end{frame}

\begin{frame}{展望}
我们来看看下学期我们继续学什么
\begin{itemize}
\item
一般李群的李代数
\item
相对论的对称群:洛伦兹群和旋量。
\item
膨胀宇宙与共形代数。
\item
规范对称性。
\end{itemize}
\end{frame}


\begin{frame}{下学期继续我们有趣的对称之旅}
\cpic{0.35}{az}
明年再见!
\end{frame}


















\end{document}