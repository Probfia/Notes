\documentclass[CJK]{beamer}
\input{macros.tex}


\newcommand{\field}{\mathscr{F}}

\newcommand{\reals}{\mathbb{R}}
\newcommand{\complexs}{\mathbb{C}}
\newcommand{\ints}{\mathbb{Z}}
%\newcommand{\dim}{\mathrm{dim\ }}
\newcommand{\diag}{\mathrm{diag \ }}
\newcommand{\up}{\uparrow}
\newcommand{\down}{\downarrow}
\newcommand{\su}{\mathfrak{su}}
\newcommand{\so}{\mathfrak{so}}
\newcommand{\tr}{\mathrm{tr\ }}
\newcommand{\card}{\mathrm{card \ }}

\newtheorem{thm}{定理}[section]
\newtheorem{axm}{公理}[section]
\newtheorem{dfn}{定义}[section]

%\cpic{<尺寸>}{<文件名>}}用于生成居中的图片。
\newcommand{\cpic}[2]{
\begin{center}
\includegraphics[scale=#1]{#2}
\end{center}
}

%\cpicn{<尺寸>}{<文件名>}{<注释>}用于生成居中且带有注释的图片,其label为图片名。
\newcommand{\cpicn}[3]
{
\begin{figure}[h!]
\cpic{#1}{#2}
\caption{#3\label{#2}}
\end{figure}
}

\title{Group Theory\\ Talk 1-Introduction and Discrete Groups}
  \author{}
  \date{}


\begin{document}

\begin{frame}
 
\begin{center}
\begin{Large}
\bch
{\bf Group Theory}

{\vskip 0.3in}

Talk 1-Introduction and Discrete Groups

\ech
\end{Large}
\end{center}

\vskip 0.2in
\begin{center}
Han Gao
\vskip 0.1in
gaoh26@mail2.sysu.edu.cn
\vskip 0.2in
{\tiny \url{https://github.com/Probfia/Notes/blob/master/groups/seminar_lec/talk1.pdf} }\\
\end{center}


\end{frame}

\section{The Book}
\begin{frame}
\frametitle{\bch 参考书目 \ech}
\bch
\emph{Group Theory in a Nutshell for Physicists} by Anthony Zee. (徐一鸿)
\cpic{0.2}{book}
\ech
\end{frame}

\begin{frame}
\frametitle{\bch A. Zee \ech}
\bch
为什么徐一鸿的英文姓是Zee?
\cpic{0.3}{az.jpg}
\ech
\end{frame}

\begin{frame}
\frametitle{\bch 上海阿拉 \ech}
\bch
因为他是上海人。上海人就要说上海话。
\cpic{0.06}{zikawei}
\ech
\end{frame}

\section{Introduction}
\begin{frame}
\frametitle{\bch 对称性 \ech}
\bch
观察下面几组图形,不要思考,直观地给每组中图形的对称性排个序。
\cpic{0.2}{sym1}

\ech
\end{frame}

\begin{frame}
\frametitle{\bch 对称性 \ech}
\bch

\begin{center}
\includegraphics[scale=0.2]{tri}
\includegraphics[scale=0.2]{etri}
\includegraphics[scale=0.2]{rtri}
\end{center}

\ech
\end{frame}

\begin{frame}
\frametitle{\bch 对称性 \ech}
\bch

\begin{center}
\includegraphics[scale=0.15]{rtri}
\includegraphics[scale=0.15]{pol4}
\includegraphics[scale=0.15]{pol6}
\includegraphics[scale=0.15]{pol24}
\includegraphics[scale=0.15]{cir}
\end{center}

\ech
\end{frame}

\begin{frame}
\frametitle{\bch 衡量对称性 \ech}
\bch
直观告诉我们可以分出图形对称性的大小,如何定量地去衡量?
\par
设$T$是对图形的一个操作,例如镜面反射和旋转。称$T$是一个对称操作,若操作前后图形完全相同。
\par
定义操作的乘法$T_2 T_1$为先进行操作1,再进行操作2。显然,若$T_1$和$T_2$都是对称操作,则它们的乘积也是对称操作(封闭性)。


\ech
\end{frame}

\begin{frame}
\frametitle{\bch 三角形的对称群 \ech}
\bch
记$I$为恒等操作(什么也不干),$R(\theta)$为逆时针旋转$\theta$角的操作,$r$为沿竖直的轴镜像反射的操作,则对三角形组,它们各自的所有对称操作组成的集合为
\begin{center}
\includegraphics[scale=0.2]{tri}
\includegraphics[scale=0.2]{etri}
\includegraphics[scale=0.2]{rtri}
\end{center}
$$S_1 = \{I\},\ S_2 = \{I,r\},\ S_3 =\{I,r,R(120^\circ),R(240^\circ)\}$$

\ech
\end{frame}

\begin{frame}
\frametitle{\bch 正多边形的对称群 \ech}
\bch

\begin{center}
\includegraphics[scale=0.15]{rtri}
\includegraphics[scale=0.15]{pol4}
\includegraphics[scale=0.15]{pol6}
\includegraphics[scale=0.15]{pol24}
\includegraphics[scale=0.15]{cir}
\end{center}
正多边形各自的所有对称操作组成的集合为
$$D_3 = \{I,r,R(120^\circ),R(240^\circ)\},\ D_4 = \{I,r,R(90^\circ),R(180^\circ),R(270^\circ),rR(\cdots)\},$$
$$D_6 = \{I,r,R(60^\circ),R(120^\circ),R(180^\circ),R(240^\circ),R(300^\circ),rR(\cdots)\},$$
$$D_{24} = \{I,r,R(n\cdot 15^\circ) \ \mathrm{for} \ n=1,2,\cdots,23\ ,rR(\cdots)\},$$
$$O(2) \equiv D_\infty = \{I,r,R(\theta) \ \mathrm{for} \ \theta \in (0^\circ, 360^\circ)\ ,rR(\cdots)\}$$


\ech
\end{frame}

\begin{frame}
\frametitle{\bch 对称群 \ech}
\bch
群是一个对“乘法”封闭的集合$G$,并且
\begin{itemize}
\item 存在单位元$I \in G$,使得对任意其他元素$g \in G$都有$gI = Ig = g$。
\item 存在逆元,对任意元素$g \in G$,都存在一个元素$g^{-1}\in G$使得$g^{-1} g = gg^{-1} = I$。
\item 结合律成立,即$g_1 ( g_2 g_3) = (g_1 g_2) g_3$,乘法可以保持顺序任意加括号。
\end{itemize}
试验证刚才讨论的对称操作组成的集合满足上面三个要求,因此这些集合称为是图形对应的对称群。 
\par
经验总结:对称性越“大”,对称群的元素就越多。

\ech
\end{frame}

\begin{frame}
\frametitle{\bch Why Symmetry \ech}
\bch
我们看到群论是研究对称性的工具。但为什么要研究对称性?
\par
对称性越大的理论,往往越“对”:
\begin{itemize}
\item 牛顿力学:3维空间旋转对称性 $\Rightarrow$ 相对论:时空(3+1)维对称性。
\par
群论语言:$SO(3) \to SO(3,1)$
\item 核物理的同位旋:质子中子对称性 $\Rightarrow$ 三种夸克对称性。
\par
群论语言:$SU(2) \to SU(3)$
\item 粒子物理标准模型:$SU(3) \times SU(2) \times U(1)$。
\end{itemize}

\ech
\end{frame}

\begin{frame}
\frametitle{\bch 对称性的其他应用 \ech}
\bch
除了指导理论以外,对称性还有其他的直接物理效应:

\begin{itemize}
\item 连续对称性:时空平移对称性:能量/动量守恒;空间旋转对称性:角动量守恒。
\item 离散对称性:凝聚态物理学:格矢平移对称性$\Rightarrow$波函数的Bloch定理;晶体的旋转对称性:石墨烯具有的$D_6$对称性。
\cpic{0.2}{zx}
\begin{center}
{\small \bf 凝聚态物理学专家周翔}
\end{center}
\item 量子力学:自旋$1/2$粒子:$SO(3)$与$SU(2)$群的联系(同构)。
\end{itemize}

\ech
\end{frame}

\section{Groups}
\begin{frame}
\frametitle{\bch 阿贝尔群 \ech}
\bch
群中的乘法一般是不可交换的,即$gh \not= hg$。若$gh = hg$对任意元素都成立,则称这个群是可交换群或阿贝尔群。

\ech
\end{frame}

\begin{frame}
\frametitle{\bch 群的其他实例 \ech}
\bch
三维空间旋转群$SO(3)$非阿贝尔,但$SO(2)$是阿贝尔群。
\par
方程$z^N = 1$在复数域的所有根构成阿贝尔群$Z_N$。
\par
$U(1) = \{ e^{i \theta},\ \theta \in \reals\}$是$Z_N$的连续极限。
\par
所有的$n$阶方阵不构成群(不一定有逆元);所有$n$阶不可逆方阵不构成群(没有单位元);所有$n$阶可逆方阵构成群。

\ech
\end{frame}

\begin{frame}
\frametitle{\bch 四元数群 \ech}
\bch
考虑集合$Q = \{ 1,-1,i,j.k,-i,-j,-k\}$,定义元素间的乘法如下
$$ i^2 = j^2 = k^2 = -1;$$
$$ ij = k,\ jk = i,\ ki = j;$$
$$ji = -k,\ kj = -i,\ ik = -j.$$
这使得$Q$构成一个非阿贝尔群,称为(哈密顿)四元数群。

\ech
\end{frame}


\begin{frame}
\frametitle{\bch 子群 \ech}
\bch
若(从集合角度)$H\subset G$,且$H$也构成群,则称$H$是$G$的子群。\par
例如$SO(2) \subset SO(3)$。


\ech
\end{frame}

\begin{frame}
\frametitle{\bch 直积 \ech}
\bch
两集合$F$和$G$的笛卡尔积定义为$F\otimes G = \{ (f,g) | f\in F, g \in G\}$。
若$F$和$G$构成群,则可以定义$F\otimes G$上的乘法
$$ (f_1,g_1) (f_2,g_2) = (f_1 f_2,g_1g_2)$$
这显然使得$F\otimes G$构成群,称为$F$和$G$的直积。
\par
显然有$F,G \subset F \otimes G$。


\ech
\end{frame}

\begin{frame}
\frametitle{\bch 同态和同构 \ech}
\bch
称群$G$和$H$同态,若存在映射$f: G \to H$使得对任意元素有$f(g_1 g_2) = f(g_1) f(g_2)$。映射$f$称为同态映射。
\par
例如一个群总和自己的某个子群同态。
\par
若$f$是双射(单且满),则称$G$和$H$同构。例如$SO(2) \simeq U(1)$,再例如,$\reals$同加法构成的群与$\reals_+$同乘法构成的群同构,同构映射$f = \ln$。
\par
同构的意义在于,若两个群同构,只需要研究一个就可以知道另一个群的性质。


\ech
\end{frame}


\begin{frame}
\frametitle{\bch 不变子群 \ech}
\bch
设$F$是一个群,$g$是某个可以与$F$中元素做乘法的元素,记$gF = \{ gf | f \in F\},\ Fg = \{ fg | f \in F\}$。
\par
称$G$的子群$H$是一个不变子群(规范子群),若对任意$g \in G/H$,有$g^{-1} H g = H$。例如$Z_2$是$Z_4$的一个不变子群。
\par
试说明一个群总有两个平凡不变子群$\{I\}$和$G$本身。若群$G$没有非平凡不变子群,则称$G$是简单的。


\ech
\end{frame}


\begin{frame}
\frametitle{\bch 衡量非阿贝尔性:导群 \ech}
\bch
定义群$G$的导群(换位子群,derived subgroup)为$D[G] = \{ \langle a,b \rangle \equiv (ba)^{-1} (ab) | a,b \in G \}$。\par
试说明:
\begin{itemize}
\item 导群的确是一个子群(我也不会证)。
\item 导群的大小衡量了群的非阿贝尔性。阿贝尔群$A$的派生子群$D[A] = \{I\}$。
\item 导群是不变子群(证明$g^{-1} \langle a,b \rangle g = \langle g^{-1} a g , g^{-1} b g \rangle$)。
\end{itemize}
试计算四元素群的导群,并验证它是不变子群。

\ech
\end{frame}


\begin{frame}
\frametitle{\bch 余集 \ech}
\bch
设$H$是$G$的一个不变子群,对选定的元素$g \in G$,称$gH$为$H$的一个左余集(试说明左余集一般不构成群)。
\par
定义左余集见的乘法$(g_a H) (g_b H) \equiv \{ g_a h_i g_b h_j | h_i,h_j \in H\}$。试证明代表元的表达式一定在另一个左余集中。


\ech
\end{frame}

\begin{frame}
\frametitle{\bch 左余集间的乘法 \ech}
\bch
证明:{\blue 
$$ g_a h_i g_b h_j = g_a g_b g_b^{-1} h_i g_b h_j = g_a g_b (g_b^{-1} h_i g_b ) h_j$$
由于$G$封闭,存在一个元素$g_c = g_a g_b$;又因为$H$是不变子群,$g_b^{-1} h_i g_b$依然是$H$的元素,记作$h_k$。再利用$H$的封闭性,$h_k h_j$依然在$H$中,记作$h_l$。故$g_a h_i g_b h_j = g_c h_l$。 }
\par \ \\
由此可以看出$(g_a H) (g_b H) = (g_a g_b) H$,因此$H$的所有左余集构成一个群,称为商群$Q = G/H$。


\ech
\end{frame}


\begin{frame}
\frametitle{\bch 商群 \ech}
\bch
动动手,计算以下商群
\begin{itemize}
\item 计算$F\otimes G/G$。
\item  计算$Q/Z_2$。
\item 证明$Z_2 = \{1,-1\}$是$Z_4 = \{1,i,-1,-i\}$的一个不变子群,并计算$Z_4/Z_2$。
\item 计算$Z_{2N}/Z_2$。
\item 计算上例的连续极限$U(1)/Z_2$,认识对群除以$Z_2$的作用。
\end{itemize}


\ech
\end{frame}





\section{Discrete Groups}
\begin{frame}
\frametitle{\bch 离散群 \ech}
\bch
元素有限的群称为有限群,元素可数的群称为离散群。$G$中元素的个数记为$\card G$\par
循环群,拉格朗日定理,乘法表,表示论......看黑板,懒得打字了。


\ech
\end{frame}

\begin{frame}
\frametitle{\bch $D_n$的更多性质 \ech}
\bch
二面体群$D_n$是正$n$边形的对称群。
$$
D_n = \{I, r, R\left(\frac{2k\pi}{n}\right) \ \mathrm{for}\ k=1,2,\cdots,n-1 \ ,rR(\cdots)\}
$$
证明以下关系
$$R(\theta)^{-1} r R(\theta) = r R(2\theta),$$
$$rR(\theta) r = R(2\pi - \theta),$$
$$\langle r , R(\theta) \rangle = R(2\theta)$$

\ech
\end{frame}

\begin{frame}
\frametitle{\bch $D_n$的子群 \ech}
\bch
由刚刚证明的关系可以得到$D_n$的导群
$$D[D_n] = \{I,R\left(\frac{4k\pi}{n}\right) \ \mathrm{for}\ k=1,2,\cdots\}$$
\par
$D_n$显然还有一个循环子群,生成元是$g = R\left(\frac{2\pi}{n}\right)$。

\ech
\end{frame}








\section{Rings and Fields}
\begin{frame}
\frametitle{\bch 半群和环 \ech}
\bch
将群的定义中对逆元存在的要求删去,集合连同乘法构成的代数结构称为半群。
\par
设$R$是一个集合,其上定义有两种运算:乘法$\times$和加法$+$,若$R$对乘法构成半群,对加法构成阿贝尔群,则称$R(\times,+)$为一个环。
\par
环中的乘法单位元记为1,加法单位元记为0,$a$的加法逆元记为$-a$。加法和乘法不是孤立的,例如利用环的定义可以证明对任意元素$a$,$0 \times a = a \times 0 = 0$。
\par
(你在说什么......)


\ech
\end{frame}

\begin{frame}
\frametitle{\bch 域 \ech}
\bch
将环的定义中对乘法构成半群的要求改为群(但对有限个元素可以不存在逆元),则$F(\times,+)$构成域。$a$的乘法逆元记为$\frac{1}{a}$。\par
例子:
\begin{itemize}
\item 全体自然数连同自然乘法和加法不构成环,全体整数连同自然乘法和加法构成环;
\item 全体有理数连同自然乘法和加法构成域;
\item 全体$n$阶方阵连同矩阵乘法和加法构成环;
\item 全体次数$<n$的多项式构成连同多项式乘法和加法构成环,称为多项式环,全体次数$<n$的分式构成域,称为分数域。
\end{itemize}
更多关于环和域的性质参见抽象代数的教材,这两个代数结构虽然和群紧密相连,但不属于群论的讨论内容。


\ech
\end{frame}


















\end{document}
