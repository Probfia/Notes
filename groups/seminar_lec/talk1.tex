\documentclass[CJK]{beamer}
\input{macros.tex}


\newcommand{\field}{\mathscr{F}}

\newcommand{\reals}{\mathbb{R}}
\newcommand{\complexs}{\mathbb{C}}
\newcommand{\ints}{\mathbb{Z}}
%\newcommand{\dim}{\mathrm{dim\ }}
\newcommand{\diag}{\mathrm{diag \ }}
\newcommand{\up}{\uparrow}
\newcommand{\down}{\downarrow}
\newcommand{\su}{\mathfrak{su}}
\newcommand{\so}{\mathfrak{so}}
\newcommand{\tr}{\mathrm{tr\ }}

\newtheorem{thm}{定理}[section]
\newtheorem{axm}{公理}[section]
\newtheorem{dfn}{定义}[section]

%\cpic{<尺寸>}{<文件名>}}用于生成居中的图片。
\newcommand{\cpic}[2]{
\begin{center}
\includegraphics[scale=#1]{#2}
\end{center}
}

%\cpicn{<尺寸>}{<文件名>}{<注释>}用于生成居中且带有注释的图片,其label为图片名。
\newcommand{\cpicn}[3]
{
\begin{figure}[h!]
\cpic{#1}{#2}
\caption{#3\label{#2}}
\end{figure}
}

\title{Group Theory\\ Talk 1-Introduction and Discrete Groups}
  \author{}
  \date{}


\begin{document}

\begin{frame}
 
\begin{center}
\begin{Large}
\bch
Group Theory

{\vskip 0.3in}

Talk 1-Introduction and Discrete Groups

\ech
\end{Large}
\end{center}

\vskip 0.2in


\end{frame}

\section{Book}
\begin{frame}
\frametitle{\bch 参考书目 \ech}
\bch
\emph{Group Theory in a Nutshell for Physicists} by Anthony Zee. (徐一鸿)
\cpic{0.2}{book}
\ech
\end{frame}

\begin{frame}
\frametitle{\bch A. Zee \ech}
\bch
为什么徐一鸿的英文姓是Zee?
\cpic{0.3}{az.jpg}
\ech
\end{frame}

\begin{frame}
\frametitle{\bch 上海阿拉 \ech}
\bch
因为他是上海人。上海人就要说上海话。
\cpic{0.06}{zikawei}
\ech
\end{frame}

\section{Introduction}
\begin{frame}
\frametitle{\bch 对称性 \ech}
\bch
观察下面几组图形,不要思考,直观地给每组中图形的对称性排个序。
\cpic{0.2}{sym1}

\ech
\end{frame}

\begin{frame}
\frametitle{\bch 对称性 \ech}
\bch

\begin{center}
\includegraphics[scale=0.2]{tri}
\includegraphics[scale=0.2]{etri}
\includegraphics[scale=0.2]{rtri}
\end{center}

\ech
\end{frame}

\begin{frame}
\frametitle{\bch 对称性 \ech}
\bch

\begin{center}
\includegraphics[scale=0.15]{rtri}
\includegraphics[scale=0.15]{pol4}
\includegraphics[scale=0.15]{pol6}
\includegraphics[scale=0.15]{pol24}
\includegraphics[scale=0.15]{cir}
\end{center}

\ech
\end{frame}

\begin{frame}
\frametitle{\bch 衡量对称性 \ech}
\bch
直观告诉我们可以分出图形对称性的大小,如何定量地去衡量?
\par
设$T$是对图形的一个操作,例如镜面反射和旋转。称$T$是一个对称操作,若操作前后图形完全相同。
\par
定义操作的乘法$T_2 T_1$为先进行操作1,再进行操作2。显然,若$T_1$和$T_2$都是对称操作,则它们的乘积也是对称操作(封闭性)。


\ech
\end{frame}

\begin{frame}
\frametitle{\bch 三角形的对称群 \ech}
\bch
记$I$为恒等操作(什么也不干),$R(\theta)$为逆时针旋转$\theta$角的操作,$r$为沿竖直的轴镜像反射的操作,则对三角形组,它们各自的所有对称操作组成的集合为
\begin{center}
\includegraphics[scale=0.2]{tri}
\includegraphics[scale=0.2]{etri}
\includegraphics[scale=0.2]{rtri}
\end{center}
$$S_1 = \{I\},\ S_2 = \{I,r\},\ S_3 =\{I,r,R(120^\circ),R(240^\circ)\}$$

\ech
\end{frame}

\begin{frame}
\frametitle{\bch 正多边形的对称群 \ech}
\bch

\begin{center}
\includegraphics[scale=0.15]{rtri}
\includegraphics[scale=0.15]{pol4}
\includegraphics[scale=0.15]{pol6}
\includegraphics[scale=0.15]{pol24}
\includegraphics[scale=0.15]{cir}
\end{center}
正多边形各自的所有对称操作组成的集合为
$$D_3 = \{I,r,R(120^\circ),R(240^\circ)\},\ D_4 = \{I,r,R(90^\circ),R(180^\circ),R(270^\circ)\},$$
$$D_6 = \{I,r,R(60^\circ),R(120^\circ),R(180^\circ),R(240^\circ),R(300^\circ)\},$$
$$D_{24} = \{I,r,R(n\cdot 15^\circ)\},\ D_\infty = \{I,r,R(\theta) \ for \ \theta \in [0^\circ, 360^\circ)\}$$


\ech
\end{frame}

\begin{frame}
\frametitle{\bch 对称群 \ech}
\bch
群是一个对”乘法“封闭的集合$G$,并且
\begin{itemize}
\item 存在单位元$I \in G$,使得对任意其他元素$g \in G$都有$gI = Ig = g$。
\item 存在逆元,对任意元素$g \in G$,都存在一个元素$g^{-1}\in g$使得$g^{-1} g = gg^{-1} = I$。
\item 结合律成立,即$g_1 ( g_2 g_3) = (g_1 g_2) g_3$,乘法可以保持顺序任意加括号。
\end{itemize}
试验证刚才讨论的对称操作组成的集合满足上面三个要求,因此这些集合称为是图形对应的对称群。 
\par
经验总结:对称性越“大”,对称群的元素就越多。

\ech
\end{frame}

\section{Groups}
\begin{frame}
\frametitle{\bch 阿贝尔群 \ech}
\bch
群中的乘法一般是不可交换的,即$gh \not= hg$。若$gh = hg$对任意元素都成立,则称这个群是可交换群或阿贝尔群。

\ech
\end{frame}

\begin{frame}
\frametitle{\bch 群的其他实例 \ech}
\bch
三维空间旋转群$SO(3)$非阿贝尔,但$SO(2)$是阿贝尔群。
\par
方程$z^N = 1$在复数域的所有根构成阿贝尔群$Z_N$。
\par
$U(1) = \{ e^{i \theta},\ \theta \in \reals\}$是$Z_N$的连续极限。
\par
所有的$n$阶方阵构成非阿贝尔群。

\ech
\end{frame}

\begin{frame}
\frametitle{\bch 子群 \ech}
\bch
若(从集合角度)$H\subset G$,且$H$也构成群,则称$H$是$G$的子群。\par
例如$SO(2) \subset SO(3)$。


\ech
\end{frame}

\begin{frame}
\frametitle{\bch 直积 \ech}
\bch
两集合$F$和$G$的笛卡尔积定义为$F\otimes G = \{ (f,g) | f\in F, g \in G\}$。
若$F$和$G$构成群,则可以定义$F\otimes G$上的乘法
$$ (f_1,g_1) (f_2,g_2) = (f_1 f_2,g_1g_2)$$
这显然使得$F\otimes G$构成群,称为$F$和$G$的直积。
\par
显然有$F,G \subset F \otimes G$。


\ech
\end{frame}



\section{Discrete Groups}
\begin{frame}
\frametitle{\bch 离散群 \ech}
\bch
元素有限的群称为有限群,元素可数的群称为离散群。\par
循环群,拉格朗日定理,乘法表,表示论,同态和同构,不变子群和派生子群,商群......看黑板,懒得打字了。


\ech
\end{frame}



























\end{document}
