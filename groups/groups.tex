\documentclass[a4paper,11pt]{ctexart}

\usepackage{amsmath}
\usepackage{color}
\usepackage{mathrsfs}
\usepackage{hyperref}
\usepackage{graphicx}
\usepackage{cleveref}
\usepackage{amssymb}
\usepackage{mathrsfs}

\crefname{equation}{}{}
\crefname{figure}{图}{图}
\crefname{footnote}{注释}{注释}

\newcommand{\beq}{\begin{equation}}
\newcommand{\eeq}{\end{equation}}
\newcommand{\bea}{\begin{equation}\begin{aligned}}
\newcommand{\eea}{\end{aligned}\end{equation}}
%\renewcommand{\Omega}{\varOmega}
\newcommand{\field}{\mathscr{F}}
\newcommand{\red}{\color{red}}
\newcommand{\reals}{\mathbb{R}}
\newcommand{\complexs}{\mathbb{C}}
\newcommand{\ints}{\mathbb{Z}}
%\newcommand{\dim}{\mathrm{dim\ }}

\newtheorem{thm}{定理}[section]
\newtheorem{axm}{公理}[section]
\newtheorem{dfn}{定义}[section]

%\cpic{<尺寸>}{<文件名>}}用于生成居中的图片。
\newcommand{\cpic}[2]{
\begin{center}
\includegraphics[scale=#1]{#2}
\end{center}
}

%\cpicn{<尺寸>}{<文件名>}{<注释>}用于生成居中且带有注释的图片,其label为图片名。
\newcommand{\cpicn}[3]
{
\begin{figure}[h!]
\cpic{#1}{#2}
\caption{#3\label{#2}}
\end{figure}
}

\title{群的实例和表示论简介}
\author{Probfia}

\begin{document}
\maketitle
\tableofcontents

\section{群及其实例}
\subsection{群的定义}
简单的说,群是一个集合连同一个运算构成的封闭代数结构。群的弱化为半群,它的定义如下:
\begin{dfn}[半群]
设$G$是一个集合,$\cdot$为集合中元素的一个运算,称$(G,\cdot)$构成一个半群。若
\begin{enumerate}
\item 封闭性: $\forall a,b \in G,\ a \cdot b \in G$;
\item 存在单位元: $\exists e \in G, \ \forall a \in G,\ a\cdot e = e \cdot a = a$
\item 结合律: $\forall a,b,c \in G,\ a\cdot(b\cdot c) = (a \cdot b) \cdot c =$,因此可以将连乘无歧义地记作$a\cdot b\cdot c$。
\end{enumerate}
\end{dfn}
举例来说,全体自然数连同自然加法运算构成半群,单位元为$0$;全体$n$阶方阵构成的集合连同矩阵乘法运算也构成半群,单位元为$n$阶单位阵$E$。一般在不引起歧义的时候将$(G,\cdot)$简记为$G$,并且将运算$a \cdot b$简记成$ab$。
\par
群的定义为半群加上逆元的存在性
\begin{dfn}[群]
若$(G,\cdot)$为一个半群,称$(G,\cdot)$一个群,若
\begin{enumerate}
\item 存在逆元: $\forall a \in G,\ \exists r \in G,\ a\cdot r = r\cdot a = e$,其中$e$为$G$的单位元。
\end{enumerate}
\end{dfn}
举例来说,全体整数连同自然加法运算构成群,单位元为$0$,逆元为某个元素的负;全体可逆$n$阶方阵构成的集合连同矩阵乘法运算也构成群,单位元为$n$阶单位阵$E$,逆元为某个矩阵的逆。上面的例子建议了下面定理的正确性。
\begin{thm}[逆元的唯一性]
群$G$中任意元素的逆元唯一,因此可以将$a$的逆元无歧义地记作$a^{-1}$。
\end{thm}
证明如下:\par
\emph{设$r,s$同时为$a$的逆元,则有$e = ar$和$e = sa$,在第一个等式两边左乘$s$得到
\beq
s = se = sar = er = r
\eeq
}
\par
有时候我们会遇到阿贝尔群,它的定义是满足交换律的群
\begin{dfn}[阿贝尔群]
称群$G$是一个阿贝尔群,或者说,群$G$是阿贝尔的,或可交换的,若群$(G,\cdot)$满足
\begin{enumerate}
\item 交换律: $\forall a,b \in G,\ a\cdot b = b \cdot a$。
\end{enumerate}
\end{dfn}
从上面的例子看出,全体整数构成的群是阿贝尔的,而可逆方阵构成的群则不是。

\subsection{旋转群}
\subsubsection{SO(2)群}
我们考虑2维平面对一个向量绕原点旋转的操作。例如将一个向量$\vec{x}$旋转一定角度$\theta$的操作后得到的向量记作$\hat{R}(\theta) \vec{x}$,我们可以对$\hat{R}(\theta) \vec{x}$再次沿另一个角度旋转,得到向量$\hat{R}(\phi) (\hat{R}(\theta) \vec{x})$。但我们知道,旋转两次这个操作也可以通过一次旋转$\hat{R}(\theta + \phi)$完成,因此,我们定义旋转操作的乘法运算
\beq
\hat{R}_1 \hat{R}_2, \ \forall \vec{x}\in \reals^2,\ \hat{R}_1 (\hat{R}_2 \vec{x}) = \hat{R}_1 (\hat{R}_2 \vec{x})
\eeq
即两个旋转操作的乘积的结果是,先按第二个操作旋转,再按第一个操作旋转。
\par
我们看到,旋转操作的乘积依然是旋转操作,再加之,零角度旋转是旋转操作的单位元;每个旋转操作$\hat{R}(\theta)$的逆元就是$\hat{R}(-\theta)$。因此,所有二维旋转操作$\hat{R}$连同它们的乘法运算构成一个群。这个群记作$SO(2)$。此外我们注意到,$SO(2)$事实上是一个阿贝尔群,因为两次旋转的总旋转角度就是两个角度的代数和,而代数和是可交换的。
\subsubsection{SO(3)群}
三维空间中绕原点的转动不能仅仅由一个角度定义,还需要一个转动轴$\vec{n}$作为转动方向的表征。我们将一个向量$\vec{x}$沿$\vec{n}$轴(逆时针)转动$\theta$角后得到的向量记作$\hat{R}(\vec{n},\theta) \vec{x}$。两次旋转$\hat{R}(\vec{n_2},\phi) \hat{R}(\vec{n_1},\theta) \vec{x}$事实上可以由一次总的旋转完成(给定一个初末位置,你总能找个一个旋转方法让向量一次就由初位置转到末位置)。仿照之前的乘法定义,我们发现,三维空间中的旋转操作也对乘法运算封闭,且零角度旋转是旋转操作的单位元;每个旋转操作$\hat{R}(\vec{n},\theta)$的逆元就是$\hat{R}(\vec{n},-\theta)$,因此,所有三维空间内的旋转操作构成群,这个群记作$SO(3)$。\par
与$SO(2)$群不同,我们很容易发现,$SO(3)$是非阿贝尔的。对$\vec{n_1} \not= \vec{n_2}$,显然有$\hat{R}(\vec{n_2},\phi) \hat{R}(\vec{n_1},\theta) \not=  \hat{R}(\vec{n_1},\theta) \hat{R}(\vec{n_2},\phi)$。
\subsubsection{SO(n)群、子群}
一般地,$\reals^n$中的旋转操作也构成一个群,这个群称为$SO(n)$群,其中$S$代表special(特殊),$O$代表orthogonal(正交)。其意义将在下一节阐明。一般地,除了之前提到的$n=2$的情况,$SO(n)$群都是非阿贝尔的\footnote{我们不讨论$n=1$的平凡情况}。\par
任何$n-1$维空间中的旋转操作都可以视作$n$维空间中的旋转操作,即$SO(n-1) \subset SO(n)$。一般地,我们有如下子群定义:
\begin{dfn}[子群]
若$(G,\cdot)$为一个群,集合$S \subset G$,称$(S,\cdot)$为$(G,\cdot)$的子群,若$(S,\cdot)$构成群。
\end{dfn}
$SO(n)$中除了这种降维的子群外,也有限制旋转角度的子群,例如$SO(2)$中的子群$D_4 = \{ \hat{R}(\theta) \in SO(2) | \theta = n\cdot 90^\circ,\ n \in \ints \}$表征了$90^\circ$的旋转等,这种限制旋转角度的子群在晶体学中十分有用,因为它表征了晶体的旋转对称性。 

\subsection{旋转群的矩阵表示}
旋转群事实上一个线性变换
\beq
\hat{R}: \reals^n \to \reals^n,\ \vec{x} \to \hat{R} \vec{x}
\eeq
而线性变换总可以用一个矩阵表示。我们先寻找$SO(2)$群的矩阵表示,设$\vec{x} = x \vec{e}_x + y \vec{e}_y = r\cos \phi \vec{e}_x + r\sin \phi \vec{e}_y$,将它逆时针旋转$\theta$角得到的向量是
\bea
\hat{R}(\theta) \vec{x} &= r \cos(\phi + \theta) \vec{e}_x + r \sin(\phi + \theta) \vec{e}_y \\
&= (r\cos \phi \cos \theta - r \sin \phi \sin \theta)\vec{e}_x + (r \sin \phi \cos \theta + r \sin \theta \cos \phi) \vec{e}_y \\
&= (x \cos \theta - y \sin \theta) \vec{e}_x + (y \cos \theta + x \sin \theta) \vec{e}_y \\
&= \begin{pmatrix}
\cos \theta & - \sin \theta \\
\sin \theta & \cos \theta
\end{pmatrix}
\begin{pmatrix}
x \\
y
\end{pmatrix}
\eea
于是旋转操作的矩阵表示
\beq
R(\theta) = \begin{pmatrix}
\cos \theta & - \sin \theta \\
\sin \theta & \cos \theta
\end{pmatrix}
\eeq
上式称为$SO(2)$的代表元,它具有以下性质
\begin{enumerate}
\item $\det R = 1$
\item $RR^T = E$
\end{enumerate}
满足第二条性质的所有矩阵构成$O(2)$群,即二维正交群;在二维正交群上加上第一条限制就成为了二维特殊正交群。可以想象,二维正交群包括了选择操作和镜像反转操作($\vec{x} \to -\vec{x}$),而特殊性的要求剔除了镜像反转操作的存在。

\subsection{洛伦兹群及其矩阵表示}

\subsection{量子力学:$SU(n)$群}

\section{旋转群的李代数表示}
\subsection{李代数}
\subsection{旋转群的李代数}

\section{表示论简介}

\end{document}