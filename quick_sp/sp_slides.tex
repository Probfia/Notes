\documentclass[CJK]{beamer}
\input{macros.tex}
\author{}
\date{}

\usepackage{amsmath}
\usepackage{mhchem}
\usepackage{color}
\usepackage{mathrsfs}
\usepackage{hyperref}
\usepackage{graphicx}

%\cpic{<尺寸>}{<文件名>}}用于生成居中的图片。
\newcommand{\cpic}[2]{
\begin{center}
\includegraphics[scale=#1]{#2}
\end{center}
}


\newcommand{\bit}{\ \mathrm{bit}}
\newcommand{\grandz}{\mathcal{Z}}
\begin{document}

\begin{frame}
 
\begin{center}
\bch
\begin{Large}
{\bf \LARGE 习题}

\skipline
{\small 我没有答案}


\end{Large}
\skipline

高寒   


gaoh26@mail2.sysu.edu.cn 

\ech
\end{center}
\end{frame}


\section{PF \& Classical Stat.}

\begin{frame}
\chtitle{1.0}
\bch
What is the partition function?
\begin{itemize}
\item 玻尔兹曼分布的归一化因子;
\item 均匀分布的矩母函数;
\item 自由能的指数$e^{-\beta F}$;
\item 能量态密度$g(E)$的拉普拉斯变换;
\end{itemize}
\ech
\end{frame}


\begin{frame}
\chtitle{1.1.1}
\bch
Compute the partition function of a quantum harmonic oscillator with frequency $\omega$
and energy levels$$E_{n}=\hbar \omega\left(n+\frac{1}{2}\right) \quad n \in \mathbf{Z}$$
Find the average energy $U$ and entropy $S$ as a function of temperature $T$.\par

\ech
\end{frame}

\begin{frame}
\chtitle{1.1.2}
\bch
Einstein constructed a simple model of a solid as $N$ atoms, each of which vibrates with
the same frequency $\omega$. Treating these vibrations as a harmonic oscillator, show that at
high temperatures, $k_B T \gg \hbar \omega$, the Einstein model correctly predicts the Dulong-Petit
law for the heat capacity of a solid, $$C_V = 3Nk_B$$ At low temperatures, the heat capacity of many solids is experimentally observed to
tend to zero as $c_V \sim T^3$. Was Einstein right about this?
\ech
\end{frame}

\begin{frame}
\chtitle{1.2}
\bch
A particle moving in one dimension has Hamiltonian $$H=\frac{p^{2}}{2 m}+\lambda q^{4}$$ Show that the heat capacity for a gas of $N$ such particles is $C_V = 3Nk_B/4$. Explainwhy the heat capacity is the same regardless of whether the particles are distinguishable or indistinguishable.
\ech
\end{frame}

\begin{frame}
\chtitle{1.3}
\bch
A non-relativistic particle is confined in a spherical container with radii $R$ with potential$$ V(r) = \begin{cases} -h&,\ r<a \\ 0 &,\ a\leq r <R \\ +\infty &,\ r>R\end{cases}$$
Firstly predict the rough behavior of the pressure for a gas of $N$ such particles as a function of $T$, then determine $P(T)$ by statistics.
\ech
\end{frame}

\section{TD \& Chem. Reac.}
\begin{frame}
\chtitle{2.1}
\bch
Show that
\begin{enumerate}
\item $$C_P - C_V = T \left(\frac{\partial V}{\partial T}\right)_P \left(\frac{\partial P}{\partial T}\right)_V = - T \left( \frac{\partial V}{\partial T}\right)_P^2 \left( \frac{\partial P}{\partial V}\right)_T$$
\item $$\pfrac{U}{V}{T} = T \pfrac{P}{T}{V} - P$$
\item $$\pfrac{U}{P}{T} = -T \pfrac{V}{T}{P} - P \pfrac{V}{P}{T}$$
\end{enumerate}
\ech
\end{frame}


\begin{frame}
\chtitle{2.1}
\bch
\begin{enumerate}
\item[4.] $$\pfrac{C_V}{V}{T} = T \left( \frac{\partial^2 P}{\partial T^2} \right)_V$$
\item[5.] $$\pfrac{C_P}{P}{T} = -T \left( \frac{\partial^2 V}{\partial T^2} \right)_P$$
\end{enumerate}
From equations above, show that for any non-ideal gas with constant $C_V$ and $C_P$, the equation of state can be written as $$ (C_P - C_V) T = (P+a)(V+b)$$ where $a$ and $b$ are constants.
\ech
\end{frame}

\begin{frame}
\chtitle{2.2}
\bch
Consider the neutral gas of electrons, protons and Hydrogen. They undergo a reaction as $$ \ce{e- + p+ <=> H}$$ You know from Quantum Mechanics that the Hydrogen atom
has binding energy $E = -I$ ( where $I = 13.6 \mathrm{\ eV}$). Let the number of Hydrogen atoms
be $N_{\rm H} = (1-x) N$ and the e number of electrons and protons be $N_{\rm e} = N_{\rm p} = xN$ with $x\in [0,1]$.
\begin{enumerate}
\item What is the equilibrium condition for the gas?
\item From the condition above, show that $$\frac{x^{2}}{1-x}=\frac{V}{N}\left(\frac{m_{\rm e} m_{\rm p}}{2 \pi \hbar^{2} m_{\rm H}}\right)^{3 / 2}\left(k_{B} T\right)^{3 / 2} e^{-I / k_{B} T}$$
\end{enumerate}
\ech
\end{frame}

\section{Quant. Stat. \& Phase Trans.}
\begin{frame}
\chtitle{3.1}
\bch
A Wigner crystal is a triangular lattice of electrons in a {\bf two dimensional plane}.
The longitudinal vibration modes of this crystal are bosons with dispersion relation $\omega = \sigma \sqrt{k}$. Show that, at low temperatures, these modes provide a contribution to the
heat capacity that scales as $C \sim T^4$.
\ech
\end{frame}

\begin{frame}
\chtitle{3.2}
\bch
Consider a gas of non-interacting ultra-relativistic electrons, whose mass may be
neglected. Show that $U = 3PV$. Show that
at zero temperature $PV^{4/3} = {\rm const}$. Show that at high temperatures $ E = 3Nk_B T$ and
the equation of state coincides with that of a classical ultra-relativistic gas.
\par
What are the results if we count the existance of positrons ($\mu_{\rm e^+} = -\mu_{\rm e^-}$)?
\ech
\end{frame}

\begin{frame}
\chtitle{3.3}
\bch
Consider the free energy $F = a(T)m^2 + b(T)m^4 + c(T)m^6$ where $b(T)<0$ and, for stability, $c(T) > 0$ for all $T$. Sketch the possible behaviours
of the free energy as $a(T)$ varies and, in each case, identify the ground state and
metastable states. Show that the system undergoes a first order phase transition at
some temperature $T_c$. Determine the value $a(T_c)$  and the discontinuity in $m$ at the transition.
\ech
\end{frame}

\begin{frame}
\chtitle{3.4}
\bch
The purpose of this question is to explain why the microwave background radiation
still has a black body spectrum, even though it has not been in thermal equilibrium
with matter since very early in the universe’s history.\par
Consider a region of volume V in the cosmos containing black body radiation of
temperature $T$. Suppose the cosmos expands (slowly) by a scale factor $\alpha$, so that
the wavevector $\vec{k}$ and angular frequency $\omega$ of each electromagnetic radiation mode are
rescaled by $1/\alpha$. Explain why you should expect the mean number of photons in each
mode not to change. Show that the Planck distribution is valid after the expansion
provided the temperature is also rescaled by $1/\alpha$.\par
Verify, from the formula for the entropy of black body radiation, that the entropy in
the expanded volume is the same as the original entropy, thus confirming the adiabatic
character of the expansion.

\ech
\end{frame}

\begin{frame}
\chtitle{3.5}
\bch
Consider a diluted plasma which can be treated as non-relastivistic and non-interacting gas. Each molecule carries a electric charge of $e$. Now apply a magenitic field along the $z$-direction $\vec{B} = B\vec{e}_z$.
\begin{enumerate}
\item In classical picture, the hamitonian for each molecule is $H = \left(\vec{p} + e \vec{A}\right)^2/2m$. Determine the heat energy of such a plasma as a function of $(N,T,V)$.
\item In quantum picture, the energy of a molecule is determined by a integer $n$ and the $z$-component of the momentum $p_z$ as $$E = \left(n + \frac{1}{2}\right) \hbar \omega_c + \frac{p_z^2}{2m}$$, where $n = 0,1,2,\cdots$ and $\omega_c = \frac{eB}{m}$. The degeneracy for a given $n$ is $g(n) = \frac{eB A}{h}$ where $A$ stands for the area of the $xy$ plane. Determine the heat energy of such a plasma as a function of $(N,T,V)$.
\end{enumerate}
\ech
\end{frame}

\end{document}



