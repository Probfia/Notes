\documentclass[CJK]{beamer}
\usepackage{CJKutf8}
\usepackage{beamerthemesplit}
\usetheme{Malmoe}
\useoutertheme[footline=authortitle]{miniframes}
\usepackage{amsmath}
\usepackage{amssymb}
\usepackage{graphicx}
\usepackage{eufrak}
\usepackage{color}
\usepackage{slashed}
\usepackage{simplewick}
\usepackage{tikz}
\usepackage{tcolorbox}
\graphicspath{{../figures/}}
%%figures
\def\lfig#1#2{\includegraphics[width=#1 in]{#2}}
\def\addfig#1#2{\begin{center}\includegraphics[width=#1 in]{#2}\end{center}}
\def\wulian{\includegraphics[width=0.18in]{emoji_wulian.jpg}}
\def\bigwulian{\includegraphics[width=0.35in]{emoji_wulian.jpg}}
\def\bye{\includegraphics[width=0.18in]{emoji_bye.jpg}}
\def\bigbye{\includegraphics[width=0.35in]{emoji_bye.jpg}}
\def\huaixiao{\includegraphics[width=0.18in]{emoji_huaixiao.jpg}}
\def\bighuaixiao{\includegraphics[width=0.35in]{emoji_huaixiao.jpg}}
\def\jianxiao{\includegraphics[width=0.18in]{emoji_jianxiao.jpg}}
\def\bigjianxiao{\includegraphics[width=0.35in]{emoji_jianxiao.jpg}}
%% colors
\def\blacktext#1{{\color{black}#1}}
\def\bluetext#1{{\color{blue}#1}}
\def\redtext#1{{\color{red}#1}}
\def\darkbluetext#1{{\color[rgb]{0,0.2,0.6}#1}}
\def\skybluetext#1{{\color[rgb]{0.2,0.7,1.}#1}}
\def\cyantext#1{{\color[rgb]{0.,0.5,0.5}#1}}
\def\greentext#1{{\color[rgb]{0,0.7,0.1}#1}}
\def\darkgray{\color[rgb]{0.2,0.2,0.2}}
\def\lightgray{\color[rgb]{0.6,0.6,0.6}}
\def\gray{\color[rgb]{0.4,0.4,0.4}}
\def\blue{\color{blue}}
\def\red{\color{red}}
\def\green{\color{green}}
\def\darkgreen{\color[rgb]{0,0.4,0.1}}
\def\darkblue{\color[rgb]{0,0.2,0.6}}
\def\skyblue{\color[rgb]{0.2,0.7,1.}}
%%control
\def\be{\begin{equation}}
\def\ee{\nonumber\end{equation}}
\def\bea{\begin{eqnarray}}
\def\eea{\nonumber\end{eqnarray}}
\def\bch{\begin{CJK}{UTF8}{gbsn}}
\def\ech{\end{CJK}}
\def\bitem{\begin{itemize}}
\def\eitem{\end{itemize}}
\def\bcenter{\begin{center}}
\def\ecenter{\end{center}}
\def\bex{\begin{minipage}{0.2\textwidth}\includegraphics[width=0.6in]{jugelizi.png}\end{minipage}\begin{minipage}{0.76\textwidth}}
\def\eex{\end{minipage}}
\def\chtitle#1{\frametitle{\bch#1\ech}}
\def\bmat#1{\left(\begin{array}{#1}}
\def\emat{\end{array}\right)}
\def\bcase#1{\left\{\begin{array}{#1}}
\def\ecase{\end{array}\right.}
\def\bmini#1{\begin{minipage}{#1\textwidth}}
\def\emini{\end{minipage}}
\def\tbox#1{\begin{tcolorbox}#1\end{tcolorbox}}
\def\pfrac#1#2#3{\left(\frac{\partial #1}{\partial #2}\right)_{#3}}
%%symbols
\def\bropt{\,(\ \ \ )}
\def\sone{$\star$}
\def\stwo{$\star\star$}
\def\sthree{$\star\star\star$}
\def\sfour{$\star\star\star\star$}
\def\sfive{$\star\star\star\star\star$}
\def\rint{{\int_\leftrightarrow}}
\def\roint{{\oint_\leftrightarrow}}
\def\stdHf{{\textit{\r H}_f}}
\def\deltaH{{\Delta \textit{\r H}}}
\def\ii{{\dot{\imath}}}
\def\skipline{{\vskip0.1in}}
\def\skiplines{{\vskip0.2in}}
\def\lagr{{\mathcal{L}}}
\def\hamil{{\mathcal{H}}}
\def\vecv{{\mathbf{v}}}
\def\vecx{{\mathbf{x}}}
\def\vecy{{\mathbf{y}}}
\def\veck{{\mathbf{k}}}
\def\vecp{{\mathbf{p}}}
\def\vecn{{\mathbf{n}}}
\def\vecA{{\mathbf{A}}}
\def\vecP{{\mathbf{P}}}
\def\vecsigma{{\mathbf{\sigma}}}
\def\hatJn{{\hat{J_\vecn}}}
\def\hatJx{{\hat{J_x}}}
\def\hatJy{{\hat{J_y}}}
\def\hatJz{{\hat{J_z}}}
\def\hatj#1{\hat{J_{#1}}}
\def\hatphi{{\hat{\phi}}}
\def\hatq{{\hat{q}}}
\def\hatpi{{\hat{\pi}}}
\def\vel{\upsilon}
\def\Dint{{\mathcal{D}}}
\def\adag{{\hat{a}^\dagger}}
\def\bdag{{\hat{b}^\dagger}}
\def\cdag{{\hat{c}^\dagger}}
\def\ddag{{\hat{d}^\dagger}}
\def\hata{{\hat{a}}}
\def\hatb{{\hat{b}}}
\def\hatc{{\hat{c}}}
\def\hatd{{\hat{d}}}
\def\hatN{{\hat{N}}}
\def\hatH{{\hat{H}}}
\def\hatp{{\hat{p}}}
\def\Fup{{F^{\mu\nu}}}
\def\Fdown{{F_{\mu\nu}}}
\def\newl{\nonumber \\}
\def\vece{\mathrm{e}}
\def\calM{{\mathcal{M}}}
\def\calT{{\mathcal{T}}}
\def\calR{{\mathcal{R}}}
\def\barpsi{\bar{\psi}}
\def\baru{\bar{u}}
\def\barv{\bar{\upsilon}}
\def\qeq{\stackrel{?}{=}}
\def\torder#1{\mathcal{T}\left(#1\right)}
\def\rorder#1{\mathcal{R}\left(#1\right)}
\def\contr#1#2{\contraction{}{#1}{}{#2}#1#2}
\def\trof#1{\mathrm{Tr}\left(#1\right)}
\def\trace{\mathrm{Tr}}
\def\comm#1{\ \ \ \left(\mathrm{used}\ #1\right)}
\def\tcomm#1{\ \ \ (\text{#1})}
\def\slp{\slashed{p}}
\def\slk{\slashed{k}}
\def\calp{{\mathfrak{p}}}
\def\veccalp{\mathbf{\mathfrak{p}}}
\def\Tthree{T_{\tiny \textcircled{3}}}
\def\pthree{p_{\tiny \textcircled{3}}}
\def\dbar{{\,\mathchar'26\mkern-12mu d}}
\def\erf{\mathrm{erf}}
\def\const{\mathrm{constant}}
\def\pheat{\pfrac p{\ln T}V}
\def\vheat{\pfrac V{\ln T}p}
%%units
\def\fdeg{{^\circ \mathrm{F}}}
\def\cdeg{^\circ \mathrm{C}}
\def\atm{\,\mathrm{atm}}
\def\angstrom{\,\text{\AA}}
\def\SIL{\,\mathrm{L}}
\def\SIkm{\,\mathrm{km}}
\def\SIyr{\,\mathrm{yr}}
\def\SIGyr{\,\mathrm{Gyr}}
\def\SIV{\,\mathrm{V}}
\def\SImV{\,\mathrm{mV}}
\def\SIeV{\,\mathrm{eV}}
\def\SIkeV{\,\mathrm{keV}}
\def\SIMeV{\,\mathrm{MeV}}
\def\SIGeV{\,\mathrm{GeV}}
\def\SIcal{\,\mathrm{cal}}
\def\SIkcal{\,\mathrm{kcal}}
\def\SImol{\,\mathrm{mol}}
\def\SIN{\,\mathrm{N}}
\def\SIHz{\,\mathrm{Hz}}
\def\SIm{\,\mathrm{m}}
\def\SIcm{\,\mathrm{cm}}
\def\SIfm{\,\mathrm{fm}}
\def\SImm{\,\mathrm{mm}}
\def\SInm{\,\mathrm{nm}}
\def\SImum{\,\mathrm{\mu m}}
\def\SIJ{\,\mathrm{J}}
\def\SIW{\,\mathrm{W}}
\def\SIkJ{\,\mathrm{kJ}}
\def\SIs{\,\mathrm{s}}
\def\SIkg{\,\mathrm{kg}}
\def\SIg{\,\mathrm{g}}
\def\SIK{\,\mathrm{K}}
\def\SImmHg{\,\mathrm{mmHg}}
\def\SIPa{\,\mathrm{Pa}}

\author{}
\date{}

\usepackage{amsmath}
\usepackage{mhchem}
\usepackage{color}
\usepackage{mathrsfs}
\usepackage{hyperref}
\usepackage{graphicx}

%\cpic{<尺寸>}{<文件名>}}用于生成居中的图片。
\newcommand{\cpic}[2]{
\begin{center}
\includegraphics[scale=#1]{#2}
\end{center}
}


\newcommand{\bit}{\ \mathrm{bit}}
\newcommand{\grandz}{\mathcal{Z}}
\begin{document}

\begin{frame}
 
\begin{center}
\bch
\begin{Large}
{\bf \LARGE 习题}

\skipline
{\small 我没有答案}


\end{Large}
\skipline

高寒   


gaoh26@mail2.sysu.edu.cn 

\ech
\end{center}
\end{frame}


\section{Lec 1}

\begin{frame}
\chtitle{1.1.1}
\bch
Compute the partition function of a quantum harmonic oscillator with frequency $\omega$
and energy levels$$E_{n}=\hbar \omega\left(n+\frac{1}{2}\right) \quad n \in \mathbf{Z}$$
Find the average energy $U$ and entropy $S$ as a function of temperature $T$.\par

\ech
\end{frame}

\begin{frame}
\chtitle{1.1.2}
\bch
Einstein constructed a simple model of a solid as $N$ atoms, each of which vibrates with
the same frequency $\omega$. Treating these vibrations as a harmonic oscillator, show that at
high temperatures, $k_B T \gg \hbar \omega$, the Einstein model correctly predicts the Dulong-Petit
law for the heat capacity of a solid, $$C_V = 3Nk_B$$ At low temperatures, the heat capacity of many solids is experimentally observed to
tend to zero as $c_V \sim T^3$. Was Einstein right about this?
\ech
\end{frame}

\begin{frame}
\chtitle{1.2}
\bch
A particle moving in one dimension has Hamiltonian $$H=\frac{p^{2}}{2 m}+\lambda q^{4}$$ Show that the heat capacity for a gas of $N$ such particles is $C_V = 3Nk_B/4$. Explainwhy the heat capacity is the same regardless of whether the particles are distinguishable or indistinguishable.
\ech
\end{frame}

\begin{frame}
\chtitle{1.3}
\bch
A non-relativistic particle is confined in a spherical container with radii $R$ with potential$$ V(r) = \begin{cases} -h&,\ r<a \\ 0 &,\ a\leq r <R \\ +\infty &,\ r>R\end{cases}$$
Firstly predict the rough behavior of the pressure for a gas of $N$ such particles as a function of $T$, then determine $P(T)$ by statistics.
\ech
\end{frame}

\section{Lec 2}
\begin{frame}
\chtitle{2.1}
\bch
Show that
\begin{enumerate}
\item $$C_P - C_V = T \left(\frac{\partial V}{\partial T}\right)_P \left(\frac{\partial P}{\partial T}\right)_V = - T \left( \frac{\partial V}{\partial T}\right)_P^2 \left( \frac{\partial P}{\partial V}\right)_T$$
\item $$\pfrac{U}{V}{T} = T \pfrac{P}{T}{V} - P$$
\item $$\pfrac{U}{P}{T} = -T \pfrac{V}{T}{P} - P \pfrac{V}{P}{T}$$
\end{enumerate}
\ech
\end{frame}


\begin{frame}
\chtitle{2.1}
\bch
\begin{enumerate}
\item[4.] $$\pfrac{C_V}{V}{T} = T \left( \frac{\partial^2 P}{\partial T^2} \right)_V$$
\item[5.] $$\pfrac{C_P}{P}{T} = -T \left( \frac{\partial^2 V}{\partial T^2} \right)_P$$
\end{enumerate}
From equations above, show that for any non-ideal gas with constant $C_V$ and $C_P$, the equation of state can be written as $$ (C_P - C_V) T = (P+a)(V+b)$$ where $a$ and $b$ are constants.
\ech
\end{frame}

\begin{frame}
\chtitle{2.2}
\bch
Consider the neutral gas of electrons, protons and Hydrogen. They undergo a reaction as $$ \ce{e- + p+ <=> H}$$ You know from Quantum Mechanics that the Hydrogen atom
has binding energy $E = -I$ ( where $I = 13.6 \mathrm{\ eV}$). Let the number of Hydrogen atoms
be $N_{\rm H} = (1-x) N$ and the e number of electrons and protons be $N_{\rm e} = N_{\rm p} = xN$ with $x\in [0,1]$.
\begin{enumerate}
\item What is the equilibrium condition for the gas?
\item From the condition above, show that $$\frac{x^{2}}{1-x}=\frac{V}{N}\left(\frac{m_{\rm e} m_{\rm p}}{2 \pi \hbar^{2} m_{\rm H}}\right)^{3 / 2}\left(k_{B} T\right)^{3 / 2} e^{-I / k_{B} T}$$
\end{enumerate}
\ech
\end{frame}


\end{document}



